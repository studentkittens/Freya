\section{Aufbau des Clients}

Aus den oben genannten Anforderungen kann eine grobe Architektur abgeleitet werden:

%Hier ein erstes Klassendiagramm zu  
%     * BaseClient
%     * Client 
%     * Connection 
%     * Listener
%     * NotifyData
%bzw. deren Verbindung

\subsection{Hauptklassen}

\subsubsection{BaseClient}
\begin{itemize}
    \item Kann nicht selbst instanziert werden.
    \item Verwaltet connect / disconnect und reconnect vorgänge
    \item Bietet Funktionen zum einfachen verlassen und eintreten des idlemodes an 
    \item Implementiert keine konkreten Kommandos die er an den server schicken kann
    \item Geht die verbindung verloren (ohne dass \emph{disconnect()} explizit aufgerufen wurde),
        so versucht er periodisch sich zu reconnecten.
\end{itemize}
Er soll mindestens folgende public Methoden bieten:

\begin{verbatim}
        Connection& get_connection(void);
\end{verbatim}

\begin{verbatim}
        bool is_connected(void);
\end{verbatim}
Die get\_status() Funktion soll den letzten aktuellen MPD::Status zurückliefern,
oder NULL falls nicht verbunden. 
\begin{verbatim}        
        Status * get_status(void);
\end{verbatim}

Die folgenden Funktion rufen einfach die entsprechenden Funktionen von MPD::Listener auf,
prüfen aber zusätzlich noch ob eine Verbindung besteht:
\begin{verbatim}
        EventNotifier& signal_client_update(void);
        ConnectionNotifier& signal_connection_change(void);

        void force_update(void);
        void go_idle(void);
        void go_busy(void);
\end{verbatim}

\subsubsection{Listener}
\begin{itemize}
    \item Verwaltet das ein-(enter()) und austreten (leave()) aus dem Idlemode
    \item Parst die Responseliste (also changed: player)
    \item Verfügt über ein ,,EventNotifer'' (ein sigc::signal)
        Module können sich über connect() registrieren,
        bemerkt der Listener events so ruft er emit() auf dem signal auf
        und teilt allen anderen Modulen so mit welche events geschehen sind.
\end{itemize}

Es folgt eine Liste von Funktionen die der Listener mindestens haben soll.
\\
enter(), leave() wurden oben bereits erwähnt. 
is\_idling() sollte selbsterklärend sein.
\begin{verbatim}
       bool enter(void);
       void leave(void);
       bool is_idling(void);
\end{verbatim}

Es soll zudem eine force\_update() Funktion geben die ,,künstlich'' alle Events auslöst.
\begin{verbatim}
       void force_update(void);
\end{verbatim}

\subsubsection{Connection}
\begin{itemize}
    \item Ein Wrapper um die mpd\_connection Struktur von libmpdclient
    \item Ruft beim Verbindungsvorgang letzendlich mpd\_connection\_new() auf 
    \item Bietet eine Schnittstelle um sich über Fehler informieren zu lassen (signal\_error())
    \item Bietet eine get\_connection() methode die bei jedem aufruf prüft ob fehler passiert sind
        In diesem Falle versucht MPD::Connection den Fehler zu bereinigen (falls ein nicht fataler Fehler war).
        Anschließend benachrichtigt MPD::Connection alle module die sich vorher über signal\_error() 
        registriert haben (wie der BaseClient es beispielsweise mit handle\_error() tut)
\end{itemize}
Es folgt eine Liste von Funktionen die mindestens vorhanden sein sollten.
Ein boolean-Rückgabewert von true zeigt stets Erfolg an.
\\
\begin{verbatim}
    bool is_connected(void);
    bool connect(void);
    bool disconnect(void);
    mpd_connection * get_connection(void);
\end{verbatim}

connect() soll die eigentliche Verbindung herstellen, disconnect() löscht die Verbindung wieder.
get\_connection() liefert einen Pointer auf die darunter liegende C-Struktur. Alle 4 Funktionen prüfen 
zudem intern bereits auf Fehler.
\\

\begin{verbatim}
    typedef sigc::signal<void, bool,mpd_error> ErrorNotify;
    typedef sigc::signal<void,bool,bool> ConnectionNotifier;
    
    ErrorNotify& signal_error(void);
    ConnectionNotifier& signal_connection_change(void)
\end{verbatim}
Auf den Rückgabewert dieser Funktionen kann sigc::signal::connect() aufgerufen werden, 
um einen Funktionspointer zu registrieren der aufgerufen wird sobald ein Fehler eintritt,
bzw. sich die Verbindung ändert. Die Prototypen sollen jeweils sein:
\begin{verbatim}
    void error_handler(bool is_fatal, mpd_error err_code);
    void conn_change_handler(bool server_changed, bool is_connected); 
\end{verbatim} 
Die Prototypen entsprechen den Templateargumenten in den typedefs.

\begin{verbatim}
    bool check_error(void);
\end{verbatim}
libmpdclient verbietet es weitere Kommandos an den Server zu senden wenn vorher ein Fehler passiert ist.
Fehler müssen zuerst mit \emph{mpd\_connection\_clear\_error()} ,,bereinigt'' werden. 
Dies tut check\_error(). Die Funktion wird normal nicht selbst aufgerufen, da sie von allen anderen Funktionen der Klasse
implizit aufgerufen wird. Ist ein Fehler passiert so werden alle Klienten die sich zuvor
mit signal\_error() registriert haben benachrichtigt. 


\subsubsection{Client}
\begin{itemize}
    \item Der Client erbt von BaseClient und implementiert konkrete Commandos wie ,,play'',,,random'' etc.
    \item Er bietet zudem Schnittstellen zur Befüllung der Datenbank, der Queue und des Playlistmanagers
    \item Er bietet die Methoden connect() und disconnect() 
    \item Ist in der config ,,settings.connection.autoconnect'' gesetzt so connected er sich automatisch.
    \item Er bietet zudem eine schnittstelle um sich beim listener zu registrieren und im falle von 
        änderungen des connection zustands benachrichtigt zu werden.
\end{itemize}

connect() und disconnect() stellen die öffentliche Schnittstelle zum Verbinden dar.
Sie rufen intern lediglich \_\_connect() bzw. \_\_disconnect() von MPD::BaseClient auf.
\begin{verbatim}
    void connect(void);
    void disconnect(void);
\end{verbatim}

Test

\begin{verbatim}
    // 
    bool playback_next(void);
    bool playback_prev(void);
    bool playback_stop(void);
    bool playback_play(void);
    void playback_crossfade(unsigned seconds);
    
    bool playback_pause(void);
    void playback_seek(unsigned song_id, unsigned abs_time);
    
    // 4 Funktionen um jeweils random, consume, repeat 
    // und single-modi umzuschalten
    void toggle_random(void);
    void toggle_consume(void);
    void toggle_repeat(void);
    void toggle_single(void);

    void play_song_at_id(unsigned song_id);
    void playlist_save(const char * name);
    void queue_add(const char * url);
    void queue_clear(void);
    void queue_delete(unsigned pos);

    // Sendet MPD Server Hinweis um DB zu aktualisieren
    unsigned database_update(const char * path);

    // Sendet MPD Server Hinweis um DB neu einzulesen (teuer)
    unsigned database_rescan(const char * path);

    // Volume von 0-100
    void set_volume(unsigned vol);
    
    // Folgende Funktionen sollen von AbstractItemGenerator
    // voll implementiert werden.
    void fill_queue(AbstractItemlist& data_model);
    void fill_queue_changes(AbstractItemlist& data_model,
                            unsigned last_version,
                            unsigned& first_pos);
    void fill_playlists(AbstractItemlist& data_model);
    void fill_outputs(AbstractItemlist& data_model);
    void fill_filelist(AbstractItemlist& data_model, const char * path);
\end{verbatim}

\subsubsection{NotifyData}
\begin{itemize}
    \item Speichert den Status, den aktuellen Song und die aktuelle Datenbankstatistik
    \item Der Listener...
        \begin{itemize}
            \item instanziert NotifyData im Konstruktor
            \item sagt NotifyData wann er sich updaten soll (update\_all())
            \item gibt bei einem Event eine Referenz auf NotifyData an alle registrierten Module weiter,
                damit diese konkrete Informationen beziehen können.
        \end{itemize}
\end{itemize}

\begin{verbatim}
        Status& get_status(void);
        Statistics& get_statistics(void);
        Song * get_song(void);
        Song * get_next_song(void);
\end{verbatim}

Test
\begin{verbatim}
        /**
         * @brief Update internal client state
         */
        void update_all(unsigned event = UINT_MAX);
\end{verbatim}


\subsection{Weitere Klassen}
Desweiteren gibt es einige weitere Klassen die am Rande eine Rolle spielen,
und meist Objektorientierte Wrapperklassen für die C-Strukturen von libmpdclient bereitstellen.

\subsubsection{Song}

Die Song Klasse für Wrapper für mpd\_song Struktur und die dazugehörigen Klassen (libmpdclient).
Soll alle Funktionen von libmpdclient \footnote{http://www.musicpd.org/doc/libmpdclient/song\_8h.html} anbieten,
diese werden hier nur aufgelistet aber nicht erklärt da sie genau wie ihre Vorbilder funktionieren:

\begin{verbatim}
    const char * get_path(void);
    const char * get_tag(enum mpd_tag_type type, unsigned idx);
    unsigned get_duration(void);
    time_t get_last_modified(void);
    void set_pos(unsigned pos);
    unsigned get_pos(void);
    unsigned get_id(void);
\end{verbatim}

MPD::Song soll zudem eine Funktion bieten um die Metadaten des Songs in einer printf änhlichen Art als String zurückzuliefern:
\begin{verbatim}
    Glib::ustring song_format(const char* format, bool markup=true);
\end{verbatim}

Ein beispielhafter Aufruf:
\begin{verbatim}
    SomeSong.song_format("Artist is by ${artist}") 
\end{verbatim}

Die folgenden Tagarten sollen dabei unterstützt werden (sie spiegeln in etwa die mpd\_tag\_type Enumeration von libmpdclient wieder)
Foldende Tags sollen daher unterstützt werden: \it artist, title, album, track, name, data, album\_artist, genre, composer, performer, comment, disc\rm.
Ist ein Escapestring nicht bekannt, so wird er nicht escaped. Ist der tag nicht vorhanden soll mit "unknown" escaped werden.


\subsubsection{Directory}
Die Directory Klasse ist Wrapper für mpd\_directory C-Strukutr. Diese wird als Anzeige für ein Verzeichniss benutzt,
jedoch nicht als Container für andere Elemente.

Entsprechend implementiert bietet MPD::Directory nur:
\begin{verbatim}
    void get_path(void);
\end{verbatim}

Dies ist von der AbstractComposite vorgegeben.

\newpage
\subsubsection{Statistics}
Die Statistics Klasse ist Wrapper für mpd\_stats, implementiert gemäß
\\http://www.musicpd.org/doc/libmpdclient/stats\_8h.html
folgende Funktionen:
\begin{verbatim}
    unsigned get_number_of_artists(void);
    unsigned get_number_of_albums(void);
    unsigned get_number_of_songs(void);
    unsigned long get_uptime(void);
    unsigned long get_db_update_time(void);
    unsigned long get_play_time(void);
    unsigned long get_db_play_time(void);
\end{verbatim}


\subsubsection{Playlist}
Die Playlist Klasse ist Wrapper für die mpd\_playlist Struktur, implementiert von http://www.musicpd.org/doc/libmpdclient/playlist\_8h.html folgende Funktionen:
\begin{verbatim}
    const char * get_path(void);
    time_t get_last_modified(void);
\end{verbatim}

Bietet desweiteren funktionen zum:
Entfernen der Playlist vom Server (Das Playlistobjekt ist danach invalid):
\begin{verbatim}
    void remove(void);
\end{verbatim}

Laden der Playlist in die Queue:
\begin{verbatim}
    void load(void);
\end{verbatim}

Umbennen der Playlist:
\begin{verbatim}
    void rename(const char * new\_name);
\end{verbatim}

Hinzufügen von Songs zur Playlist:
\begin{verbatim}
    void add_song(const char * uri);
    void add_song(MPD::Song& song);
\end{verbatim}

Die genannten Funktionen benötigen müssen den idlemode verlassen können,
daher leitet MPD::Playlist von AbstractClientExtension ab.

\subsubsection{AudioOutput}
Die AudioOutput Klasse ist ein Wrapper für mpd\_output, implementiert von http://www.musicpd.org/doc/libmpdclient/output\_8h.html folgende Funktionen:
\begin{verbatim}
    unsigned get_id(void);
    const char * get_name(void);
    bool get_enabled(void);
\end{verbatim}

Bietet desweiteren funktionen zum:
\begin{itemize}
    \item Enablen des Ausgabegerätes:
        \begin{verbatim}
            bool enable(void);
        \end{verbatim}
    \item Disablen des Ausgabegerätes:
        \begin{verbatim}
            bool disable(void);
        \end{verbatim}
\end{itemize}


Die genannten Funktionen benötigen müssen den idlemode verlassen können,
daher leitet MPD::AudioOutput von AbstractClientExtension ab. 

\subsection{Abstrakte Klassen}
\subsubsection{AbstractClientExtension}
Diese abstrakte Klasse erlaubt abgeleiteten Klasse ähnlich zum BaseClient eigene Kommandos zu implementieren.
Wird von MPD::Playlist und MPD::AudioOutput benutzt
%TODO 


\subsubsection{AbstractClientUser}
\begin{itemize}
    \item Verwaltet einen Pointer auf die MPD::Client Klasse,
        so dass der Anwender der Klasse dies nicht selbst tun muss.
    \item Leitet man ab so müssen folgenden Methoden implementiert werden:
        \begin{verbatim}
            void on_client_update(enum mpd_idle event, MPD::NotifyData& data);
        \end{verbatim}  

        Wird aufgerufen sobald der Listener eine Änderunge feststellt,
        siehe weiter unten "Interaktion des Clients mit anderen Modulen" für eine genauere Erklärung.
        \begin{verbatim}
            void on_connection_change(bool server_changed, bool is_connected);
        \end{verbatim}

        Wird aufgerufen sobald sich der verbunden/getrennt hat. Im ersten Fall
        ist is\_connected true, im anderen false. Sollte sich der Client verbunden haben,
        und der neue Server entspricht nicht mehr dem neuen so ist auch server\_changed true.
        Dies ist automatisch wahr beim ersten Start.
        Diese werden automatisch durch Ableiten von AbstractClientUser registriert.
        Weiterhin können alle Klassen über den mp\_Client Pointer auf den Client zugreifen.
\end{itemize}


\subsubsection{AbstractItemlist}
Für bestimmte Client funktionen muss eine Nutzerklasse von AbstractItemlist ableiten.
Leitet man ab so muss die Methode add\_item(AbstractComposite * data) implementiert werden. 
Je nach Bedarf kann über \verb+static_cast<Zieltyp*>(data)+ der entsprechende Datentyp rausgecasted werden.
Beim Aufruf von MPD::Client::fill\_queue ruft der Client die add\_item methode für jeden 
song den er vom server bekommt auf. Die ableitende Klasse kann diese dann verarbeiten.

Dadurch werden alle Methoden von AbstractItemGenerator (bzw. die Klassen die davon ableiten) benutzbar:
\begin{itemize}
    \item fill\_queue
    \item fill\_queue\_changes
    \item fill\_playlists
    \item fill\_ouputs
    \item fill\_filelist
\end{itemize} 

%<Klassendiagramm, bzw. Klassen die es verwenden von Doxygen nehmen>

\subsubsection{AbstractItemGenerator}
Lässt ableitende Klasse folgende Methoden implementieren:
Jede dieser Methoden ruft MPD::Playlist add\_item() von AbstractItemlist auf um ihre Resultate weiterzugeben.

%<Sequenzdiagramm>   
Holt alle Songs der aktuellen Queue.
\begin{verbatim}            
    void fill_queue(AbstractItemlist& data_model);
\end{verbatim}

Holt alle geänderten Songs in der Queue seit der Version last\_version. Die Position des ersten geänderten Songs wird in first\_pos gespeichert. 
\begin{verbatim}
    void fill_queue_changes(AbstractItemlist& data_model, unsigned last_version, unsigned& first_pos);
\end{verbatim}

Holt alle gespeicherten Playlisten vom Server.
\begin{verbatim}              
    void fill_playlists(AbstractItemlist& data_model);
\end{verbatim}

Holt alle Audio Outputs vom Server.
\begin{verbatim}
    void fill\_outputs(AbstractItemlist& data\_model);
\end{verbatim}

Holt alle Songs und Directories aus der Datenbank im Pfad path (nicht rekursiv!)              
\begin{verbatim}
    void fill_filelist(AbstractItemlist& data_model, const char * path);
\end{verbatim}

%<Klassendiagramm, bzw. Klassen die es verwenden von Doxygen nehmen>

%-------------------------------------------

\subsubsection{AbstractComposite}
Vereinheitlicht Zugriff auf Komponenten verschiedenen Types.
Die abstrakte Klasse zwingt seine Kinder dazu eine \emph{get\_path()} zu implementieren die die Lage im virtuellen Filesystem des Servers angibt.
Der Hauptnutznieser dieser Klasse ist der Databasebrowser, bzw. den dahinter gelagerten Cache Songs und Verzeichnisse gleich zu behandeln.

Die erbende Klasse muss im Konstruktor angeben ob es sich bei der Klasse um ein ,,File'' (\emph{true} für MPD::Song) oder um einen ,,Container'' (\emph{false} für MPD::Directory) handelt.
Diese ,,is\_leaf'' Eigenschaft kann später mit der Funktion \emph{is\_leaf()} abgefragt werden.

% <Klassendiagramm für alle Klassen die von AbstractComposite erben, siehe Doxygen>

%=============================================

%% Übergang zu GUI Zeugs..

\section{Interaktion des Clients mit anderen Modulen}
\begin{itemize}
\item Die meisten GUI Klassen leiten von AbstractClientUser ab und speichern daher eine Referenz auf eine Instanz von MPD::Client
        Sie können daher Funktionen wie queue\_add() direkt aufrufen.
\item AbstractClientUser zwingt die abgeleitenden Klassen folgende Funktionen zu implementieren: 
\begin{verbatim} 
   void on_client_update(mpd_idle event, MPD::NotifyData& data)
   void on_connection_change(bool server_changed, bool is_connected)
\end{verbatim}

1) wird aufgerufen sobald der Listener ein Event festgestellt hat. Für jedes eingetretene Event wird 1)
   einmal aufgerufen. 'event' ist dabei eine Enumeration aller möglichen Events, die von libmpdclient 
   vorgegeben werden. \verb+(Siehe auch http://www.musicpd.org/doc/libmpdclient/idle\_8h.html#a3378f7a24c714d7cb1058232330d7a1c)+
   ,,data'' ist eine Referenz auf eine Instanz von MPD::NotifyData. Die benutzenden Klassen können folgenden Funktionen so
   bei Events sofort die aktuellen Änderungen auslesen. 
   \begin{itemize} 
     \item get\_status() gibt den aktuellen MPD::Status
     \item get\_song() gibt den aktuellen MPD::Song
     \item get\_statistics() gibt die aktuellen MPD::Statistics
   \end{itemize} 

2) on\_connection\_change wird vom Client aufgerufen sobald die Verbindung verloren geht.
   Dabei zeigt der übergebene boolean Wert ,,is\_connected'' an ob man connected wurde, oder disconnected wurde.
   ,,server\_changed'' soll dann anzeigen ob der Server derselbe ist beim zuvor geschehenen Connectvorgang.
   Dies ist beim ersten Start stets wahr. ,,server+\_changed'' kann nicht wahr sein wenn ,,is\_connected'' falsch ist.
\item Ableitung von den oben beschriebenen abstrakten Klassen AbstractItemlist und AbstractFilebrowser, um alle Funktionen von AbstractItemGenerator nutzen zu können  
\end{itemize}