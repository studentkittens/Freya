\documentclass[11pt]{scrreprt}

\usepackage[utf8]{inputenc}
\usepackage{ngerman}
%\usepackage{fullpage} % kleinere Ränder

\usepackage{comment} % für größere comments: \begin{comment} ... \end{comment}

% *** für eingefügte (pdf-)Grafiken
\usepackage[pdftex]{graphicx} 
\pdfminorversion=6
% ***

\usepackage{enumerate} %für geschachtelte Aufzählungen
\usepackage{tabularx}
\usepackage{textcomp}


% *** java listings
\usepackage{listings} 
\usepackage{courier} % courier schrift
\lstset{numbers=left, numberstyle=\tiny, basicstyle=\ttfamily  ,numbersep=1pt, tabsize=1}
\lstset{frame=single, framesep=1pt,framerule=0.5pt,xleftmargin=1.0pt,showstringspaces=false}
    

% ***

\usepackage{amsmath} %Matheformeln usw.
\usepackage{amssymb} %mathfrak

\usepackage[bookmarks=true]{hyperref} % hyperrefs aktivieren
\setcounter{secnumdepth}{4} %Numerierung bis Tiefe 3, also ab \paragraph ohne
\setcounter{tocdepth}{4}

%kitteh stuff
\linespread{1.25}


%*** title usw.
\title{MPD-Client}
\subtitle{Teil der Software Engineering II Studienarbeit WS 2011/2012, Inf 3}
\author{
Christopher Pahl,\\
Christoph Piechula,\\
Eduard Schneider,\\
und Marc Tigges}
\date{\today}

%***
%newcommands
%\newcommand{\neuesKommando}{Was zu tun ist}

\begin{document}
\maketitle
\tableofcontents
%\part{welcher Teil}
\chapter{Einleitung}
Ziel dieser Studienarbeit ist die vollständige Bearbeitung einer vorgegebenen Aufgabenstellung nach einem selbst gewählten Vorgehensmodell. Die Aufgabenstellung schreibt vor, sich in einer Gruppe zusammen zu finden und gemeinsam ein Software-Projekt zu bearbeiten und dabei strukturiert und professionell vorzugehen.\\

\begin{quote}
\section{Rahmenbedingungen}

\renewcommand{\labelitemi}{•}
\begin{itemize}
	\item Persistente Datenspeicherung (Datei oder Datenbank)
	\item Netzwerk Programmierung (Verteilte Architektur z.B. Client-Server)
	\item Grafisches-User-Interface (Swing, Web-basiert,...)
\end{itemize}

\section{Prozess-Anforderungen}

\begin{itemize}
	\item Dokumentation aller Phasen(Analyse bis Testen)
	\item Auswahl eines konkreten Prozessmodells (Begründung der Wahl)
	\item Erstellung der Dokumente und UML-Diagramme (Freie Wahl der Werkzeuge)
	\item Fertige Implementierung (Es kann mehr spezifiziert sein als implementiert)
	\item Spezifikation von Testszenarien (Und Beleg der erfolgreichen Ausführung)
	\item Lauffähiges System 
\end{itemize}

\footnote{Folie Anforderungen, Autor Prof. Dr. Philipp Schaible, WS 2011/2012, Inf 3}
\end{quote}

Diese Arbeit ist wichtig, um den Studenten zu zeigen, wie man in einem Team zusammenarbeitet und nach Software-Engineering-Methoden qualitativ hochwertige Software erstellt.
Es geht im Folgenden um einen Music-Player-Daemon-Client (Näheres bitte der Definition entnehmen). Dieses Thema wird behandelt, da es alle Rahmenbedingungen abdeckt und im Interesse der Autoren liegt. Die Besonderheit liegt darin, dass sich diese Software nach Fertigstellung auch wirklich anwenden lässt. Ziel ist die Erweiterung der Fähigkeiten im Bereich des Software Engineering sowie das Erlernen von Methoden für wissenschaftliches Arbeiten.

\chapter{Wasserfallmodell mit Rücksprung}
\section{Definition}
\begin{quote}
Das Wasserfallmodell ist ein lineares (nicht iteratives) vorgehensmodell in der Softwareentwicklung, bei dem der
 Sotwareentwicklungsprozess in Phasen organisiert wird. Dabei gehen die Phasenergebnisse wie bei einem Wasserfall immmer
als bindende Vorgaben für die nächsttiefere Phase ein.\ \\ \\
Im Wasserfallmodell hat jede Phase vordefinierte Start- und Endpunkte mit eindeutig definierten Ergebnissen.
In Meilensteinsitzungen am jeweiligen Phasenende werden die Ergenisdokumente verabschiedet. Zu den wichtigsten
Dokumenten zählen dabei das Lastenheft sowie das Pflichenheft. In der betrieblichen Praxis gibt es viele Varianten
des reinen Modells. Es ist aber das traditionell am weitesten verbreitete Vorgehensmodell.\ \\ \\
Der Name \"Wasserfall\" kommt von der häufig gewählten grafischen Darstellung der fünf bis sechs als Kaskade
angeordneten Phasen. Ein erweitertes Wasserfallmodell mit Rücksprungmöglichkeiten (gestrichelt).\ \\ \\
Erweiterungen des einfachen Modells (Wasserfallmodell mit Rücksprung) führen iterative Aspekte ein und erlauben
ein schrittweises \"Aufwärtslaufen\" der Kaskade, sofern in der aktuellen Phase etwas schieflaufen sollte,
um den Fehler auf der nächsthöheren Stufe beheben zu können.
\footnote{Zitat aus:  http://de.wikipedia.org/wiki/Wasserfallmodell}
\end{quote}
\begin{figure}[h]
\centering
\includegraphics[scale=0.35]{567px-Wasserfallmodell.png}
\end{figure}
\footnote{Wasserfallmodell mit Rücksprung, \\ Bild-Quelle: http://upload.wikimedia.org/wikipedia/commons/thumb/e/e5/Wasserfallmodell.svg/567px-Wasserfallmodell.svg.png}
\section{Warum dieses Modell?}
Wir haben uns für das Wasserfallmodell mit Rücksprung entschieden, weil dieses Modell alle Phasen der 
Entwicklung klar abgrenzt und sich optimal auf einen professionellen Softwareentwicklungsvorgang
abbilden lässt. Dieses Modell ermöglicht eine klare Planung und Kontrolle unseres Softwareprojekts,
da die Anforderungen stets die gleichen bleiben und der Umfang einigermaßen gut abschätzbar ist.
Für die erweiterte Version dieses Modells, nämlich mit Rücksprung, haben wir uns entschieden, um ein 
paar Nachteile dieses Modells auszuhebeln. Beispielsweise sind die klar voneinander abgegrenzten Phasen
in der Realität oft nicht umsetzbar. Des weiteren sind wir somit flexibler gegenüber Änderungen.
\section{Tatsächliche Umsetzung}
Im Laufe der Entwicklung mussten wir feststellen, dass das Wasserfallmodell als Vorgehensweise für unser
Projekt doch nicht so gut geeignet war wie wir erwarteten. Da das Projekt eine viel zu komplexe Infrastruktur
besitzt, die wir vorher unmöglich abschätzen konnten, war es unmöglich zu planen, welcher Arbeitsaufwand nötig ist.
Somit konnten keine brauchbaren Entwürfe der Software gemacht werden, da man sich manche Funktionen des MPD
völlig anders vorgestellt hatte, als sie in Wirklichkeit funktionierten. Daher mussten wir, nachdem
das Lasten- und Pflichtenheft fertiggestellt waren (welche unabhängig von der Infrastruktur sind)), auf agile
Entwicklungsmethoden umschwänken. Aufgrund der hohen Komplexität der MPD Libraries war es uns nicht möglich
sich einen Überblick über interne Abläufe zu verschaffen so mussten wir parallel zu den eigentlichen Planungen
erste Testanwendungen schrieben um sich mit der Materie vertraut zu machen.
Als Beispiel sei hier der Testclient genannt, der letzlich auch die Grundlage für die heutige
Architektur darstellt, bzw. die Grundlage aus der sie entstanden ist.

(TODO: Spiralmodell?)
\chapter{Richtlinien}
\section{Programmierrichtlinien}
\subsection{Begründung}
\section{Toolauswahl}
\subsection{Begründung}

\chapter{Definition}
Der MPD ist eine Client/Server-Architektur, in der die Clients und Server (MPD ist der Server)
über ein Netzwerk interagieren. MPD ist also nur die Hälfte der Gleichung. Zur Nutzung von 
MPD, muss ein MPD-Client (auch bekannt als MPD-Schnittstelle) installiert werden.
\section{Definition des MPD}
\begin{quote}
Der Music Player Daemon (kurz MPD) ist ein Unix-Systemdienst, der das Abspielen von Musik auf 
einem Computer ermöglicht. Er unterscheidet sich von gewöhnlichen Musik-Abspielprogrammen dadurch, 
dass eine strikte Trennung von Benutzeroberfläche und Programmkern vorliegt. Dadurch ist die 
grafische Benutzeroberfläche auswechselbar und auch eine Fernsteuerung des Programms über das 
Netzwerk möglich. Die Schnittstelle zwischen Client und Server ist dabei offen dokumentiert und 
der Music Player Daemon selbst freie und quelloffene Software.\ \\ \\
Der MPD kann wegen seines geringen Ressourcenverbrauchs nicht nur auf Standartrechnern sondern 
auch auf einem abgespeckten Netzwerkgerät mit Audioausgang betrieben werden und von allen Computern
oder auch Mobiltelefonen / PDAs im Netzwerk ferngesteuert werden.\ \\ \\
Es ist auch möglich den Daemon und den Client zur Fernsteuerung lokal auf dem gleichen Rechner
zu betreiben, er fungiert dann als normaler Medienspieler, der jedoch von einer Vielzahl 
unterschiedlicher Clients angesteuert werden kann, die sich in Oberflächengestaltung und Zusatzfunktionen
unterscheiden. Mittlerweile existierten auch zahlreiche Clients, die eine Webschnittstelle bereitstellen.\ \\ \\
Der MPD spielt die Audioformate Ogg Vorbis, FLAC, OggFLAC, MP2, MP3, MP4/AAC, MOD, Musepack und wave ab.
Zudem können FLAC-, OggFLAC-, MP3- und OggVorbis-HTTP-Streams abgespielt werden. Die Schnittstelle kann
auch ohne manuelle Konfigruation mit der Zeroconf-Technik angesteuert werden. Des Weiteren wird Replay
Gain, Gapless Playback, Crossfading und das Einlesen von Metadaten aus ID3-Tags, Vorbis comments oder
der MP4-Metadatenstruktur unterstützt.
\footnote{Zitat aus: http://de.wikipedia.org/wiki/Music\_Player\_Daemon}
\end{quote}
\newpage
\subsection{Der MPD kann:}
\renewcommand{\labelitemi}{•}
\begin{itemize}
        \item Musik abspielen
        \item Musik kontrollieren und in Warteschlangen reihen 
        \item Musik Dateien dekodieren
        \item HTTP(Hyper Text Transfer Protocol) streamen
        \renewcommand{\labelitemi}{--}
        \begin{itemize}
                \item Eine HTTP-URL kann zur Warteschlange hinzugefügt oder direkt abgespielt werden.\\
        \end{itemize}
\end{itemize}

\subsection{Der MPD kann nicht:}
\begin{itemize}
        \item Album-Cover speichern
        \item Funktionen eines Equalizers bereitstellen
        \item Musik Taggen (Informationen aus dem Web suchen)
        \item Text für Playlist-Dateien parsen
        \item Statistische Auswertungen machen
        \item Musik visualisieren
        \item Funktionen eines Remote-File-Servers bereitstellen
        \item Funktionen eines Video-Servers bereitstellen
\end{itemize}
\section{Definiton des MPD-Client}
Der Music Player Daemon Client ist nun die Schnittstelle zum MPD. Über diesen Client kann der MPD
gesteuert werden. Es gibt viele verschiedene Clients mit unterschiedlichsten Funktionen, da der 
Client nicht auf den Funktionsumfang des MPD begrenzt ist. Das heißt im Klartext, dass der Client
zwar nur die Funktionen über das Netzwerk steuern kann, die vom MPD implementiert sind aber nicht, 
dass er deshalb auch keine lokalen Dienste bzw. Funktionen anwenden kann. So kann ein Client 
beispielsweise alle Funktionen lokal implementieren, die unter dem Punkt \"3.1.2 Der MPD kann nicht:\" 
erwähnt wurden.
\newpage
\section{Grafische Übersicht}
\begin{figure}[h]
\centering
\includegraphics[scale=0.6]{Mpd-overview.png}
\end{figure}
\footnote{Bild-Quelle: http://images.wikia.com/mpd/images/6/68/Mpd-overview.png}
Der MPD-Server bekommt als Input mp3, ogg, flac, wav, mp4/aac,... Musik-Dateien die entweder in einer
Musik-Datenbank oder in Playlisten gespeichert sind. Der Standardoutput des MPD ist Alsa, libao, jack 
oder OSS, die Musik kann aber auch über einen Icecast oder Pulseaudio Clienten ausgegeben werden.
Der MPD-Client steuert den MPD-Server und hat selbst keinen Audio-Output.

\chapter{Lastenheft}
\section{Zielbestimmungen}
Welche Ziele sollen durch den Einsatz der Software erreicht werden?\ \\ \\
Dem einzelnen Benutzer soll das abspielen von Musik über eine Netzwerkverbindung ermöglicht
werden, dabei soll die Steuerung von einem lokalen Client übernommen werden. Die Musik soll
in eine rzentralen Datenbank angelegt und über die Soundkarte eines Servers abgespielt werden.
Die Client-Rechner sollen die Ausgabe steuern und Abspiellisten auf dem Server verwalten
können. Die Bedienung soll für alle Benutzer sehr einfach und komfortabel über einen lokalen
Client realisiert werden. Bei jedem Start des Clients, soll die letzte Sitzung wiederhergestellt
werden, falls keine Daten eine beendeten Sitzung gefunden werden, sollen Standarteinstellungen
verwendet werden.\ \\ \\
Standartmäßig sollen den Benutzern folgende Funktionen zuf Verfügung stehen:
\renewcommand{\labelitemi}{•}
\begin{itemize}
        \item Abspielen von Musik
        \item Steuerung von Musik (Play, Stop, Skip, ...)
        \item Decodieren von Musik
        \item Input-Stream via HTTP
\end{itemize}
Weitere Funktionen müssen modular integrierbar sein, allerdings müssen sie noch nicht implementiert
werden. Einige Beispiele für weitere Funktionen wären:
\begin{itemize}
	\item Finden von Album-Informationen
	\item Profil-Steuerung
	\item Visualisierung
\end{itemize}
Die Systemsprache soll auf Englisch festgelegt werden.
\subsection{Projektbeteiligte}
Wer soll an dem Projekt teilnehmen?
\begin{itemize}
	\item Christopher Pahl
	\item Christoph Piechula
	\item Eduard Schneider
	\item Marc Tigges
\end{itemize}
\section{Produkteinsatz}
Für welche Anwendungsbereiche und Zielgruppe ist die Software vorgesehen?\ \\ \\
Der MPD-Client ist nicht auf bestimmte Gewerbe beschränkt, ein jeder soll diesen
Client verwenden können. Grundlage für die Verwendung der Software ist die General
Public License (GPL) Version 3 vom 29 Juni 2007.\ \\ \\
Definition der GPL v3:
\begin{center}
	http://www.gnu.org/licenses/gpl.html
\end{center}
Die Software soll überall da eingesetzt werden, wo Musik abgespielt werden soll. 
Dabei ist man nicht auf einen Rechner beschränkt, auch Fernseher und Musik-Spieler
mit Internetzugang, entsprechender Softwareunterstützung und Audio output können 
theoretisch ein solches Programm verwenden.\ \\ \\
Hauptsächlich soll sich diese Software allerdings an Nutzer eines Rechners mit einem Unix-
artigen System richten. Des weiteren soll die Zielgruppe vorerst auf Benutzer beschränkt
sein, die Englisch verstehen.
\section{Produktfunktionen}
Welche sind die Hauptfunktionen aus Sicht des Auftraggebers?\ \\ \\
\subsection{Benutzerfunktionen}
Beim ersten Start des Systems soll eine Standard-Konfiguration geladen werden und die Verbindungseinstellungen
zu einem MPD-Server müssen vorgenommen werden. Bei jedem weiteren Start soll die Konfiguration geladen werden,
die vom Benutzer erstellt wurde, falls diese denn lokal gefunden werden kann. Der Benutzer soll sämtliche
Einstellungen selbstverständlich zu jeder Zeit ändern können.
\subsubsection{Starten und Beenden}
\begin{itemize}
	\item F\_0010 Der Benutzer kann das System zu jedem Zeitpunkt starten.
	\item F\_0020 Der Benutzer kann das System zu jedem Zeitpunkt beenden.
\end{itemize}
\subsubsection{Persönliche Daten}
Ein Benutzer verfügt über eine persönliche Verbindungseinstellung zum gewünschten MPD-Server.
Diese Daten können von dem Benutzer zu jeder Zeit angepasst werden.
\begin{itemize}
	\item F\_0110 Der Benutzer kann sich zu jeder Zeit seine Verbindungsdaten anzeigen lassen.
	\item F\_0120 Der Benutzer kann zu jeder Zeit seine persönlichen Daten anpassen.
\end{itemize}
\subsubsection{Persönliches Profil}
Da die Software auf Unix-artige Systeme beschränkt werden soll, geht ein angenehmer Vorteil mit einher, nämlich das
eine Profil-Verwaltung seitens des MPD-Clients nicht implementiert werden muss. Die verschiedenen Profile werden
durch die verschiedenen Profile des gesamten Betriebssystems definiert und differenziert.
\subsubsection{Persönliche Datenbank}
Eine persönliche Datenbank soll lokal nicht vorhanden sein. Die Datenbank des Benutzers befindet sich auf dem MPD-Server.
Einzig und alleine modulare Erweiterungen des MPD-Clients können lokale Datenbank-Implementierungen erfordern.
\subsubsection{Kommunikation (Chat)}
Kommunikation von MPD-Client zu MPD-Client kann theoretisch implementiert werden, eine solche Schnittstelle ist vorhanden.
Allerdings soll hierauf verzichtet werden, da im Vordergrund das Abspielen und Verwalten von Musik steht und es deutlich
einfachere und bessere Systeme gibt, mit Hilfe derer man kommunizieren kann.
\subsubsection{Suchen}
Eine einfache Textsuche zum finden von Titeln, Alben oder Interpreten innerhalb der Abspiellisten soll implementiert werden.
\begin{itemize}
	\item F\_0210 Der Benutzer kann seine Queue durchsuchen.
\end{itemize}
\subsection{Administrator-Funktionen}
Durch das Unix-artige System soll auch der Administrator-Zugriff geregelt werden. Sobald sich der Benutzer im Unix System
als Administrator befindet, kann er auch den MPD-Client administrieren. Ein zusätzlicher Administrator-Modus muss also
nicht implementiert werden.
\section{Produktdaten}
Welche Daten sollen persistent gespeichert werden?\ \\ \\
Die vom Benutzer vorgenommenen Verbindungseinstellungen und Client spezifischen Einstellungen,
sollen auf dem Rechner lokal und persistent gespeichert werden. Nur so kann ermöglicht werden,
dass nach jedem Start des Systems diese Einstellungen geladen und übernommen werden können.\ \\
Außerdem soll eine Log-Datei auf den einzelnen Rechnern angelegt werden, die dieses System
verwenden. In dieser Log-Datei werden Nachrichten des Systems gespeichert, um eventuelle Fehler
leicht finden und beheben zu können. Es soll zusätzlich der Zustand des Systems abgespeichert werden,
wenn das System beendet wird um das System beim nächsten Start in diesen Zustand versetzen zu können.
\begin{itemize}
	\item D\_0010 Persönlichen Verbindungseinstellungen.
	\begin{itemize}
		\item Platzhalter
		\item Platzhalter
	\end{itemize}
	\item D\_0020 Client spezifische Einstellungen.
	\begin{itemize}
		\item Platzhalter
		\item Platzhalter
	\end{itemize}
	\item D\_0030 Eine Log-Datei.
	\begin{itemize}
		\item Platzhalter
		\item Platzhalter
	\end{itemize}
	\item D\_0040 Der Zustand.
	\begin{itemize}
		\item Platzhalter
		\item Platzhalter
	\end{itemize}
\end{itemize}
\section{Produktleistungen}
Werden für bestimmte Funktionen besondere Ansprüche in Bezug auf Zeit, Datenumfang oder Genauigkeit gestellt?\ \\ \\
Wenn das System beendet wird, soll der aktuelle Zustand des Systems gespeichert werden.
\begin{itemize}
	\item L\_0010 Speicherung des Systemzustandes
\end{itemize}
Es soll möglichst wenig Speicher gebraucht werden, die CPU soll möglichst wenig belastet werden und der Netzwerk-Traffic
soll gering gehalten werden.
\begin{itemize}
	\item L\_0020 Möglichst wenig Ressourchen-Verbrauch
\end{itemize}
Die Geschwindigkeit der Software ist auch abhänig von der jeweiligen Server-Lokation, der Benutzer wählt den Server
d.h. somit ist auch der Benutzer teil-verantworltich für die Geschwindigkeit.\ \\ \\
Der Status eines Liedes (Liedposition) wird alle 500 ms aktualisiert.
\begin{itemize}
	\item L\_0030 Lokaler Heartbeat alle 500 ms
\end{itemize}
\section{Qualitätsanforderungen}
Welche qualitativen Anforderungen sind von besonderer Bedeutung?\ \\ \\
Es soll auf folgende Priorität unter den Qualitätsanforderungen geachtet werden,
dabei ist das erste Element das wichtigste und das letzte das unwichtigste.\ \\ \\
Priorität 1: Robustheit\ \\
Priorität 2: Zuverlässigkeit\ \\
Priorität 3: Effizienz\ \\
Priorität 4: Intuitive Benutzung\ \\
Priorität 5: Design\ \\ \\
\section{Ergänzungen}
\subsection{Realisierung}
Das System muss mit den Programmiersprachen C und/oder C++ realisiert werden. Dabei ist auf
Objektorientierung zu achten, um Modularität und Wartbarkeit gewährleisten zu können.
Es können beliebige Entwicklungsumgebungen verwendet werden. Um einfaches und sicheres arbeiten
ermöglichen zu können, soll die Versionsverwaltungssoftware \"git\" benutzt werden, um die
Entwicklungsdateien zu speichern und zu bearbeiten. Zu dem Projekt soll eine ausführliche
Dokumentation erstellt werden, um dauerhafte Wartbarkeit und Anpassung des MPD-Client  gewährleisten
zu können, dazu gehören auch entsprechende Software-Diagramme (wie z.B. UML).
\subsection{Die nächste Version}
Aufgrund des modularen Aufbaus kann das System beliebig oft und in verschiedene Richtungen weiterentwickelt werden.

\chapter{Pflichtenheft}
\section{Zielbestimmungen}
\subsection{Muss-Kriterien}
\renewcommand{\labelitemi}{•}
\begin{itemize}
	\item Server-Verbindung
	\begin{itemize}
		\item Platzhalter
		\item Platzhalter
		\item Platzhalter
	\end{itemize}
	\item Client-Einstellungen
	\begin{itemize}
		\item Platzhalter
		\item Platzhalter
		\item Platzhalter
	\end{itemize}
	\item Musik-Steuerung
	\begin{itemize}
		\item Platzhalter
		\item Platzhalter
		\item Platzhalter
	\end{itemize}
	\item Sonstiges
	\begin{itemize}
		\item Platzhalter
		\item Platzhalter
		\item Platzhalter
	\end{itemize}
\end{itemize}
\subsection{Wunsch-Kriterien}
\begin{itemize}
		\item Platzhalter
		\item Platzhalter
		\item Platzhalter
\end{itemize}
\subsection{Abgrenzungskriterien}
\begin{itemize}
		\item Platzhalter
		\item Platzhalter
		\item Platzhalter
\end{itemize}
\section{Produkteinsatz}
Welche Anwendungsbereiche (Zweck), Zielgruppen (Wer mit welchen Qualifikationen), Betriebsbedingungen (Betriebszeit,
Aufsicht)?\ \\ \\
Der MPD-Client ist nicht auf bestimmte Gewerbe beschränkt, ein jeder soll diesen Client
verwenden können. Grundlage für die Verwendung der Software ist die General Public License (GPL)
Version 3 vom 29 Juni 2007.\ \\ \\
Definition der GPL:
\begin{center}
http://www.gnu.org/licenses/gpl.html
\end{center}
\subsection{Anwendungsbereiche}
Einzelpersonen verwenden dieses System, um überall da wo mit 
einem Unix-artigen Betriebssystem Musik abgespielt werden soll.
\subsection{Zielgruppen}
Personengruppen die komfortabel von überall aus auf ihre Musik und Playlist zugreifen
wollen ohne diese jedes mal aufwändig synchronisieren zu müssen (z.B. durch Abgleich von Datenträgern).\ \\ \\
Es werden Basiskenntnisse zum Aufbau einer Netzwerkverbindung und zur Nutzung des Internets vorausgesetzt.
Aufgrund der für das System vorgesehenen Betriebsumgebung sind ebenso Kenntnisse im Umgang mit Unix nötig.\ \\ \\
Der Benutzer muss die Systemsprache Englisch beherrschen.
\subsection{Betriebsbedingungen}
Das System soll sich bezüglich der Betriebsbedingungen nicht sonderlich von vergleichbaren Systemen bzw.
Anwendungen unterscheiden und dementsprechend folgend Punkte erfüllen:
\begin{itemize}
	\item Betriebsdauer: Täglich, 24 Stunden
	\item Keinerlei Wartung soll nötig sein
	\item Sicherungen der Konfiguration müssen vom Benutzer vorgenommen werden
\end{itemize}
\section{Produktumgebung}
\subsection{Software}
\begin{itemize}
	\item Avahi Daemon
	\item MPD-Client
\end{itemize}
Ein MPD-Server ist nicht unbedingt von nöten.
\subsection{Hardware}
Minimale Hardwareanforderungen:
Empfohlene Hardwareanforderungen:
\subsection{Orgware}
\begin{itemize}
	\item git (Versionsverwaltungssoftware)
	\item cmake (Compiler)
	\item doxygen (Dokumentation)
	\item Editor nach Wahl
	\item Glade
\end{itemize}
\section{Produktfunktionen}
Funktionen des MPD-Clients.\ \\ \\
Beim ersten Start des Systems soll eine Standard-Konfiguration geladen werden und die Verbindungseinstellungen
zu einem MPD-Server müssen vorgenommen werden. Bei jedem weiteren Start soll die Konfiguration geladen werden,
die vom Benutzer erstellt wurde, falls diese denn lokal gefunden werden kann. Der Benutzer soll sämtliche
Einstellungen selbstverständlich zu jeder Zeit ändern können.
\subsection{Allgemeine Funktionen}
\subsubsection{Starten und Beenden}
\begin{itemize}
	\item F\_0010 Der Benutzer kann das System zu jedem Zeitpunkt starten.
	\item F\_0020 Der Benutzer kann das System zu jedem Zeitpunkt beenden.
	\item F\_0030 Beim ersten Start wird ein Standart-System-Zustand geladen.
	\item F\_0040 Beim Beenden wird der aktuelle System-Zustand gespeichert.
	\item F\_0050 Bei jedem weiteren Start wird der letzte System-Zustand geladen.
\end{itemize}
\subsection{Benutzerfunktionen}
\subsubsection{Benutzer-Kennung}
Eine Benutzerkennung ist nicht erforderlich und wurde deshalb auch nicht implementiert.
\subsubsection{Persönliche Daten}
Verbindungseinstellungen müssen vorgenommen werden und können zu jedem Zeitpunkt geändert werden.
\begin{itemize}
	\item F\_0110 Der Benutzer kann Verbindungseinstellungen vornehmen und sie ändern
\end{itemize}
\subsubsection{Persönliche Konfiguration}
config.xml
log datei
\subsubsection{Persönliches Profil}
Da die Software auf Unix-artige Systeme beschränkt ist, wurde keine Profil-Verwaltung implementiert. Die
verschiedenen Profile werden durch die verschiedenen Profile des gesamten Betriebssystems definiert und differenziert.
\subsubsection{Persönliche Datenbank}
Eine persönliche Datenbank ist lokal nicht vorhanden. Die Datenbank des Benutzers befindet sich auf dem MPD-Server.
Einzig und alleine modulare Erweiterungen des MPD-Clients können lokale Datenbank-Implementierungen erfordern.
\subsubsection{Kommunikation (Chat)}
Kommunikation von MPD-Client zu MPD-Client kann theoretisch implementiert werden, eine solche Schnittstelle ist vorhanden.
Allerdings wurde hierauf verzichtet, da im Vordergrund das Abspielen und Verwalten von Musik steht und es deutlich
einfachere und bessere Systeme gibt, mit Hilfe derer man kommunizieren kann.
\subsubsection{Suchen}
Eine einfache Textsuche zum finden von Titeln, Alben oder Interpreten innerhalb der Abspiellisten wurde implementiert.
Dabei springt die Markierung des Textes beim eingeben von Zeichen in die Suche zu der ersten übereinstimmenden
Stelle in der Playlist des Clients. Erst beim bestätigen der Eingabe im Suchfeld wird die Auswahl gefiltert.
\begin{itemize}
        \item F\_0210 Der Benutzer kann seine Queue durchsuchen.
\end{itemize}
\subsection{Abspielfunktionen}
\begin{itemize}
	\item Play
	\item Skip
	\item Stop
	\item Pause
	\item Shuffle
	\item Loop
\end{itemize}
\subsubsection{Initialisierung}
Platzhalter
\subsubsection{Verlauf}
\subsection{Administrator-Funktionen}
Durch das Unix-artige System wird auch der Administrator-Zugriff geregelt. Sobald sich der Benutzer im Unix System
als Administrator befindet, kann er auch den MPD-Client administrieren. Ein zusätzlicher Administrator-Modus wurde also
nicht implementiert.
\section{Produktdaten}
\section{Produktleistungen}
\section{Benutzeroberfläche}
\subsection{Bildschirmlayout}
\subsubsection{Startbildschirm}
\subsubsection{Einstellungsfenster}
\subsubsection{Verbindungsfenster}
\subsubsection{Benutzermenü}
\section{Produktmodellierung}
\section{Qualitätsanforderungen}
\section{Globale Testszenarien und Testfälle}
\section{Entwicklungsumgebung}
\subsection{Software}
\subsection{Hardware}
\subsection{Orgware}
\section{Ergänzungen}
\section{Glossar}

\section{Globale Testszenarien und Testfälle}
\subsection{Cxxtest}
Als Testframework wurde CxxTest ausgewählt. Die Gründe für diese Entscheidung werden gut von der offiziellen Definition zusammengefasst: 
\begin{quote}\footnote{http://cxxtest.tigris.org/}
    CxxTest is a JUnit/CppUnit/xUnit-like framework for C/C++.\ \\
    It is focussed on being a lightweight framework that is well suited for 
    integration into embedded systems development projects.\ \\
    CxxTest's advantages over existing alternatives are that it:
    \begin{itemize}
        \item Doesn't require RTTI
        \item Doesn't require member template functions
        \item Doesn't require exception handling
        \item Doesn't require any external libraries (including memory management, file/console I/O, graphics libraries)
        \item Is distributed entirely as a set of header files (and a python script).
        \item Doesn't require the user to manually register tests and test suites 
    \end{itemize}
    This makes it extremely portable and usable.\ \\ \\
\end{quote} 
\subsection{Testfälle}
Für Teile des Programmes die nicht vom Testprotokoll erfasst werden und automatisch gestestet werden, sollen Testfälle mit Cxxtest
geschrieben werden.   
\subsection{Testprotokoll}
Um Fehler aufzuspüren, die die grafische Oberfläche betreffen, wurde ein Testprotokoll erstellt, in dem zunächst
alle möglichen Funktionen der grafischen Oberfläche aufgelistet werden. Außerdem müssen diese Funktionen mit 
anderen Funktionen kombiniert und mehrfach ausgeführt werden. Zu jedem dieser Fälle ist ein zu erwartendes Ergebnis
festzulegen und anschließend zu überprüfen ob das erwartete Ergebnis eingetroffen ist. Das eingetroffene Ergebnis
ist ebenfalls zu protokollieren. Es wurden jeweils die Buttons, sowie die Shortcuts geprüft.


\newpage
\subsubsection{Abspielfunktionen}
\textbf{Einfache Ausführung:}
\\
\begin{tabularx}{\textwidth}{|X|X|p{3cm}|}
    \hline
    \textbf{Testfall} & \textbf{Erwartetes Ergebnis} & \textbf{Ergebnis eingetroffen?}\\
    \hline
    Play & Musik spielt ab. Play wird zu Pause. & Ja\\
    \hline
    Pause & Musik pausiert. Pause wird zu Play. & Ja\\
    \hline
    Next & Nächstes Lied abspielen & Ja\\
    \hline
    Previous & Vorheriges Lied abspielen & Ja\\
    \hline
    Stop & Beende abspielen Pause wird zu Play. & Ja\\
    \hline
    Skipping & An Liedposition springen & Ja\\
    \hline
    Random & Musik der Queue zufällig abspielen & Ja\\
    \hline 
    Repeat & Ein Lied wiederholen & Ja\\
    \hline
    Repeat all & Queue wiederholen & Ja\\
    \hline
    Consume Mode & Ein abgespieltes Lied entfernen & Ja\\
    \hline
    Single Mode & Ein Lied abspielen, dann Stoppen & Ja\\
    \hline
\end{tabularx}
\newpage
\textbf{Kombinierte Ausführung:}
\\
\begin{tabularx}{\textwidth}{|X|p{7cm}|p{3cm}|}
    \hline
    \textbf{Testfall} & \textbf{Erwartetes Ergebnis} & \textbf{Ergebnis eingetroffen?}\\
    \hline
    Play, Stop, Play(1) & Musik spielt ab. Musik stoppt. Musik spielt ab. & Ja\\
    \hline
    Play, Pause, Play(2) & Musik spiel ab. Musik pausiert. Musik spiel ab. & Ja\\
    \hline
    Next, Next(3) & Skip weiter. Skip weiter. & Ja\\
    \hline
    Previous, Previous(4) & Skip zurück. Skip zurück & Ja\\
    \hline
    Next, Previous(5) & Skip weiter. Skip zurück & Ja\\
    \hline
    Previous, Next(6) & Skip zurück. Skip weiter & Ja\\
    \hline
    Random, Repeat all & Musik der Queue zufällig abspielen Queue wiederholen & Ja\\
    \hline
    Random, Consume Mode & Musik der queue zufällig abspielen Ein abgespieltes Lied entfernen & Ja\\
    \hline
    Random, Single Mode & Musik der Queue zufällig abspielen Ein Lied abspielen, dann stoppen & Ja\\
    \hline
    Consume Mode, Single Mode & Ein abgespieltes Lied entfernen Ein Lied abspielen, dann Stoppen & Ja\\
    \hline
    Consume Mode, Repeat all & Kann nur einmal durchlaufen & Ja\\
    \hline
    Random, 1 & Musik der Queue zufällig abspielen 1 & Ja\\
    \hline
    Random, 2 & Musik der Queue zufällig abspielen 2 & Ja\\
    \hline
    Random, 3 & Musik der Queue zufällig abspielen 3 & Ja\\
    \hline
    Random, 4 & Musik der Queue zufällig abspielen 4 & Ja\\
    \hline
    Random, 5 & Musik der Queue zufällig abspielen 5 & Ja\\
    \hline
    Random, 6 & Musik der Queue zufällig abspielen 6 & Ja\\
    \hline
    Repeat all, 1 & Queue wiederholen 1 & Ja\\
    \hline
    Repeat all, 2 & Queue wiederholen 2 & Ja\\
    \hline
    Repeat all, 3 & Queue wiederholen 3 & Ja\\
    \hline
    Repeat all, 4 & Queue wiederholen 4 & Ja\\
    \hline
\end{tabularx}


\begin{tabularx}{\textwidth}{|X|l|p{3cm}|}
    %\begin{tabularx}[c]{|p{6cm}|p{6cm}|c|}
    \hline
    \textbf{Testfall} & \textbf{Erwartetes Ergebnis} & \textbf{Ergebnis eingetroffen?}\\
    \hline
    Repeat all, 5 & Queue wiederholen 5 & Ja\\
    \hline
    Repeat all, 6 & Queue wiederholen 6 & Ja\\
    \hline
    Consume Mode, 1 & Ein abgespieltes Lied entfernen 1 & Ja\\
    \hline
    Consume Mode, 2 & Ein abgespieltes Lied entfernen 2 & Ja\\
    \hline
    Consume Mode, 3 & Ein abgespieltes Lied entfernen 3 & Ja\\
    \hline
    Consume Mode, 4 & Ein abgespieltes Lied entfernen 4 & Ja\\
    \hline
    Consume Mode, 5 & Ein abgespieltes Lied entfernen 5 & Ja\\
    \hline
    Consume Mode, 6 & Ein abgespieltes Lied entfernen 6 & Ja\\    
    \hline   
    Single Mode, 1 & Ein Lied abspielen, dann Stoppen 1 & Ja\\
    \hline
    Single Mode, 2 & Ein Lied abspielen, dann Stoppen 2 & Ja\\
    \hline
    Single Mode, 3 & Ein Lied abspielen, dann Stoppen 3 & Ja\\
    \hline
    Single Mode, 4 & Ein Lied abspielen, dann Stoppen 4 & Ja\\
    \hline
    Single Mode, 5 & Ein Lied abspielen, dann Stoppen 5 & Ja\\
    \hline
    Single Mode, 6 & Ein Lied abspielen, dann Stoppen 6 & Ja\\
    \hline
\end{tabularx}
\newpage
Im folgenden wird auf das Protokoll der kombinierten Ausführung referenziert.
\\
\\
\textbf{Mehrfache Ausführung:}
\\
\begin{tabularx}{\textwidth}{|X|X|p{3cm}|}
    \hline
    \textbf{Testfall} & \textbf{Erwartetes Ergebnis} & \textbf{Ergebnis eingetroffen?}\\
    \hline
    Fall 1 x 10 & Fall 1 x 10 & Ja\\
    \hline
    Fall 2 x 10 & Fall 2 x 10 & Ja\\
    \hline
    Fall 3 x 10 & Fall 3 x 10 & Ja\\
    \hline
    Fall 4 x 10 & Fall 4 x 10 & Ja\\
    \hline
    Fall 5 x 10 & Fall 5 x 10 & Ja\\
    \hline
    Fall 6 x 10 & Fall 6 x 10 & Ja\\
    \hline
    Fall 12 x 10 & Fall 12 x 10 & Ja\\
    \hline
    Fall 13 x 10 & Fall 13 x 10 & Ja\\
    \hline
    Fall 14 x 10 & Fall 14 x 10 & Ja\\
    \hline
    Fall 15 x 10 & Fall 15 x 10 & Ja\\
    \hline
    Fall 16 x 10 & Fall 16 x 10 & Ja\\
    \hline
    Fall 17 x 10 & Fall 17 x 10 & Ja\\
    \hline
    Fall 18 x 10 & Fall 18 x 10 & Ja\\
    \hline
    Fall 19 x 10 & Fall 19 x 10 & Ja\\
    \hline
    Fall 20 x 10 & Fall 20 x 10 & Ja\\
    \hline
    Fall 21 x 10 & Fall 21 x 10 & Ja\\
    \hline
    Fall 22 x 10 & Fall 22 x 10 & Ja\\
    \hline
    Fall 23 x 10 & Fall 23 x 10 & Ja\\
    \hline
    Fall 24 x 10 & Fall 24 x 10 & Ja\\
    \hline
    Fall 25 x 10 & Fall 25 x 10 & Ja\\
    \hline
    Fall 26 x 10 & Fall 26 x 10 & Ja\\
    \hline
    Fall 27 x 10 & Fall 27 x 10 & Ja\\
    \hline
    Fall 28 x 10 & Fall 28 x 10 & Ja\\
    \hline 
    Fall 29 x 10 & Fall 29 x 10 & Ja\\
    \hline
    Fall 30 x 10 & Fall 30 x 10 & Ja\\
    \hline
    Fall 31 x 10 & Fall 31 x 10 & Ja\\
    \hline
    Fall 32 x 10 & Fall 32 x 10 & Ja\\
    \hline
    Fall 33 x 10 & Fall 33 x 10 & Ja\\
    \hline
    Fall 34 x 10 & Fall 34 x 10 & Ja\\
    \hline
    Fall 35 x 10 & Fall 35 x 10 & Ja\\
    \hline
\end{tabularx}


\subsubsection{Queue-Funktionen}
\textbf{Einfache Ausführung} \\
\begin{tabularx}{\textwidth}{|X|l|p{3cm}|}
    \hline
    \textbf{Testfall} & \textbf{Erwartetes Ergebnis} & \textbf{Ergebnis eingetroffen?}\\
    \hline
    Remove & Ein Lied aus Queue entfernen & Ja\\
    \hline
    Clear & Alle Lieder aus Queue entfernen & Ja\\
    \hline
    Save as Playlist & Queue als Playlist speichern & Ja\\
    \hline
    Suchen & Nach eingegebenem Wort suchen & Ja\\
    \hline
\end{tabularx}
\\
\\
\textbf{Kombinierte Ausführung} \\
Kombinierte Ausführung der Funktionen der Queue machen nicht wirklich viel Sinn da z.B.
die Funktion Clear die Queue löscht. Die einzige Kombination die Sinn macht getestet zu werden ist die 
folgende:\ \\ \\
\begin{tabularx}{\textwidth}{|X|X|p{3cm}|}
    \hline
    \textbf{Testfall} & \textbf{Erwartetes Ergebnis} & \textbf{Ergebnis eingetroffen?}\\
    \hline
    Suchen, Remove & Nach eingegebenem Wort suchen\newline Ein Lied aus der Queue entfernen & Ja\\
    \hline
\end{tabularx}
\\
\\
\textbf{Mehrfache Ausführung} \\
Die mehrfache Ausführung ist ähnlich unsinnig wie die der kombinierten Ausführung.
Mehrmals hintereinander die Queue löschen ist nicht möglich. So bleibt wieder nur ein Testfall zu prüfen:\ \\ \\
\begin{tabularx}{\textwidth}{|X|p{7cm}|p{3cm}|}
    \hline
    \textbf{Testfall} & \textbf{Erwartetes Ergebnis} & \textbf{Ergebnis eingetroffen?}\\
    \hline
    Suchen, Remove x 10 & Nach eingegebenem Wort suchen\newline Ein Lied aus der Queue entfernen x 10& Ja\\
    \hline
\end{tabularx}

\newpage
\subsubsection{Playlist-Funktionen}
\textbf{Einfache Ausführung} \\ 
\begin{tabularx}{\textwidth}{|X|X|p{3cm}|}
    \hline
    \textbf{Testfall} & \textbf{Erwartetes Ergebnis} & \textbf{Ergebnis eingetroffen?}\\
    \hline
    Hinzufügen & Playlist hinzufügen & Ja\\
    \hline
    Ersetzen & Playlist ersetzen & Ja\\
    \hline
    Playlist entfernen & Playlist löschen & Ja\\
    \hline
\end{tabularx}
\\
\\
\textbf{Kombinierte Ausführung} \\
\begin{tabularx}{\textwidth}{|X|X|p{3cm}|}
    \hline
    \textbf{Testfall} & \textbf{Erwartetes Ergebnis} & \textbf{Ergebnis eingetroffen?}\\
    \hline
    Hinzufügen, Hinzufügen & Playlist hinzufügen\newline Playlist hinzufügen & Ja\\
    \hline
    Ersetzen, Ersetzen & Playlist ersetzen\newline Playlist ersetzen & Ja\\
    \hline
    Entfernen, Entfernen & Playlist entfernen\newline Playlist entfernen & Ja\\
    \hline
    Hinzufügen, Entfernen & Playlist hinzufügen\newline Playlist löschen & Ja\\
    \hline
    Ersetzen, Entfernen & Playlist ersetzen\newline Playlist entfernen & Ja\\
    \hline
\end{tabularx}
\\
\\
\\
Im folgenden wird auf das Protokoll der kombinierten Ausführung referenziert.
\\
\textbf{Mehrfache Ausführung:}
\\
\begin{tabularx}{\textwidth}{|X|X|l|}
    \hline
    \textbf{Testfall} & \textbf{Erwartetes Ergebnis} & \textbf{Ergebnis eingetroffen?}\\
    \hline
    Fall 1 x 10 & Fall 1 x 10 & Ja\\
    \hline
    Fall 2 x 10 & Fall 2 x 10 & Ja\\
    \hline
    Fall 3 x 10 & Fall 3 x 10 & Ja\\
    \hline
    Fall 4 x 10 & Fall 4 x 10 & Ja\\
    \hline
    Fall 5 x 10 & Fall 5 x 10 & Ja\\
    \hline
\end{tabularx}

\newpage
\subsubsection{Dateibrowser-Funktionen}
\textbf{Einfache Ausführung}
\\
\begin{tabularx}{\textwidth}{|X|p{7cm}|p{3cm}|}
    \hline
    \textbf{Testfall} & \textbf{Erwartetes Ergebnis} & \textbf{Ergebnis eingetroffen?}\\
    \hline
    Hinzufügen & Zur Queue hinzufügen & Ja\\
    \hline
    Alle Hinzufügen & Alle zur Queue hinzufügen & Ja\\
    \hline
    Ersetzen & Queue durch Auswahl ersetzen & Ja\\
    \hline
    Aktualisieren & Dateibrowser aktualisieren & Ja\\
    \hline
    Neu einlesen & Dateibrowser neu einlesen & Ja\\
    \hline
    Suchen & Nach eingegebenem Wort suchen & Ja\\
    \hline
\end{tabularx}
\\
\\
\textbf{Kombinierte Ausführung:}
\\
\begin{tabularx}{\textwidth}{|X|p{7cm}|p{3cm}|}
    \hline
    \textbf{Testfall} & \textbf{Erwartetes Ergebnis} & \textbf{Ergebnis eingetroffen?}\\
    \hline
    Hinzufügen\newline Hinzufügen & Zur Queue hinzufügen\newline Zur Queue hinzufügen & Ja\\
    \hline
    Alle Hinzufügen\newline Alle hinzufügen &  Alle zur Queue hinzufügen\newline Alle zur Queue hinzufügen & Ja\\
    \hline
    Ersetzen\newline Ersetzen & Queue durch Auswahl ersetzen\newline Queue durch Auswahl ersetzen & Ja\\
    \hline
    Aktualisieren\newline Aktualisieren & Dateibrowser aktualisieren\newline Dateibrowser aktualisieren & Ja\\
    \hline
    Neu einlesen\newline Neu einlesen & Dateibrowser neu einlesen\newline Dateibrowser neu einlesen & Ja\\
    \hline
    Suchen\newline Suchen & Nach eingegebenem Wort suchen\newline Nach eingegebenem Wort suchen & Ja\\
    \hline
    Hinzufügen\newline Alle Hinzufügen & Zur Queue hinzufügen\newline Alle zur Queue hinzufügen & Ja\\
    \hline
    Hinzufügen\newline Ersetzen & Zur Queue hinzufügen\newline Queue durch Auswahl ersetzen & Ja\\
    \hline
    Hinzufügen\newline Aktualisieren & Zur Queue hinzufügen\newline Dateibrowser aktualisieren & Ja\\
    \hline
    Hinzufügen\newline Neu einlesen & Zur Queue hinzufügen\newline Dateibrowser neu einlesen & Ja\\
    \hline
\end{tabularx}


\begin{tabularx}{\textwidth}{|X|X|l|}
    \hline
    \textbf{Testfall} & \textbf{Erwartetes Ergebnis} & \textbf{Ergebnis eingetroffen?}\\
    \hline
    Hinzufügen\newline Suchen & Zur Queue hinzufügen\newline Nach eingegebenem Wort suchen & Ja\\
    \hline
    Alle Hinzufügen\newline Ersetzen & Alle zur Queue hinzufügen\newline Queue durch Auswahl ersetzen & Ja\\
    \hline
    Alle Hinzufügen\newline Aktualisieren & Alle zur Queue hinzufügen\newline Dateibrowser aktualisieren & Ja\\
    \hline
    Alle Hinzufügen\newline Neu einlesen & Alle zur Queue hinzufügen\newline Dateibrowser neu einlesen & Ja\\
    \hline
    Alle Hinzufügen\newline Suchen & Alle zur Queue hinzufügen\newline Nach eingegebenem Wort suchen & Ja\\
    \hline
    Ersetzen\newline Aktualisieren & Queue durch Auswahl ersetzen\newline Dateibrowser aktualisieren & Ja\\
    \hline
    Ersetzen\newline Neu einlesen & Queue durach Auswahl ersetzen\newline Dateibrowser neu einlesen & Ja\\
    \hline
    Ersetzen\newline Suchen & Queue durch Auswahl ersetzen\newline Nach eingegebenem Wort suchen & Ja\\
    \hline
    Aktualisieren\newline Neu einlesen & Dateibrowser aktualisieren\newline Dateibrowser neu einlesen & Ja\\
    \hline
    Aktualisieren\newline Suchen & Dateibrowser aktualisieren\newline Nach eingegebenem Wort suchen & Ja\\
    \hline
    Neu einlesen\newline Suchen & Dateibrowser neu einlesen\newline Nach eingegebenem Wort suchen & Ja\\
    \hline
\end{tabularx}

\newpage
Im folgenden wird auf das Protokoll der kombinierten Ausführung referenziert.
\\
\textbf{Mehrfache Ausführung}
\\
\begin{tabularx}{\textwidth}{|X|X|l|}
    \hline
    \textbf{Testfall} & \textbf{Erwartetes Ergebnis} & \textbf{Ergebnis eingetroffen?}\\
    \hline
    Fall 1 x 10 & Fall 1 x 10 & Ja\\
    \hline
    Fall 2 x 10 & Fall 2 x 10 & Ja\\
    \hline
    Fall 3 x 10 & Fall 3 x 10 & Ja\\
    \hline
    Fall 4 x 10 & Fall 4 x 10 & Ja\\
    \hline
    Fall 5 x 10 & Fall 5 x 10 & Ja\\
    \hline
    Fall 6 x 10 & Fall 6 x 10 & Ja\\
    \hline
    Fall 7 x 10 & Fall 7 x 10 & Ja\\
    \hline
    Fall 8 x 10 & Fall 8 x 10 & Ja\\
    \hline
    Fall 9 x 10 & Fall 9 x 10 & Ja\\
    \hline
    Fall 10 x 10 & Fall 10 x 10 & Ja\\
    \hline
    Fall 11 x 10 & Fall 11 x 10 & Ja\\
    \hline
    Fall 12 x 10 & Fall 12 x 10 & Ja\\
    \hline
    Fall 13 x 10 & Fall 13 x 10 & Ja\\
    \hline
    Fall 14 x 10 & Fall 14 x 10 & Ja\\
    \hline
    Fall 15 x 10 & Fall 15 x 10 & Ja\\
    \hline
    Fall 16 x 10 & Fall 16 x 10 & Ja\\
    \hline
    Fall 17 x 10 & Fall 17 x 10 & Ja\\
    \hline
    Fall 18 x 10 & Fall 18 x 10 & Ja\\
    \hline
    Fall 19 x 10 & Fall 19 x 10 & Ja\\
    \hline
    Fall 20 x 10 & Fall 20 x 10 & Ja\\
    \hline
    Fall 21 x 10 & Fall 21 x 10 & Ja\\
    \hline
\end{tabularx}

\newpage
\subsubsection{Statistik}
Für die Statistik kann kein Testprotokoll angewandt werden.
\subsubsection{Einstellungen}
\textbf{Einfache Ausführung}
\\
\begin{tabularx}{\textwidth}{|X|X|l|}
    \hline
    \textbf{Testfall} & \textbf{Erwartetes Ergebnis} & \textbf{Ergebnis eingetroffen?}\\
    \hline
    Zeige Liste & Zeige Avahi Liste & Ja\\
    \hline
\end{tabularx}
\\
\\
\textbf{Kombinierte Ausführung}
\\
\begin{tabularx}{\textwidth}{|X|X|l|}
    \hline
    \textbf{Testfall} & \textbf{Erwartetes Ergebnis} & \textbf{Ergebnis eingetroffen?}\\
    \hline
    Fall 1 x 10 & Fall 1 x 10 & Ja\\
    \hline
\end{tabularx}
\\
\\
Es existieren keine Buttons oder Shortcuts die kombiniert werden könnten. 
\subsubsection{Lautstärke}
\textbf{Einfache Ausführung}
\\
\begin{tabularx}{\textwidth}{|X|X|l|}
    \hline
    \textbf{Testfall} & \textbf{Erwartetes Ergebnis} & \textbf{Ergebnis eingetroffen?}\\
    \hline
    Lautstärke erhöhen & Lautstärke erhöhen & Ja\\
    \hline
    Lautstärke verringern & Lautstärke verringern & Ja\\
    \hline
\end{tabularx}
\\
\\
\textbf{Kombinierte Ausführung}
\\
\begin{tabularx}{\textwidth}{|X|X|l|}
    \hline
    \textbf{Testfall} & \textbf{Erwartetes Ergebnis} & \textbf{Ergebnis eingetroffen?}\\
    \hline
    Lautstärke erhöhen\newline Lautstärke verringern & Lautstärke erhöhen\newline Lautstärke verringern & Ja\\
    \hline
\end{tabularx}
\\
\\
\\
Im folgenden wird auf das Protokoll der kombinierten Ausführung referenziert.
\\
\textbf{Mehrfache Ausführung}
\\
\begin{tabularx}{\textwidth}{|X|X|l|}
    \hline
    \textbf{Testfall} & \textbf{Erwartetes Ergebnis} & \textbf{Ergebnis eingetroffen?}\\
    \hline
    Fall 1 x 10 & Fall 1 x 10 & Ja\\
    \hline
\end{tabularx}

\newpage
\subsubsection{Sonstiges}
\textbf{Einfache Ausführung}
\\
\begin{tabularx}{\textwidth}{|X|X|l|}
    \hline
    \textbf{Testfall} & \textbf{Erwartetes Ergebnis} & \textbf{Ergebnis eingetroffen?}\\
    \hline
    Verbinden & Verbindung zum MPD-Server & Ja\\
    \hline
    Trennen & Verbindung zum Server trennen & Ja\\
    \hline
    Beenden & MPD-Client beenden & Ja\\
    \hline
\end{tabularx}
\\
\\
\textbf{Kombinierte Ausführung}
\\
\begin{tabularx}{\textwidth}{|X|X|l|}
    \hline
    \textbf{Testfall} & \textbf{Erwartetes Ergebnis} & \textbf{Ergebnis eingetroffen?}\\
    \hline
    Verbinden\newline Verbinden & Verbindung zum MPD-Server\newline Verbindung zum MPD-Server & Ja\\
    \hline
    Verbinden\newline Trennen & Verbindung zum MPD-Server\newline Verbindung zum Server trennen & Ja\\
    \hline
    Verbinden\newline Beenden & Verbindung zum MPD-Server\newline MPD-Client beenden & Ja\\
    \hline
    Trennen\newline Beenden & Verbindung zum Server trennen\newline MPD-Client beenden & Ja\\
    \hline
\end{tabularx}
\\
\\
Im folgenden wird auf das Protokoll der kombinierten Ausführung referenziert.
\\
\textbf{Mehrfache Ausführung}
\\
\begin{tabularx}{\textwidth}{|X|X|l|}
    \hline
    \textbf{Testfall} & \textbf{Erwartetes Ergebnis} & \textbf{Ergebnis eingetroffen?}\\
    \hline
    Fall 1 x 10 & Fall 1 x 10 & Ja\\
    \hline
    Fall 2 x 10 & Fall 2 x 10 & Ja\\
    \hline
    Fall 3 x 10 & Nur 1 x ausführbar & Ja\\
    \hline
    Fall 4 x 10 & Nur 1 x ausführbar & Ja\\
    \hline
\end{tabularx}


\chapter{Software Design}

\section{Einführung}

\subsubsection{Namespace-Übersicht}
\begin{figure}[h]
    \centering
    \includegraphics[scale=0.3]{Namespace_Uebersicht.png}
\end{figure}

\section{,,Das Problem''}

Da die grundlegende Funktionsweise des MPD Server auf einer Client Server Architektur beruht, muss der MPD Client
verschiedene Kommandos wie zum Beispiel play, pause, listplaylists etc. an den Server schicken
und zur gleichen Zeit aber auch auf Änderungen reagieren können, d.h. zum Beispiel wenn sich die Lautstärke ändert,
da jederzeit auch andere Clients oder Server den MPD internen Zustand ändern können.
Diese Änderungen müssen auch anderen Programmteilen bekannt gemacht werden. (Observer Pattern)
\\
Der Client sollte im ,,idle''-Mode möglichst keine Ressourcen verschwenden und auch beim 
disconnecten und conntecten die entprechenden Änderungen anderen Teilen des Programms mitteilen
können(Observer Pattern)\footnote{http://de.wikipedia.org/wiki/Observer\_(Entwurfsmuster)}. 
\\

Das MPD Protokoll \footnote{http://www.musicpd.org/doc/protocol/index.html} bietet folgende Möglichkeiten das zu realisieren
\begin{description}
    \item [Periodisch] (zB. alle 500ms) das ,,status'' command absetzen und nach Bedarf auch commands wie ,,currentsong''
        senden
        \\
        \emph{Problem:} Bei langsamen Netzwerkverbindungen erzeugt dies unnötige Netzwerklast 
        Prinzipiell würde sich auf diese Art jedoch die z. B. Musik Bitrate anzeigen lassen, es ist jedoch ein
        wenig komfortabler Weg da hier wieder einmal das Rad neu erfunden werden müsste.
    \item [Nutzung der ,,idle'' und ,,noidle'' commands:]
        ,,idle'' versetzt die Verbindung zum Server in einen Schlafzustand, sobald ,,events'' wie 'player' (also z. B. pause oder play) 
        eintreten, wacht die Verbindung aus diesem Zustand auf und sendet an den Client eine Liste der Events die aufgetreten sind:

\lstinputlisting[language=bash]{state.txt}

        Einschränkung: Während die Verbindung im idle mode ist kann kein reguläres Kommando wie ,,play'' gesendet werden!
        Sollte man es doch tun wird man vom Server augenblicklich mit einem Disconnect belohnt.
        Die einzige Möglichkeit aus dem idle mode aufzuwachen ist das 'noidle' command das gesendet werden
        kann während die verbindung schlafen gelegt wurde.
        Jedoch gibt es auch hier ein Problem, denn das ,,idle'' command blockiert, sprich es sendet kein ,,OK'' zurück zum Sender.
        Ein Warten auf dieses ,,OK'' würde mit den Wunsch eine bedienbare Oberfläche zu haben kollidieren.
\end{description}

Prinzipiell gibt es 2 Möglichkeiten dieses Problem zu lösen:
\begin{itemize}
    \item Man hält zwei Verbindungen zum Server, eine die Kommandos sendet, eine die stets im ,,idle'' mode liegt,
        Für die Realisierung müssten Threads herangezogen werden. Ein Thread würde dann im Hintergrund auf events lauschen,
        der andere würde zum Abschicken der Kommandos benutzt werden.
        Problem: Es müssen 2 Verbindungen gehandelt werden, was wiederum ein Mehraufwand an Code bedeutet.
        Desweiteren werden Threads benötigt die auch in anderen Bereichen des Programms Lockingmechanismen bedeuten würden.
    \item Man hält eine asynchrone verbindung zu dem server.
        Diese kann das 'idle' command zum server schicken, returned aber sofort. Um nun eine Liste der events zu bekommen setzt man 
        einen ,,Watchdog''auf die asynchrone verbindung an (Vergleiche dazu den Systemaufrug 'man 3 poll'). Da poll() ebenfalls den
        aufrufenden Prozess blockiert, wird Glib::signal\_io() benutzt, das sich in den laufenden MainLoop (*) einhängt und eine 
        Callbackfunktion aufruft sobald auf der verbindung etwas interessantes passiert. Da während des Wartens der MainLoop weiterarbeitet,
        bleibt die GUI (und andere Module) aktiv und benutzbar.
        Problem: Vor dem Senden eines Kommandos wie ,,play'' muss der idle mode verlassen werden.
        Lösung: Man kann das ,,noidle'' Kommando zum verlassen senden, und nach dem Absenden des eigentlichen Kommandos wieder den idle-mode betreten.
\end{itemize}

%* MainLoop: Vergleiche Event Dispatcher auf wikipedia. Gtk+ benutzt intern einen MainLoop um auf die user ergebnisse reagieren zu können.
%            Desweiteren kann man eigene events in den Loop einhängen, wie beispielsweise ein Timeoutevent das alle 500ms ausgeführ

\lstinputlisting[language=bash]{telnet.txt}


Die Idee zu dieser Implementierung (speziell das Benutzen einer asynchronen Verbindung), kommt von ,,ncmpc'',
der inoffiziellen offiziellen Referenzimplementierung des MPD Mit-Authors \emph{Max Kellermann}.
Vergleiche \href{http://mpd.wikia.com/wiki/Client:Ncmpc}{ncmpc quellcode}: src/gidle.c und src/mpdclient.c


\section{Aufbau des Clients}
Aus den oben genannten Anforderungen kann eine grobe Architektur abgeleitet werden:

%Hier ein erstes Klassendiagramm zu  
%     * BaseClient
%     * Client 
%     * Connection 
%     * Listener
%     * NotifyData
%bzw. deren Verbindung

\subsection{Hauptklassen}

\subsubsection{BaseClient}
\begin{itemize}
    \item Kann nicht selbst instanziert werden.
    \item Verwaltet connect / disconnect und reconnect vorgänge
    \item Bietet Funktionen zum einfachen verlassen und eintreten des idlemodes an 
    \item Implementiert keine konkreten Kommandos die er an den server schicken kann
    \item Geht die verbindung verloren (ohne dass \emph{disconnect()} explizit aufgerufen wurde),
        so versucht er periodisch sich zu reconnecten.
\end{itemize}

\subsubsection{Listener}
\begin{itemize}
    \item Verwaltet das ein- und austreten aus dem Idlemode
    \item Parst die Responseliste (also changed: player)
    \item Verfügt über ein ,,EventNotifer'' (ein sigc::signal)
        Module können sich über connect() registrieren,
        bemerkt der Listener events so ruft er emit() auf dem signal auf
        und teilt allen anderen Modulen so mit welche events geschehen sind.
    \item Zudem bietet er eine Möglichkeit ein update zur forcen, das heißt ,,künstlich'' alle
        möglichen events auzulösen was nützlich zum Initialisieren ist (force\_update())
    \item Bei einem connect vorgang wird eine Instanz des Listeners instanziert und 
        sofort der idlemodus betreten
    \item Bei einem disconnect wird der Listener gelöscht.
\end{itemize}

\subsubsection{Connection}
\begin{itemize}
    \item Ein Wrapper um die mpd\_connection Struktur von libmpdclient
    \item Ruft letzendlich mpd\_connection\_new() auf 
    \item Bietet eine Schnittstelle um sich über Fehler informieren zu lassen (signal\_error())
    \item Bietet eine get\_connection() methode die bei jedem aufruf prüft ob fehler passiert sind
        In diesem Falle versucht MPD::Connection den Fehler zu bereinigen (falls ein nicht fataler Fehler war).
        Anschließend benachrichtigt MPD::Connection alle module die sich vorher über signal\_error() 
        registriert haben (wie der BaseClient es beispielsweise mit handle\_error() tut)
\end{itemize}

\subsubsection{Client}
\begin{itemize}
    \item Der Client erbt von BaseClient und implementiert konkrete Commandos wie ,,play'',,,random'' etc.
    \item Er bietet zudem Schnittstellen zur Befüllung der Datenbank, der Queue und des Playlistmanagers
    \item Er bietet die Methoden connect() und disconnect() 
    \item Ist in der config ,,settings.connection.autoconnect'' gesetzt so connected er sich automatisch.
    \item Er bietet zudem eine schnittstelle um sich beim listener zu registrieren und im falle von 
        änderungen des connection zustands benachrichtigt zu werden.
\end{itemize}

\subsubsection{NotifyData}
\begin{itemize}
    \item Speichert den Status, den aktuellen Song und die aktuelle Datenbankstatistik
    \item Der Listener...
        \begin{itemize}
            \item instanziert NotifyData im Konstruktor
            \item sagt NotifyData wann er sich updaten soll (update\_all())
            \item gibt eine Referenz auf NotifyData an alle registrierten Module weiter,
                damit diese konkrete Informationen beziehen können.
        \end{itemize}
\end{itemize}


\subsection{Weitere Klassen}
Desweiteren gibt es einige weitere Klassen die am Rande eine Rolle spielen,
und meist Objektorientierte Wrapperklassen für die C-Strukturen von libmpdclient bereitstellen.

\subsubsection{Song}

Die Song Klasse für Wrapper für mpd\_song Struktur und die dazugehörigen Klassen (libmpdclient).
Soll alle Funktionen von libmpdclient \footnote{http://www.musicpd.org/doc/libmpdclient/song\_8h.html} anbieten,
diese werden hier nur aufgelistet aber nicht erklärt da sie genau wie ihre Vorbilder funktionieren:

\begin{verbatim}
    const char * get\_path(void);
    const char * get\_tag(enum mpd\_tag\_type type, unsigned idx);
    unsigned get\_duration(void);
    time\_t get\_last\_modified(void);
    void set\_pos(unsigned pos);
    unsigned get\_pos(void);
    unsigned get\_id(void);
\end{verbatim}


MPD::Song soll zudem eine Funktion bieten um die Metadaten des Songs in einer printf änhlichen Art als String zurückzuliefern:
\begin{verbatim}
    Glib::ustring song\_format(const char* format, bool markup=true);
\end{verbatim}

Ein beispielhafter Aufruf:
\begin{verbatim}
    SomeSong.song\_format("Artist is by \${artist}") 
\end{verbatim}

Die folgenden Tagarten sollen dabei unterstützt werden (sie spiegeln in etwa die mpd\_tag\_type Enumeration von libmpdclient wieder)
\begin{itemize}
    \item artist
    \item title
    \item album
    \item track
    \item name
    \item data
    \item album\_artist
    \item genre
    \item composer
    \item performer
    \item comment
    \item disc
\end{itemize}
Ist ein Escapestring nicht bekannt, so wird er nicht escaped. Ist der tag nicht vorhanden soll mit "unknown" escaped werden.


\subsubsection{Directory}
Die Directory Klasse ist Wrapper für mpd\_directory C-Strukutr. Diese wird als Anzeige für ein Verzeichniss benutzt,
jedoch nicht als Container für andere Elemente.

Entsprechend implementiert bietet MPD::Directory nur:
\begin{verbatim}
    void get_path(void);
\end{verbatim}

Dies ist von der AbstractComposite vorgegeben.


\newpage
\subsubsection{Statistics}
Die Statistics Klasse ist Wrapper für mpd\_stats, implementiert gemäß
\\http://www.musicpd.org/doc/libmpdclient/stats\_8h.html
folgende Funktionen:
\begin{verbatim}
    unsigned get\_number\_of\_artists(void);
    unsigned get\_number\_of\_albums(void);
    unsigned get\_number\_of\_songs(void);
    unsigned long get\_uptime(void);
    unsigned long get\_db\_update\_time(void);
    unsigned long get\_play\_time(void);
    unsigned long get\_db\_play\_time(void);
\end{verbatim}


\subsubsection{Playlist}
Die Playlist Klasse ist Wrapper für die mpd\_playlist Struktur, implementiert von http://www.musicpd.org/doc/libmpdclient/playlist\_8h.html folgende Funktionen:
\begin{verbatim}
    const char * get\_path(void);
    time\_t get\_last\_modified(void);
\end{verbatim}

Bietet desweiteren funktionen zum:
Entfernen der Playlist vom Server (Das Playlistobjekt ist danach invalid):
\begin{verbatim}
    void remove(void);
\end{verbatim}

Laden der Playlist in die Queue:
\begin{verbatim}
    void load(void);
\end{verbatim}

Umbennen der Playlist:
\begin{verbatim}
    void rename(const char * new\_name);
\end{verbatim}

Hinzufügen von Songs zur Playlist:
\begin{verbatim}
    void add\_song(const char * uri);
    void add\_song(MPD::Song& song);
\end{verbatim}

Die genannten Funktionen benötigen müssen den idlemode verlassen können,
daher leitet MPD::Playlist von AbstractClientExtension ab.

\subsubsection{AudioOutput}
Die AudioOutput Klasse ist ein Wrapper für mpd\_output, implementiert von http://www.musicpd.org/doc/libmpdclient/output\_8h.html folgende Funktionen:
\begin{verbatim}
    unsigned get\_id(void);
    const char * get\_name(void);
    bool get\_enabled(void);
\end{verbatim}

Bietet desweiteren funktionen zum:
\begin{itemize}
    \item Enablen des Ausgabegerätes:
        \begin{verbatim}
            bool enable(void);
        \end{verbatim}
    \item Disablen des Ausgabegerätes:
        \begin{verbatim}
            bool disable(void);
        \end{verbatim}
\end{itemize}


Die genannten Funktionen benötigen müssen den idlemode verlassen können,
daher leitet MPD::AudioOutput von AbstractClientExtension ab. 

\subsection{Abstrakte Klassen}
\subsubsection{AbstractClientExtension}
Diese abstrakte Klasse erlaubt abgeleiteten Klasse ähnlich zum BaseClient eigene Kommandos zu implementieren.
Wird von MPD::Playlist und MPD::AudioOutput benutzt
%TODO 


\subsubsection{AbstractClientUser}
\begin{itemize}
    \item Verwaltet einen Pointer auf die MPD::Client Klasse,
        so dass der Anwender der Klasse dies nicht selbst tun muss.
    \item Leitet man ab so müssen folgenden Methoden implementiert werden:
        \begin{verbatim}
            void on\_client\_update(enum mpd\_idle event, MPD::NotifyData& data);
        \end{verbatim}  

        Wird aufgerufen sobald der Listener eine Änderunge feststellt,
        siehe weiter unten "Interaktion des Clients mit anderen Modulen" für eine genauere Erklärung.
        \begin{verbatim}
            void on\_connection\_change(bool server\_changed, bool is\_connected);
        \end{verbatim}

        Wird aufgerufen sobald sich der verbunden/getrennt hat. Im ersten Fall
        ist is\_connected true, im anderen false. Sollte sich der Client verbunden haben,
        und der neue Server entspricht nicht mehr dem neuen so ist auch server\_changed true.
        Dies ist automatisch wahr beim ersten Start.
        Diese werden automatisch durch Ableiten von AbstractClientUser registriert.
        Weiterhin können alle Klassen über den mp\_Client Pointer auf den Client zugreifen.
\end{itemize}


\subsubsection{AbstractItemlist}
Für bestimmte Client funktionen muss eine Nutzerklasse von AbstractItemlist ableiten.
Leitet man ab so muss die Methode add\_item(AbstractComposite * data) implementiert werden. 
Je nach Bedarf kann über dynamic\_cast<Zieltyp*>(data) der entsprechende Datentyp rausgecasted werden.
Beim Aufruf von MPD::Client::fill\_queue ruft der Client die add\_item methode für jeden 
song den er vom server bekommt auf. Die ableitende Klasse kann diese dann verarbeiten.

Dadurch werden alle Methoden von AbstractItemGenerator (bzw. die Klassen die davon ableiten) benutzbar:
\begin{itemize}
    \item fill\_queue
    \item fill\_queue\_changes
    \item fill\_playlists
    \item fill\_ouputs
    \item fill\_filelist
\end{itemize} 

%<Klassendiagramm, bzw. Klassen die es verwenden von Doxygen nehmen>


\subsubsection{AbstractItemGenerator}
Lässt ableitende Klasse folgende Methoden implementieren:
Jede dieser Methoden ruft MPD::Playlist add\_item() von AbstractItemlist auf um ihre Resultate weiterzugeben.

%<Sequenzdiagramm>   
Holt alle Songs der aktuellen Queue.
\begin{verbatim}            
    void fill\_queue(AbstractItemlist& data\_model);
\end{verbatim}

Holt alle geänderten Songs in der Queue seit der Version last\_version. Die Position des ersten geänderten Songs wird in first\_pos gespeichert. 
\begin{verbatim}
    void fill\_queue\_changes(AbstractItemlist& data\_model, unsigned last\_version, unsigned& first\_pos);
\end{verbatim}

Holt alle gespeicherten Playlisten vom Server.
\begin{verbatim}              
    void fill\_playlists(AbstractItemlist& data\_model);
\end{verbatim}

Holt alle Audio Outputs vom Server.
\begin{verbatim}
    void fill\_outputs(AbstractItemlist& data\_model);
\end{verbatim}

Holt alle Songs und Directories aus der Datenbank im Pfad path (nicht rekursiv!)              
\begin{verbatim}
    void fill\_filelist(AbstractItemlist& data\_model, const char * path);
\end{verbatim}

%<Klassendiagramm, bzw. Klassen die es verwenden von Doxygen nehmen>

%-------------------------------------------

\subsubsection{AbstractComposite}
Vereinheitlicht Zugriff auf Komponenten verschiedenen Types.
Die abstrakte Klasse zwingt seine Kinder dazu eine \emph{get\_path()} zu implementieren die die Lage im virtuellen Filesystem des Servers angibt.
Der Hauptnutznieser dieser Klasse ist der Databasebrowser, bzw. den dahinter gelagerten Cache Songs und Verzeichnisse gleich zu behandeln.

Die erbende Klasse muss im Konstruktor angeben ob es sich bei der Klasse um ein ,,File'' (\emph{true} für MPD::Song) oder um einen ,,Container'' (\emph{false} für MPD::Directory) handelt.
Diese ,,is\_leaf'' Eigenschaft kann später mit der Funktion \emph{is\_leaf()} abgefragt werden.

% <Klassendiagramm für alle Klassen die von AbstractComposite erben, siehe Doxygen>

%=============================================

%% Übergang zu GUI Zeugs..

\section{Interaktion des Clients mit anderen Modulen}
\begin{itemize}
    \item Die meisten GUI Klassen leiten von AbstractClientUser ab und speichern daher eine Referenz auf eine Instanz von MPD::Client
        Sie können daher Funktionen wie queue\_add() direkt aufrufen.
    \item AbstractClientUser zwingt die abgeleitenden Klassen folgende Funktionen zu implementieren: 
        \begin{verbatim} 
            void on_client_update(mpd_idle event, MPD::NotifyData& data)
            void on_connection_change(bool server_changed, bool is_connected)
        \end{verbatim}

        1) wird aufgerufen sobald der Listener ein Event festgestellt hat. Für jedes eingetretene Event wird 1)
        einmal aufgerufen. 'event' ist dabei eine Enumeration aller möglichen Events, die von libmpdclient 
        vorgegeben werden. %(Siehe auch http://www.musicpd.org/doc/libmpdclient/idle\_8h.html#a3378f7a24c714d7cb1058232330d7a1c)
        'data' ist eine Referenz auf eine Instanz von MPD::NotifyData. Die benutzenden Klassen können mit
        \begin{itemize} 
            \item get\_status() gibt den aktuellen MPD::Status
            \item get\_song() gibt den aktuellen MPD::Song
            \item get\_statistics() gibt die aktuellen MPD::Statistics
        \end{itemize} 
        so bei Änderungen sofort die aktuellen Änderungen auslesen.



        2) on\_connection\_change wird vom Client aufgerufen sobald die Verbindung verloren geht.
        Dabei zeigt der übergebene boolean Wert ,,is\_connected'' an ob man connected wurde, oder disconnected wurde.
        ,,server\_changed'' soll dann anzeigen ob der Server derselbe ist beim zuvor geschehenen Connectvorgang.
        Dies ist beim ersten Start stets wahr. ,,server+\_changed'' kann nicht wahr sein wenn ,,is\_connected'' falsch ist.
    \item Ableitung von den oben beschriebenen abstrakten Klassen AbstractItemlist und AbstractFilebrowser, um alle Funktionen von AbstractItemGenerator nutzen zu können  
\end{itemize}

\section{Config}
\section{GUI Elementklassen} 
\section{Browserimplementierungen}


\section{Glossar}
\subsection{Mainloop}
In Zusammenhang mit Freya handelt sich stets um Glib Implementation eines Mainloops.
Zum Verständniss sei daher die Lektüre der \href{http://developer.gnome.org/glib/2.30/glib-The-Main-Event-Loop.html#glib-The-Main-Event-Loop.description}{Glib Dokumentation} empfohlen.


\end{document}
