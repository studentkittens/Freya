\section{GUI Elementklassen}

\subsection{Hauptklassen}
Der GManager Namespace enthält Klassen die der Verwaltung und Kontrolle des Hauptfensters von Freya dienen,
jedoch nicht für den eigentlichen Inhalt des Hauptfensters (dies wird vom Browser namespace getan)
Alle Klassen gehören nach dem MVC Paradigma der Controllerschicht an.


\subsubsection{BrowserList}
Zeigt eine Liste von Browsern in der Sidebar.
\begin{itemize} 
\item Bietet eine add() Methode die eine Referenz auf AbstractBrowser erwartet und fügt in der Sidebar hinzu.
\item set() setzt den Browser temporär, ohne ihn hinzuzufügen.
\end{itemize}
Benutzt alle Methoden von AbstractBrowser um dise entsprechen anzuzeigen:
\\
Der Container der im Hauptbereich beim wechseln angezeigt wird
\begin{verbatim}
  Gtk::Widget * get_container();
\end{verbatim}
Welcher Name soll in der Sidebar angezeigt werden?
\begin{verbatim}
  Glib::ustring get_name();
\end{verbatim}
Welche Gtk::Stock::ID soll in der Liste angezeigt werden?
\begin{verbatim}
  Gtk::Stock::ID get_icon_stock_id();
\end{verbatim} 
Ist sichtbar in der Leiste?
\begin{verbatim}
  bool is_visible(); 
\end{verbatim}
Benötigt dieser Browser eine Verbindung zum funktionieren?
\begin{verbatim}
  bool needs_connection(); 
\end{verbatim}

Als View wird ein Gtk::TreeView benutzt, die Browserreferenzen werden in einem Gtk::ListStore gespeichert,
was damit das Model darstellt. 

\subsubsection{Heartbeat}
Sendet alle 500ms ein Signal aus, und summiert die bisher vergangene Zeit.
Dies ist nützlich bei Anzeigen wie der Sekundenanzeige.
Über signal\_client\_update() können sich Klienten registrieren:
\begin{verbatim}
  Heartbeat.signal_client_update().connect(<funktionspointer>)
\end{verbatim}
Der angegebene Funktionspointer wird dann aufgerufen und muss folgender Signatur entsprechen:
\begin{verbatim}
  void func(double time)
  {
      ...
  }
\end{verbatim}

Der übergebene Parameter ist die Zeit die seit dem Instanzieren vergangen ist. 
Sie kann durch folgende Funktionen verändert werden:
\begin{verbatim}
void pause(void)  - Setzt das Zählen aus
void play(void)   - Fängt damit wieder an
void reset(void)  - Fängt von 0 wieder an
void get(void)    - Bekommt die jetzige Zeit
void set(void)    - Setzt die jetzige Zeit absolut und zählt von dort weiter
\end{verbatim}

Zusätzlich stoppt die Heartbeat klasse das zählen wenn der client das playback pausiert.
Wird es fortgesetzt, so so wird play() aufgerufen. 
Zusätzlich wird bei jedem client update der Zähler an der vergangen Zeit im gerade spielenden Song justiert.

\subsubsection{MenuList}
Kontrolliert die Anzeige (Sensitivität) und Steuerung der Menüleiste.

\subsubsection{NotifyManager}
Kontrolliert die Anzeige von Notifications, bei entsprechenden events.
Greift dabei auf die Notifylib zurück.

\subsubsection{PlaybackButtons}
Kontrolliert die Anzeige der oberen rechten Playbackbuttons Stop, Play/Pause, Next, Previous
Das Icon des Playbuttons wird entsprechend geändert falls das Playback pausiert ist,
bzw. fortgesetzt wird.

\subsubsection{Statusbar}
Kontrolliert die Anzeige der Statusbar (was den Text miteinfasst). 
Benutzt die Heartbeatklasse um die Zeitanzeige zu aktualisieren. Ansonsten bekommt es alle Informationen rein vom Client update.

\subsubsection{StatusIcons}
Kontrolliert Anzeige und Handling der Icons unter der Sidebar.

\subsubsection{Timeslide}
Zeigt und Kontrolliert die aktuelle Zeit innerhalb des momentan spielenden Liedes.
Bei Klicken innerhalb der Timeline wird zur entsprechenden Stelle im Song gesprungen.

\subsubsection{TitleLabel}
Verwaltet und kontrolliert Anzeige des Titels bzw. Artist und Albums in der Titelleiste und Die ,,Next Song'' Anzeige unter der Sidebar.

\subsubsection{Trayicon}
Verwaltet und kontrolliert Anzeige und Interaktion des Trayicons das optional angezeigt werden kann.
Dazu gehört auch die Definition und Anzeige des Popupmenüs, weshalb die Klasse von Browser::BasePopup ableitet.

\subsubsection{Volumebutton}
Verwaltet und Kontrolliert die Anzeige des Volumebuttons. Aus Performancegründen sollen nur alle 0.05 Sekunden Volumeänderungen erlaubt.

\subsubsection{Window}
Verwaltet das Hauptfenster von Freya.
Falls das verstecken des Fensters beim Schließen gewünscht ist (\emph{,,settings.trayicon.totrayonclose''} ist 1), so wird Gtk::Window::hide() aufgerufen.
Andernfalls wird einfach der Mainloop beendet wodurch die Kontrolle zur main() Methode zurückkehrt.
Zudem wird eine get\_window() Methode bereitgestellt die das darunterliegende Fenster (ein Gtk::Window) zurückgibt.
Der Mainloop zB. benötigt das als Startargument.

