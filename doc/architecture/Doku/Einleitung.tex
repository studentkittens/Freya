\chapter{Einleitung}
Ziel dieser Studienarbeit ist die vollständige Bearbeitung einer vorgegebenen Aufgabenstellung
nach einem selbst gewählten Vorgehensmodell. Die Aufgabenstellung schreibt vor, sich in einer
Gruppe zusammen zu finden und gemeinsam ein Software-Projekt zu bearbeiten und dabei strukturiert
 und professionell vorzugehen.
\begin{quote}
    \section{Rahmenbedingungen}
    \renewcommand{\labelitemi}{•}
    \begin{itemize}
        \item Persistente Datenspeicherung
        \begin{itemize}
	    \item Datei oder Datenbank (wenn schon bekannt)
        \end{itemize}
        \item Netzwerk-Programmierung
        \begin{itemize}
	    \item Eine verteilte Architektur (z.B.: Client/Server)
        \end{itemize}
        \item GUI
        \begin{itemize}
	    \item Swing
	    \item Web-basiert
        \end{itemize}
    \end{itemize}
    \section{Prozess-Anforderungen}
    \begin{itemize}
        \item Dokumentation aller Phasen(Analyse bis Testen)
        \item Auswahl eines konkreten Prozessmodells
        \begin{itemize}
	    \item Z.B. sd\&m, M3, RUP, Agile Methoden ...
	    \item Begründung (warum dieser Prozess passt zu Ihrem System)
        \end{itemize}
        \item Erstellung der Dokumente und UML-Diagramme
        \begin{itemize}
	    \item Visio
	    \item UML Werkzeuge (freie Wahl)
        \end{itemize}
        \item Fertige Implementierung 
        \begin{itemize}
	    \item Es kann mehr spezifiziert sein als implementiert
        \end{itemize}
        \item Spezifikation von Testszenarien
        \begin{itemize}
	    \item und der Beleg der erfolgreichen Ausführung
        \end{itemize}
        \item Lauffähiges System
    \end{itemize}
    \section{Mögliche Themen}
    \begin{itemize}
        \item CRM Systeme
        \begin{itemize}
	    \item Bibliothek
	    \item Musikshop
	    \item ...
        \end{itemize}
        \item Kommunikationssysteme
	\item Chat-Variationen (Skype, etc.)
	\item File-Verwaltungs-Systeme (eigener Cloud-Dienst)
	\item ...
    \end{itemize}
    \item Portale
    \begin{itemize}
	\item Mitfahrgelegenheit
	\item Dating-Agentur ;)
	\item ...
    \end{itemize}
\footnote{Folie Anforderungen, Autor Prof. Dr. Philipp Schaible, WS 2011/2012, Inf 3}
\end{quote}
Diese Arbeit ist wichtig, um den Studenten zu zeigen, wie man in einem Team zusammenarbeitet und nach
Software-Engineering-Methoden qualitativ hochwertige Software erstellt. Es geht im Folgenden um einen
Music-Player-Daemon-Client (Näheres bitte der Definition entnehmen). Dieses Thema wird behandelt, da es
alle Rahmenbedingungen abdeckt und im Interesse der Autoren liegt. Die Besonderheit liegt darin, dass
sich diese Software nach Fertigstellung auch wirklich anwenden lässt. Ziel ist die Erweiterung der
Fähigkeitn im Bereich der Software Engineering sowie das Erlernen von Methoden für wissenschaftliches Arbeiten.



