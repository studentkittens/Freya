\chapter{Lastenheft}
\section{Zielbestimmungen}
Welche Ziele sollen durch den Einsatz der Software erreicht werden?\ \\ \\
Dem einzelnen Benutzer soll das abspielen von Musik über eine Netzwerkverbindung ermöglicht
werden, dabei soll die Steuerung von einem lokalen Client übernommen werden. Die Musik soll
in eine rzentralen Datenbank angelegt und über die Soundkarte eines Servers abgespielt werden.
Die Client-Rechner sollen die Ausgabe steuern und Abspiellisten auf dem Server verwalten
können. Die Bedienung soll für alle Benutzer sehr einfach und komfortabel über einen lokalen
Client realisiert werden. Bei jedem Start des Clients, soll die letzte Sitzung wiederhergestellt
werden, falls keine Daten eine beendeten Sitzung gefunden werden, sollen Standarteinstellungen
verwendet werden.\ \\ \\
Standartmäßig sollen den Benutzern folgende Funktionen zuf Verfügung stehen:
\renewcommand{\labelitemi}{•}
\begin{itemize}
        \item Abspielen von Musik
        \item Steuerung von Musik (Play, Stop, Skip, ...)
        \item Decodieren von Musik
        \item Input-Stream via HTTP
\end{itemize}
Weitere Funktionen müssen modular integrierbar sein, allerdings müssen sie noch nicht implementiert
werden. Einige Beispiele für weitere Funktionen wären:
\begin{itemize}
	\item Finden von Album-Informationen
	\item Profil-Steuerung
	\item Visualisierung
\end{itemize}
Die Systemsprache soll auf Englisch festgelegt werden.
\subsection{Projektbeteiligte}
Wer soll an dem Projekt teilnehmen?
\begin{itemize}
	\item Christopher Pahl
	\item Christoph Piechula
	\item Eduard Schneider
	\item Marc Tigges
\end{itemize}
\section{Produkteinsatz}
Für welche Anwendungsbereiche und Zielgruppe ist die Software vorgesehen?\ \\ \\
Der MPD-Client ist nicht auf bestimmte Gewerbe beschränkt, ein jeder soll diesen
Client verwenden können. Grundlage für die Verwendung der Software ist die General
Public License (GPL) Version 3 vom 29 Juni 2007.\ \\ \\
Definition der GPL v3:
\begin{center}
	http://www.gnu.org/licenses/gpl.html
\end{center}
\subsection{Anwendungsbereiche}
Einzelpersonen verwenden dieses System überall da, wo mit
einem Unix-artigen Betriebssystem Musik abgespielt werden soll.
Das wären z.B. Personal Computer, Musikanlagen, Laptops und evtl.
sogar diverse Smartphones
\subsection{Zielgruppen}
Personengruppen die komfortabel von überall aus auf ihre Musik und Playlist zugreifen
wollen ohne diese jedes mal aufwändig synchronisieren zu müssen (z.B. durch Abgleich von Datenträgern).\ \\ \\
Es werden Basiskenntnisse zum Aufbau einer Netzwerkverbindung und zur Nutzung des Internets vorausgesetzt.
Aufgrund der für das System vorgesehenen Betriebsumgebung sind ebenso Kenntnisse im Umgang mit Unix nötig.\ \\ \\
Der Benutzer muss die Systemsprache Englisch beherrschen.
\subsection{Betriebsbedingungen}
Das System soll sich bezüglich der Betriebsbedingungen nicht sonderlich von vergleichbaren Systemen bzw.
Anwendungen unterscheiden und dementsprechend folgend Punkte erfüllen:
\begin{itemize}
        \item Betriebsdauer: Täglich, 24 Stunden
        \item Keinerlei Wartung soll nötig sein
        \item Sicherungen der Konfiguration müssen vom Benutzer vorgenommen werden
\end{itemize}
\section{Produktumgebung}
\subsection{Software}
Softwareabhängigkeiten sollen durch den Entwickler bestimmt werden.
Dies gewährleistet, dass der Entwickler diesbezüglich nicht eingeschränkt wird
und somit mehr Möglichkeiten hat.
\subsection{Hardware}
Das Produkt soll möglichst wenig Anforderungen and die Hardware stellen, da
die Software eventuell auch auf sehr Hardwarearmen Geräten (wie z.B. Smartphones)
verwendet werden soll.
\subsection{Orgware}
Es soll nach Möglichkeit keine Orgware von nöten sein. Der Nutzer der Software soll sich
um möglichst wenig nebenläufiges zu kümmern haben.
\section{Produktfunktionen}
Welche sind die Hauptfunktionen aus Sicht des Auftraggebers?
\subsection{Allgemein}
Beim ersten Start des Systems soll eine Standard-Konfiguration geladen werden und die Verbindungseinstellungen
zu einem MPD-Server müssen vorgenommen werden. Bei jedem weiteren Start soll die Konfiguration geladen werden,
die vom Benutzer erstellt wurde, falls diese denn lokal gefunden werden kann. Der Benutzer soll sämtliche
Einstellungen selbstverständlich zu jeder Zeit ändern können.
Natürlich sollen alle üblichen Musik Abspielfunktionen vorhanden sein, dazu gehört Play, Stop, Previous
und Next. Aber auch erweiterte Funktionen wie Repeat, Consume und Random sollen einstellbar sein.
Der Benutzer soll über die Software direkten Zugriff auf sein Dateisystem haben, um nach Musik zu suchen und
diese abspielen zu können. Aus dem Dateisystem heraus soll der Nutzer ebenfalls die Möglichkeit haben, Musik-Dateien
direkt zu Playlisten und Warteschlangen hinzuzufügen.
Verbindungseinstellungen müssen auf möglichst einfache Art und Weise vorgenommen werden können, wenn möglich
sollte dem Nutzer eine Liste von verfügbaren Servern angezeigt werden. 
Dem Nutzer soll ermöglicht werden, dass er nach bestimmten Titeln, Alben oder Interpreten suchen kann, da 
es mit dieser Software möglich ist, auch sehr große Musik-Datenbanken zu steuern.
Administratorfunktionen müssen nicht implementiert werden, das sie vom Unix-System übernommen werden.
\section{Produktdaten}
Welche Daten sollen persistent gespeichert werden?\ \\ \\
Die vom Benutzer vorgenommenen Verbindungseinstellungen und Client spezifischen Einstellungen,
sollen auf dem Rechner lokal und persistent gespeichert werden. Nur so kann ermöglicht werden,
dass nach jedem Start des Systems diese Einstellungen geladen und übernommen werden können.\ \\
Außerdem soll eine Log-Datei auf den einzelnen Rechnern angelegt werden, die dieses System
verwenden. In dieser Log-Datei werden Nachrichten des Systems gespeichert, um eventuelle Fehler
leicht finden und beheben zu können. Es soll zusätzlich der Zustand des Systems abgespeichert werden,
wenn das System beendet wird um das System beim nächsten Start in diesen Zustand versetzen zu können.\ \\
Dem Nutzer sollen viele verschiedene Informationen angezeigt werden, nicht nur Standardinformationen
wie Titel, Album und Interpret, sondern auch Musik-Qualität, -Länge und Lautstärke.
Es soll außerdem eine primitive Statistik implementiert werden die anzeigt, wie viele Lieder, Alben und
Interpreten in der Datenbank vorhanden sind, wie lange man schon mit dem Server verbunden ist und wie 
lange die gesamte Abspielzeit aller Lieder in der Datenbank dauert.\ \\
Eine Profilverwaltung muss nicht implementiert werden, dies soll über das Unix-System geregelt werden.\ \\
Eine lokale Datenbank muss ebenfalls nicht vorhanden sein, dies wird durch den MPD-Server ermöglicht.\ \\
\section{Qualitätsanforderungen}
Die Software soll natürlich von hoher Qualität sein. Hierfür sollen folgende
Anforderungen erfüllt werden:\ \\ \\
Die Software soll korrekt sein, d. h. keine Fehler enthalten.
Sie soll aber auch, für den Fall das dennoch Fehler auftreten, robust
und tollerant auf diese reagieren. Außerdem spielt die Wartbarkeit 
eine wichtige Rolle, falls sich die Softwareumgebung des MPD-Clients
ändert, muss dieser leicht angepasst werden können.
Der Client soll leicht und intuitiv bedienbar sein. Sollte es Funktionen
geben, die nicht unter den Begriff "Standard" fallen, sollte eine kanppe
und präzise Beschreibung der Funktion vorhanden sein.
Nach allem dem muss die Software tortzdem noch Effizient sein, geringe
Wartezeiten, wenig Hardwareanforderungen, etc.
Das Design der Software muss zwar ansprechend sein, ist im edeffekt allerdings
zweitrangig.
\section{Ergänzungen}
\subsection{Realisierung}
Das System muss mit den Programmiersprachen C und/oder C++ realisiert werden. Dabei ist auf
Objektorientierung zu achten, um Modularität und Wartbarkeit gewährleisten zu können.
Es können beliebige Entwicklungsumgebungen verwendet werden. Um einfaches und sicheres arbeiten
ermöglichen zu können, soll die Versionsverwaltungssoftware \"git\" benutzt werden, um die
Entwicklungsdateien zu speichern und zu bearbeiten. Zu dem Projekt soll eine ausführliche
Dokumentation erstellt werden, um dauerhafte Wartbarkeit und Anpassung des MPD-Client  gewährleisten
zu können, dazu gehören auch entsprechende Software-Diagramme (wie z.B. UML).
\subsection{Die nächste Version}
Aufgrund des modularen Aufbaus kann das System beliebig oft und in verschiedene Richtungen weiterentwickelt werden.
