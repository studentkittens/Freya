\chapter{Lastenheft}
\section{Zielbestimmungen}
Welche Ziele sollen durch den Einsatz der Software erreicht werden?\ \\ \\
Dem einzelnen Benutzer soll das abspielen von Musik über eine Netzwerkverbindung ermöglicht
werden, dabei soll die Steuerung von einem lokalen Client übernommen werden. Die Musik soll
in eine rzentralen Datenbank angelegt und über die Soundkarte eines Servers abgespielt werden.
Die Client-Rechner sollen die Ausgabe steuern und Abspiellisten auf dem Server verwalten
können. Die Bedienung soll für alle Benutzer sehr einfach und komfortabel über einen lokalen
Client realisiert werden. Bei jedem Start des Clients, soll die letzte Sitzung wiederhergestellt
werden, falls keine Daten eine beendeten Sitzung gefunden werden, sollen Standarteinstellungen
verwendet werden.\ \\ \\
Standartmäßig sollen den Benutzern folgende Funktionen zuf Verfügung stehen:
\renewcommand{\labelitemi}{•}
\begin{itemize}
        \item Abspielen von Musik
        \item Steuerung von Musik (Play, Stop, Skip, ...)
        \item Decodieren von Musik
        \item Input-Stream via HTTP
\end{itemize}
Weitere Funktionen müssen modular integrierbar sein, allerdings müssen sie noch nicht implementiert
werden. Einige Beispiele für weitere Funktionen wären:
\begin{itemize}
	\item Finden von Album-Informationen
	\item Profil-Steuerung
	\item Visualisierung
\end{itemize}
Die Systemsprache soll auf Englisch festgelegt werden.
\subsection{Projektbeteiligte}
Wer soll an dem Projekt teilnehmen?
\begin{itemize}
	\item Christopher Pahl
	\item Christoph Piechula
	\item Eduard Schneider
	\item Marc Tigges
\end{itemize}
\section{Produkteinsatz}
Für welche Anwendungsbereiche und Zielgruppe ist die Software vorgesehen?\ \\ \\
Der MPD-Client ist nicht auf bestimmte Gewerbe beschränkt, ein jeder soll diesen
Client verwenden können. Grundlage für die Verwendung der Software ist die General
Public License (GPL) Version 3 vom 29 Juni 2007.\ \\ \\
Definition der GPL v3:
\begin{center}
	http://www.gnu.org/licenses/gpl.html
\end{center}
Die Software soll überall da eingesetzt werden, wo Musik abgespielt werden soll. 
Dabei ist man nicht auf einen Rechner beschränkt, auch Fernseher und Musik-Spieler
mit Internetzugang, entsprechender Softwareunterstützung und Audio output können 
theoretisch ein solches Programm verwenden.\ \\ \\
Hauptsächlich soll sich diese Software allerdings an Nutzer eines Rechners mit einem Unix-
artigen System richten. Des weiteren soll die Zielgruppe vorerst auf Benutzer beschränkt
sein, die Englisch verstehen.
\section{Produktfunktionen}
Welche sind die Hauptfunktionen aus Sicht des Auftraggebers?\ \\ \\
\subsection{Benutzerfunktionen}
Beim ersten Start des Systems soll eine Standard-Konfiguration geladen werden und die Verbindungseinstellungen
zu einem MPD-Server müssen vorgenommen werden. Bei jedem weiteren Start soll die Konfiguration geladen werden,
die vom Benutzer erstellt wurde, falls diese denn lokal gefunden werden kann. Der Benutzer soll sämtliche
Einstellungen selbstverständlich zu jeder Zeit ändern können.
\subsubsection{Starten und Beenden}
\begin{itemize}
	\item F\_0010 Der Benutzer kann das System zu jedem Zeitpunkt starten.
	\item F\_0020 Der Benutzer kann das System zu jedem Zeitpunkt beenden.
\end{itemize}
\subsubsection{Persönliche Daten}
Ein Benutzer verfügt über eine persönliche Verbindungseinstellung zum gewünschten MPD-Server.
Diese Daten können von dem Benutzer zu jeder Zeit angepasst werden.
\begin{itemize}
	\item F\_0110 Der Benutzer kann sich zu jeder Zeit seine Verbindungsdaten anzeigen lassen.
	\item F\_0120 Der Benutzer kann zu jeder Zeit seine persönlichen Daten anpassen.
\end{itemize}
\subsubsection{Persönliches Profil}
Da die Software auf Unix-artige Systeme beschränkt werden soll, geht ein angenehmer Vorteil mit einher, nämlich das
eine Profil-Verwaltung seitens des MPD-Clients nicht implementiert werden muss. Die verschiedenen Profile werden
durch die verschiedenen Profile des gesamten Betriebssystems definiert und differenziert.
\subsubsection{Persönliche Datenbank}
Eine persönliche Datenbank soll lokal nicht vorhanden sein. Die Datenbank des Benutzers befindet sich auf dem MPD-Server.
Einzig und alleine modulare Erweiterungen des MPD-Clients können lokale Datenbank-Implementierungen erfordern.
\subsubsection{Kommunikation (Chat)}
Kommunikation von MPD-Client zu MPD-Client kann theoretisch implementiert werden, eine solche Schnittstelle ist vorhanden.
Allerdings soll hierauf verzichtet werden, da im Vordergrund das Abspielen und Verwalten von Musik steht und es deutlich
einfachere und bessere Systeme gibt, mit Hilfe derer man kommunizieren kann.
\subsubsection{Suchen}
Eine einfache Textsuche zum finden von Titeln, Alben oder Interpreten innerhalb der Abspiellisten soll implementiert werden.
\begin{itemize}
	\item F\_0210 Der Benutzer kann seine Queue durchsuchen.
\end{itemize}
\subsection{Administrator-Funktionen}
Durch das Unix-artige System soll auch der Administrator-Zugriff geregelt werden. Sobald sich der Benutzer im Unix System
als Administrator befindet, kann er auch den MPD-Client administrieren. Ein zusätzlicher Administrator-Modus muss also
nicht implementiert werden.
\section{Produktdaten}
Welche Daten sollen persistent gespeichert werden?\ \\ \\
Die vom Benutzer vorgenommenen Verbindungseinstellungen und Client spezifischen Einstellungen,
sollen auf dem Rechner lokal und persistent gespeichert werden. Nur so kann ermöglicht werden,
dass nach jedem Start des Systems diese Einstellungen geladen und übernommen werden können.\ \\
Außerdem soll eine Log-Datei auf den einzelnen Rechnern angelegt werden, die dieses System
verwenden. In dieser Log-Datei werden Nachrichten des Systems gespeichert, um eventuelle Fehler
leicht finden und beheben zu können. Es soll zusätzlich der Zustand des Systems abgespeichert werden,
wenn das System beendet wird um das System beim nächsten Start in diesen Zustand versetzen zu können.
\begin{itemize}
	\item D\_0010 Persönlichen Verbindungseinstellungen.
	\begin{itemize}
		\item Platzhalter
		\item Platzhalter
	\end{itemize}
	\item D\_0020 Client spezifische Einstellungen.
	\begin{itemize}
		\item Platzhalter
		\item Platzhalter
	\end{itemize}
	\item D\_0030 Eine Log-Datei.
	\begin{itemize}
		\item Platzhalter
		\item Platzhalter
	\end{itemize}
	\item D\_0040 Der Zustand.
	\begin{itemize}
		\item Platzhalter
		\item Platzhalter
	\end{itemize}
\end{itemize}
\section{Produktleistungen}
Werden für bestimmte Funktionen besondere Ansprüche in Bezug auf Zeit, Datenumfang oder Genauigkeit gestellt?\ \\ \\
Wenn das System beendet wird, soll der aktuelle Zustand des Systems gespeichert werden.
\begin{itemize}
	\item L\_0010 Speicherung des Systemzustandes
\end{itemize}
Es soll möglichst wenig Speicher gebraucht werden, die CPU soll möglichst wenig belastet werden und der Netzwerk-Traffic
soll gering gehalten werden.
\begin{itemize}
	\item L\_0020 Möglichst wenig Ressourchen-Verbrauch
\end{itemize}
Die Geschwindigkeit der Software ist auch abhänig von der jeweiligen Server-Lokation, der Benutzer wählt den Server
d.h. somit ist auch der Benutzer teil-verantworltich für die Geschwindigkeit.\ \\ \\
Der Status eines Liedes (Liedposition) wird alle 500 ms aktualisiert.
\begin{itemize}
	\item L\_0030 Lokaler Heartbeat alle 500 ms
\end{itemize}
\section{Qualitätsanforderungen}
Welche qualitativen Anforderungen sind von besonderer Bedeutung?\ \\ \\
Es soll auf folgende Priorität unter den Qualitätsanforderungen geachtet werden,
dabei ist das erste Element das wichtigste und das letzte das unwichtigste.\ \\ \\
Priorität 1: Robustheit\ \\
Priorität 2: Zuverlässigkeit\ \\
Priorität 3: Effizienz\ \\
Priorität 4: Intuitive Benutzung\ \\
Priorität 5: Design\ \\ \\
\section{Ergänzungen}
\subsection{Realisierung}
Das System muss mit den Programmiersprachen C und/oder C++ realisiert werden. Dabei ist auf
Objektorientierung zu achten, um Modularität und Wartbarkeit gewährleisten zu können.
Es können beliebige Entwicklungsumgebungen verwendet werden. Um einfaches und sicheres arbeiten
ermöglichen zu können, soll die Versionsverwaltungssoftware \"git\" benutzt werden, um die
Entwicklungsdateien zu speichern und zu bearbeiten. Zu dem Projekt soll eine ausführliche
Dokumentation erstellt werden, um dauerhafte Wartbarkeit und Anpassung des MPD-Client  gewährleisten
zu können, dazu gehören auch entsprechende Software-Diagramme (wie z.B. UML).
\subsection{Die nächste Version}
Aufgrund des modularen Aufbaus kann das System beliebig oft und in verschiedene Richtungen weiterentwickelt werden.
