\chapter{Pflichtenheft}
\section{Zielbestimmungen}
\subsection{Projektbeteiligte}
Wer soll an dem Projekt teilnehmen?
\begin{itemize}
        \item Christopher Pahl
        \item Christoph Piechula
        \item Eduard Schneider
        \item Marc Tigges
\end{itemize}
\subsection{Muss-Kriterien}
\renewcommand{\labelitemi}{•}
\begin{itemize}
	\item Server-Verbindung
	\item Client-Einstellungen
	\item Musik-Steuerung
\end{itemize}
\subsection{Wunsch-Kriterien}
\begin{itemize}
	\item Liedinformationen taggen
	\item Musikstatistik
	\item Album Covers
	\item Liedsuche
\end{itemize}
\subsection{Abgrenzungskriterien}
\begin{itemize}
	\item Musik-Visualisierung
	\item Chat
	\item Social-Network-Schnittstelle
\end{itemize}
\section{Produkteinsatz}
Welche Anwendungsbereiche (Zweck), Zielgruppen (Wer mit welchen Qualifikationen), Betriebsbedingungen (Betriebszeit,
Aufsicht)?\ \\ \\
Der MPD-Client ist nicht auf bestimmte Gewerbe beschränkt, ein jeder soll diesen Client
verwenden können. Grundlage für die Verwendung der Software ist die General Public License (GPL)
Version 3 vom 29 Juni 2007.\ \\ \\
Definition der GPL:
\begin{center}
http://www.gnu.org/licenses/gpl.html
\end{center}
\subsection{Anwendungsbereiche}
Einzelpersonen verwenden dieses System überall da, wo mit 
einem Unix-artigen Betriebssystem Musik abgespielt werden soll.
Das wären z.B. Personal Computer, Musikanlagen, Laptops und evtl.
sogar diverse Smartphones
\subsection{Zielgruppen}
Personengruppen die komfortabel von überall aus auf ihre Musik und Playlist zugreifen
wollen ohne diese jedes mal aufwändig synchronisieren zu müssen (z.B. durch Abgleich von Datenträgern).\ \\ \\
Es werden Basiskenntnisse zum Aufbau einer Netzwerkverbindung und zur Nutzung des Internets vorausgesetzt.
Aufgrund der für das System vorgesehenen Betriebsumgebung sind ebenso Kenntnisse im Umgang mit Unix nötig.\ \\ \\
Der Benutzer muss die Systemsprache Englisch beherrschen.
\subsection{Betriebsbedingungen}
Das System soll sich bezüglich der Betriebsbedingungen nicht sonderlich von vergleichbaren Systemen bzw.
Anwendungen unterscheiden und dementsprechend folgend Punkte erfüllen:
\begin{itemize}
	\item Betriebsdauer: Täglich, 24 Stunden
	\item Keinerlei Wartung soll nötig sein
	\item Sicherungen der Konfiguration müssen vom Benutzer vorgenommen werden
\end{itemize}
\section{Produktumgebung}
\subsection{Software}
\begin{itemize}
	\item Avahi Daemon
	\item MPD-Client
\end{itemize}
Ein MPD-Server ist nicht unbedingt von nöten.
\subsection{Hardware}
Minimale Hardwareanforderungen: 500 Mhz, 512MB Ram, Festplattenspeicher < 1MB
Empfohlene Hardwareanforderungen: 1 Ghz, 512MB Ram, Festplattenspeicher < 1MB
\subsection{Orgware}
\begin{itemize}
	\item git (Versionsverwaltungssoftware)
	\item cmake (Buildsystem)
	\item doxygen (Dokumentation)
	\item Editor nach Wahl
	\item Glade (GUI)
\end{itemize}
\section{Produktfunktionen}
Funktionen des MPD-Clients.\ \\ \\
Beim ersten Start des Systems soll eine Standard-Konfiguration geladen werden und die Verbindungseinstellungen
zu einem MPD-Server müssen vorgenommen werden. Bei jedem weiteren Start soll die Konfiguration geladen werden,
die vom Benutzer erstellt wurde, falls diese denn lokal gefunden werden kann. Der Benutzer soll sämtliche
Einstellungen selbstverständlich zu jeder Zeit ändern können.
\subsection{Starten und Beenden}
\begin{itemize}
	\item F\_0010 Der Benutzer kann das System zu jedem Zeitpunkt starten.
	\item F\_0011 Der Benutzer kann das System zu jedem Zeitpunkt beenden.
	\item F\_0012 Beim ersten Start wird ein Standart-System-Zustand geladen.
	\item F\_0013 Beim Beenden wird der aktuelle System-Zustand gespeichert.
	\item F\_0014 Bei jedem weiteren Start wird der letzte System-Zustand geladen.
\end{itemize}
\subsection{Abspielen von Musik (Buttons)}
\begin{itemize}
	\item F\_0020 Der Benutzer kann Musik abspielen (Play)
	\item F\_0021 Der Benutzer kann Musik stoppen (Stop)
	\item F\_0022 Der Benutzer kann Musik pausieren (Pause)
	\item F\_0023 Der Benutzer kann Musik vor und zurück schalten (Skip)
	\item F\_0024 Der Benutzer kann Musik vor und zurück spuhlen (Seek)
	\item F\_0025 Der Benutzer kann Musik zufällig abspielen (random)
	\item F\_0026 Der Benutzer kann Musik wiederholen (repeat)
	\item F\_0027 Der Benutzer kann Musik im Consume-Mode abspielen
	\item F\_0028 Der Benutzer kann Musik im Single-Mode abspielen
\end{itemize}
\subsection{Abspielen von Musik (Short-Cuts)}
\begin{itemize}
	\item F\_0030 Play 	(ctrl + G)
        \item F\_0031 Stop 	(ctrl + S)
        \item F\_0032 Previous 	(ctrl + P)
        \item F\_0033 Next	(ctrl + N)
	\item F\_0034 Random	(ctrl + Z)
	\item F\_0035 Single	(ctrl + Y)
	\item F\_0036 Repeate	(ctrl + R)
	\item F\_0037 Consume	(ctrl + T)
\end{itemize}
\subsection{Queue (Warteschlange)}
\begin{itemize}
	\item F\_0040 Der Benutzer kann einzelne Lieder aus der Queue entfernen
	\item F\_0041 Der Benutzer kann alle Lieder aus der Queue entfernen
	\item F\_0042 Der Benutzer kann kann die Queue als Playlist speichern
	\item F\_0043 Der Benutzer kann Interpret, Album und Titel beliebig anordnen
\end{itemize}
\subsection{Playlist}
\begin{itemize}
	\item F\_0050 Der Benutzer kann eine neue Playliste erstellen
	\item F\_0051 Der Benutzer kann eine vorhandene Playlist ersetzen
	\item F\_0052 Der Benutzer kann eine Playlist löschen
\end{itemize}
\subsection{Dateibrowser}
\begin{itemize}
	\item F\_0060 Der Benutzer kann durch sein Dateisystem navigieren
	\item F\_0061 Der Benutzer kann einzelne Dateien zur Queue hinzufügen
	\item F\_0062 Der Benutzer kann mehrere Dateien zur Queue hinzufügen
	\item F\_0063 Der Benutzer kann Dateien ersetzen
	\item F\_0064 Der Benutzer kann Die Anzeige aktualisieren
	\item F\_0065 Der Benutzer kann die Anzeige neu einlesen
\end{itemize}
\subsection{Statistik}
\begin{itemize}
	\item F\_0070 Der Benutzer kann eine gesamt Statistik einsehen
	\begin{itemize}
		\item Anzahl der Interpreten
		\item Anzahl der Alben
		\item Anzahl der Lieder
		\item Musiklänge der Datenbank
		\item Abspielzeit	
		\item Zeit Online bzw. mit MPD verbunden
		\item Letzter Datenbank-Update
	\end{itemize}
\end{itemize}
\subsection{Einstellungen}
\begin{itemize}
	\item F\_0080 Der Benutzer kann Netzwerk-Einstellungen vornehmen
	\begin{itemize}
		\item Server IP / Port
		\item Reconnect Timout in Sekunden
		\item Avahi-Browser (Serverauswahl)
		\item Autoconnect
	\end{itemize}
	\item F\_0081 Der Benutzer kann Playback-Einstellungen vornehmen	
	\begin{itemize}
		\item Crossface in Sekunden
		\item Musik beim verlassen stoppen
	\end{itemize}
	\item F\_0082 Der Benutzer kann Allgemein-Einstellungen vornehmen
	\begin{itemize}
		\item Notifications(libnotify) nutzen
		\item Tray-Icon anzeigen
	\end{itemize}	
	\item F\_0083 Der Benutzer kann die Standarteinstellungen wiederherstellen
\end{itemize}
\subsection{Lautstärke}
\begin{itemize}
	\item F\_0090 Der Benutzer kann die Lautstärke regeln
\end{itemize}
\subsection{Suchen}
Eine einfache Textsuche zum finden von Titeln, Alben oder Interpreten innerhalb der 
Abspiellisten wurde implementiert. Dabei springt die Markierung des Textes beim 
eingeben von Zeichen in die Suche zu der ersten übereinstimmenden Stelle in der 
Plaxlist des Clients. Erst beim bestätigen der Eingabe im Suchfeld wird die Auswahl 
gefilter.
\begin{itemize}
	\item F\_0110 Der Benutzer kann seine Queue durchsuchen.
	\item F\_0111 Der Benutzer kann sein Dateisystem durchsuchen.
\end{itemize}
\subsection{Sonstiges}
\begin{itemize}
	\item F\_0120 Verbinden	(ctrl + C)
	\item F\_0121 Trennen	(ctrl + D)
	\item F\_0122 Beenden	(ctrl + Q)
\end{itemize}
\subsection{Administrator-Funktionen}
Durch das Unix-artige System wird der Administrator-Zugriff geregelt. Sobald sich der Benutzer im Unix System
als Administrator befindet, kann er auch den MPD-Client administrieren. Ein zusätzlicher Administrator-Modus wurde also
nicht implementiert.
\section{Produktdaten}
\subsection{Anzeige}
\subsubsection{Titelleiste}
\begin{itemize}
	\item D\_0010 Titel
	\item D\_0011 Interpret
	\item D\_0012 Album
	\item D\_0013 Liedposition
	\item D\_0014 Lautstärke
\end{itemize}
\subsubsection{Queue}
\begin{itemize}
	\item D\_0020 Titel
	\item D\_0021 Interpret
	\item D\_0022 Album
\end{itemize}
\subsubsection{Playlist}
\begin{itemize}
	\item D\_0030 Name der Playlist
	\item D\_0031 Zuletzt geändert
\end{itemize}
\subsubsection{Statistik}
\begin{itemize}
        \item D\_0040 Anzahl der Interpreten
        \item D\_0041 Anzahl der Alben
        \item D\_0042 Anzahl der Lieder
        \item D\_0043 Musiklänge der Datenbank
        \item D\_0044 Abspielzeit
        \item D\_0045 Zeit Online bzw. mit MPD verbunden
	\item D\_0046 Letzter Datenbank-Update
\end{itemize}
\subsubsection{Fußleiste}
\begin{itemize}
	\item D\_0050 Qualität in Mhz
	\item D\_0051 Qualität in bit
	\item D\_0052 Qualität in kbit
	\item D\_0053 Outputart (Stereo, Sourround,...)
	\item D\_0054 Zeit aktuell von insgesamt
	\item D\_0055 Anzahl an Liedern
	\item D\_0056 Komplette Abspielzeit
	\item D\_0057 Lautstärke
\end{itemize}
\subsubsection{Sonstiges}
\begin{itemize}
	\item D\_0060 Nächster Song (Seitenleiste)
\end{itemize}
\subsection{Persönliches Profil}
Da die Software auf Unix-artige Systeme beschränkt ist, wurde keine Profil-Verwaltung implementiert. Die
verschiedenen Profile werden durch die verschiedenen Profile des gesamten Betriebssystems definiert und differenziert.
\subsection{Persönliche Datenbank}
Eine persönliche Datenbank ist lokal nicht vorhanden. Die Datenbank des Benutzers befindet sich auf dem MPD-Server.
Einzig und alleine modulare Erweiterungen des MPD-Clients können lokale Datenbank-Implementierungen erfordern.
\subsection{Persönliche Einstellungen}
Client Einstellungen werden lokal gespeichert.
\begin{itemize}
	\item config.xml
	\item ....
\end{itemize}
\section{Qualitätsanforderungen}
Die Software soll natürlich von hoher Qualität sein. Hierfür sollen folgende
Anforderungen erfüllt werden:
\subsection{Q\_0001 Korrektheit}
Die Software muss fehlerfrei und korrekt sein. Es wurden Testszenarien und Testfälle erstellt,
um Fehler zu finden und auszubessern. Aber auch wenn nach Veröffentlichung der Software ein 
Fehler gefunden werden sollte, wird dieser sofort ausgebessert. Bei schwerwiegenden Fehlern
werden die Nutzer direkt auf den Fehler aufmerksam gemacht.
\subsection{Q\_0002 Wartbarkeit}
Der Wartungsaufwandt der Software ist gering bis garnicht vorhanden. Ändert sich die Umgebungssoftware
(z.B. der MPD-Server) dann sind die Änderungen so geringfügig bzw. trivial, dass sie den MPD-Client 
nicht beeinflussen werden. Fehler der Software (sollten Fehler auftreten) wären leicht analysier- bzw.
prüfbar und natürlich auch leicht zu beheben. Zur Warbarkeit gehört ebenso die Modularität, d. h.
die Software ist technisch so realisiert, dass sie leicht erweitert werden kann, Stichwort Model View
Controller (MVC). Kritische Stellen werden von Fehlerbehandlungroutinen abgearbeitet.
Alle diese Routinen schreiben Meldungen in eine Log-Datei.
\subsection{Q\_0003 Zuverlässigkeit}
Das System funktioniert und reagiert tollerant auf fehlerhafte Eingaben bzw. fehlerhafte Benutzung.
Das Programm funktioniert sieben Tage die Woche und 24h am Tag und muss nicht abgeschaltet werden.
\subsection{Q\_0004 Effizienz}
Der MPD-Client ist technisch effizient. Das Programm ist schnell geladen und Eingaben des Benutzers
werden praktisch sofort ausgeführt. Es gibt so gut wie keine Wartezeiten, jedenfalls sind diese 
so genannten Reaktionszeiten für den Benutzer nicht merkbar. Selbst bei sehr großen Musik-Datenbanken
und Playlists benötigt das Programm kaum Rechenzeit und sonstige Hardware-resourcen.
\subsection{Q\_0005 Benutzbarkeit}
Die Software ist leicht verständlich und intuitiv bedienbar. Nötige Kenntnisse zur Nutzung des 
MPD-Clients sind leicht zu erlernen.
\subsection{Q\_0006 Design}
Das Design soll ansprechend und modern sein, allerdings wenn es Konflikte zwischen technischer Umsetzung 
und Design oder Effizienz und Design geben sollte, ist stets im Interesse der technischen Umsetzung bzw. 
der Effizienz zu entscheiden.
\section{Globale Testszenarien und Testfälle}
\subsection{Cxxtest}
CxxTest ist mit JUnit, CPPUnit oder xUnit zu vergleichen und somit ein leichtgewichtiges Framework für C++.
Der Vorteil gegenüber ähnlichen bzw. anderen Testmöglichkeiten sind die folgenden:
\begin{itemize}
	\item Es wird kein RTTI benötigt (Run-Time Type Information)
	\item Benötigt keine Member Template Funktionen
	\item Benötigt keine Exception-Behandlung
	\item Benötigt keine externen Bibliotheken (Memory Managment, File/Console I/O, Grafische Bibliotheken)
	\item Wird allein durch Header-Dateien (und ein Python-Skript) realisiert.
	\item Benötigt keine manuelle Regestrierung von Tests und Test-Suits
\end{itemize}
All diese Punkte machen CxxTest extrem portabel und nutzbar. Der Aufwandt zur Erstellung von Tests
wird minimiert.
\subsubsection{Testfälle}
\subsection{Testprotokoll}
Um Fehler aufzuspühren, die die grafische Oberfläche betreffen, wurde ein Testprotokoll erstellt in dem zunächst
alle möglichen Funktionen der grafischen Oberfläche aufgelistet werden. Außerdem müssen diese Funktionen mit 
anderen Funktionen kombiniert und mehrfach ausgeführt werden. Zu jedem dieser Fälle ist ein zu erwartendes Ergebnis
festzulegen und anschließend zu überprüfen ob das erwartete Ergebnis eingetroffen ist. Das eingetroffene Ergebnis
ist ebenfalls zu protokollieren. Es wurden jeweils die Buttons, sowie die Shortcuts geprüft.
\subsubsection{Abspielfunktionen}
\textbf{Einfache Ausführung:}\ \\ \\
\begin{tabular}[c]{|l|p{6cm}|c|}
\hline
\textbf{Testfall} & \textbf{Erwartetes Ergebnis} & \textbf{Ergebnis eingetroffen?}\\
\hline
Play & Musik spielt ab.\newline Play wird zu Pause. & Ja\\
\hline
Pause & Musik pausiert.\newline Pause wird zu Play. & Ja\\
\hline
Next & Nächstes Lied abspielen & Ja\\
\hline
Previous & Vorheriges Lied abspielen & Ja\\
\hline
Stop & Beende abspielen\newline Pause wird zu Play. & Ja\\
\hline
Skipping & An Liedposition springen & Ja\\
\hline
Random & Musik der Queue zufällig abspielen & Ja\\
\hline 
Repeat & Ein Lied wiederholen & Ja\\
\hline
Repeat all & Queue wiederholen & Ja\\
\hline
Consume Mode & Ein abgespieltes Lied entfernen & Ja\\
\hline
Single Mode & Ein Lied abspielen, dann Stoppen & Ja\\
\hline
\end{tabular}
\newpage
\textbf{Kombinierte Ausführung:}\ \\ \\
\begin{tabular}[c]{|p{6cm}|p{6cm}|c|}
\hline
\textbf{Testfall} & \textbf{Erwartetes Ergebnis} & \textbf{Ergebnis eingetroffen?}\\
\hline
Play, Stop, Play(1) & Musik spielt ab.\newline Musik stoppt.\newline Musik spielt ab. & Ja\\
\hline
Play, Pause, Play(2) & Musik spiel ab.\newline Musik pausiert.\newline Musik spiel ab. & Ja\\
\hline
Next, Next(3) & Skip weiter.\newline Skip weiter. & Ja\\
\hline
Previous, Previous(4) & Skip zurück.\newline Skip zurück & Ja\\
\hline
Next, Previous(5) & Skip weiter.\newline Skip zurück & Ja\\
\hline
Previous, Next(6) & Skip zurück.\newline Skip weiter & Ja\\
\hline
Random, Repeat all & Musik der Queue zufällig abspielen\newline Queue wiederholen & Ja\\
\hline
Random, Consume Mode & Musik der queue zufällig abspielen\newline Ein abgespieltes Lied entfernen & Ja\\
\hline
Random, Single Mode & Musik der Queue zufällig abspielen\newline Ein Lied abspielen, dann stoppen & Ja\\
\hline
Consume Mode, Single Mode & Ein abgespieltes Lied entfernen\newline Ein Lied abspielen, dann Stoppen & Ja\\
\hline
Consume Mode, Repeat all & Kann nur einmal durchlaufen & Ja\\
\hline
Random, 1 & Musik der Queue zufällig abspielen\newline 1 & Ja\\
\hline
Random, 2 & Musik der Queue zufällig abspielen\newline 2 & Ja\\
\hline
Random, 3 & Musik der Queue zufällig abspielen\newline 3 & Ja\\
\hline
Random, 4 & Musik der Queue zufällig abspielen\newline 4 & Ja\\
\hline
Random, 5 & Musik der Queue zufällig abspielen\newline 5 & Ja\\
\hline
Random, 6 & Musik der Queue zufällig abspielen\newline 6 & Ja\\
\hline
Repeat all, 1 & Queue wiederholen\newline 1 & Ja\\
\hline
Repeat all, 2 & Queue wiederholen\newline 2 & Ja\\
\hline
Repeat all, 3 & Queue wiederholen\newline 3 & Ja\\
\hline
Repeat all, 4 & Queue wiederholen\newline 4 & Ja\\
\hline
\end{tabular}
\begin{tabular}[c]{|p{6cm}|p{6cm}|c|}
\hline
\textbf{Testfall} & \textbf{Erwartetes Ergebnis} & \textbf{Ergebnis eingetroffen?}\\
\hline
Repeat all, 5 & Queue wiederholen\newline 5 & Ja\\
\hline
Repeat all, 6 & Queue wiederholen\newline 6 & Ja\\
\hline
Consume Mode, 1 & Ein abgespieltes Lied entfernen\newline 1 & Ja\\
\hline
Consume Mode, 2 & Ein abgespieltes Lied entfernen\newline 2 & Ja\\
\hline
Consume Mode, 3 & Ein abgespieltes Lied entfernen\newline 3 & Ja\\
\hline
Consume Mode, 4 & Ein abgespieltes Lied entfernen\newline 4 & Ja\\
\hline
Consume Mode, 5 & Ein abgespieltes Lied entfernen\newline 5 & Ja\\
\hline
Consume Mode, 6 & Ein abgespieltes Lied entfernen\newline 6 & Ja\\
\hline
Single Mode, 1 & Ein Lied abspielen, dann Stoppen\newline 1 & Ja\\
\hline
Single Mode, 2 & Ein Lied abspielen, dann Stoppen\newline 2 & Ja\\
\hline
Single Mode, 3 & Ein Lied abspielen, dann Stoppen\newline 3 & Ja\\
\hline
Single Mode, 4 & Ein Lied abspielen, dann Stoppen\newline 4 & Ja\\
\hline
Single Mode, 5 & Ein Lied abspielen, dann Stoppen\newline 5 & Ja\\
\hline
Single Mode, 6 & Ein Lied abspielen, dann Stoppen\newline 6 & Ja\\
\hline
\end{tabular}
\newpage
Im folgenden wird auf das Protokoll der konbinierten Ausführung referenziert.\ \\ \\
\textbf{Mehrfache Ausführung:}\ \\ \\
\begin{tabular}[c]{|p{6cm}|p{6cm}|c|}
\hline
\textbf{Testfall} & \textbf{Erwartetes Ergebnis} & \textbf{Ergebnis eingetroffen?}\\
\hline
Fall 1 x 10 & Fall 1 x 10 & Ja\\
\hline
Fall 2 x 10 & Fall 2 x 10 & Ja\\
\hline
Fall 3 x 10 & Fall 3 x 10 & Ja\\
\hline
Fall 4 x 10 & Fall 4 x 10 & Ja\\
\hline
Fall 5 x 10 & Fall 5 x 10 & Ja\\
\hline
Fall 6 x 10 & Fall 6 x 10 & Ja\\
\hline
Fall 12 x 10 & Fall 12 x 10 & Ja\\
\hline
Fall 13 x 10 & Fall 13 x 10 & Ja\\
\hline
Fall 14 x 10 & Fall 14 x 10 & Ja\\
\hline
Fall 15 x 10 & Fall 15 x 10 & Ja\\
\hline
Fall 16 x 10 & Fall 16 x 10 & Ja\\
\hline
Fall 17 x 10 & Fall 17 x 10 & Ja\\
\hline
Fall 18 x 10 & Fall 18 x 10 & Ja\\
\hline
Fall 19 x 10 & Fall 19 x 10 & Ja\\
\hline
Fall 20 x 10 & Fall 20 x 10 & Ja\\
\hline
Fall 21 x 10 & Fall 21 x 10 & Ja\\
\hline
Fall 22 x 10 & Fall 22 x 10 & Ja\\
\hline
Fall 23 x 10 & Fall 23 x 10 & Ja\\
\hline
Fall 24 x 10 & Fall 24 x 10 & Ja\\
\hline
Fall 25 x 10 & Fall 25 x 10 & Ja\\
\hline
Fall 26 x 10 & Fall 26 x 10 & Ja\\
\hline
Fall 27 x 10 & Fall 27 x 10 & Ja\\
\hline
Fall 28 x 10 & Fall 28 x 10 & Ja\\
\hline 
Fall 29 x 10 & Fall 29 x 10 & Ja\\
\hline
Fall 30 x 10 & Fall 30 x 10 & Ja\\
\hline
Fall 31 x 10 & Fall 31 x 10 & Ja\\
\hline
Fall 32 x 10 & Fall 32 x 10 & Ja\\
\hline
Fall 33 x 10 & Fall 33 x 10 & Ja\\
\hline
Fall 34 x 10 & Fall 34 x 10 & Ja\\
\hline
Fall 35 x 10 & Fall 35 x 10 & Ja\\
\hline
\end{tabular}
\subsubsection{Queue-Funktionen}
\textbf{Einfache Ausführung}\ \\ \\
\begin{tabular}[c]{|p{6cm}|p{6cm}|c|}
\hline
\textbf{Testfall} & \textbf{Erwartetes Ergebnis} & \textbf{Ergebnis eingetroffen?}\\
\hline
Remove & Ein Lied aus Queue entfernen & Ja\\
\hline
Clear & Alle Lieder aus Queue entfernen & Ja\\
\hline
Save as Playlist & Queue als Playlist speichern & Ja\\
\hline
Suchen & Nach eingegebenem Wort suchen & Ja\\
\hline
\end{tabular}
\newpage
\textbf{Kombinierte Ausführung}\ \\ \\
Kombinierte Ausführung der Funktionen der Queue machen nicht wirklich viel Sinn da z.B.
die Funktion Clear die Queue löscht. Auch Save as Playlist wird wohl kaum öfter als einmal
pro Queue angewandt. Die einzige Kombination die Sinn macht getestet zu werden ist die 
folgende:\ \\ \\
\begin{tabular}[c]{|p{6cm}|p{6cm}|c|}
\hline
\textbf{Testfall} & \textbf{Erwartetes Ergebnis} & \textbf{Ergebnis eingetroffen?}\\
\hline
Suchen, Remove & Nach eingegebenem Wort suchen\newline Ein Lied aus der Queue entfernen & Ja\\
\hline
\end{tabular}
\ \\ \\
\textbf{Mehrfache Ausführung}\ \\ \\
Die mehrfache Ausführung ist ähnlich unsinnig wie die der kombinierten Ausführung.
Mehrmals hintereinander die Queue löschen ist nicht möglich, genauso wie man wohl kaum 
mehrmals die gleiche Playlist erstellt. So bleibt wieder nur ein Testfall zu prüfen:\ \\ \\
\begin{tabular}[c]{|p{6cm}|p{6cm}|c|}
\hline
\textbf{Testfall} & \textbf{Erwartetes Ergebnis} & \textbf{Ergebnis eingetroffen?}\\
\hline
Suchen, Remove x 10 & Nach eingegebenem Wort suchen\newline Ein Lied aus der Queue entfernen x 10& Ja\\
\hline
\end{tabular}
\subsubsection{Playlist-Funktionen}
\textbf{Einfache Ausführung}\ \\ \\
\begin{tabular}[c]{|p{6cm}|p{6cm}|c|}
\hline
\textbf{Testfall} & \textbf{Erwartetes Ergebnis} & \textbf{Ergebnis eingetroffen?}\\
\hline
\end{tabular}
\textbf{Kombinierte Ausführung}\ \\ \\
\begin{tabular}[c]{|p{6cm}|p{6cm}|c|}
\hline
\textbf{Testfall} & \textbf{Erwartetes Ergebnis} & \textbf{Ergebnis eingetroffen?}\\
\hline
\end{tabular}
\textbf{Mehrfache Ausführung}\ \\ \\
\begin{tabular}[c]{|p{6cm}|p{6cm}|c|}
\hline
\textbf{Testfall} & \textbf{Erwartetes Ergebnis} & \textbf{Ergebnis eingetroffen?}\\
\hline
\end{tabular}
\subsubsection{Dateibrowser-Funktionen}
\textbf{Einfache Ausführung}\ \\ \\
\begin{tabular}[c]{|p{6cm}|p{6cm}|c|}
\hline
\textbf{Testfall} & \textbf{Erwartetes Ergebnis} & \textbf{Ergebnis eingetroffen?}\\
\hline
\end{tabular}
\textbf{Kombinierte Ausführung}\ \\ \\
\begin{tabular}[c]{|p{6cm}|p{6cm}|c|}
\hline
\textbf{Testfall} & \textbf{Erwartetes Ergebnis} & \textbf{Ergebnis eingetroffen?}\\
\hline
\end{tabular}
\textbf{Mehrfache Ausführung}\ \\ \\\begin{tabular}[c]{|p{6cm}|p{6cm}|c|}
\hline
\textbf{Testfall} & \textbf{Erwartetes Ergebnis} & \textbf{Ergebnis eingetroffen?}\\
\hline
\end{tabular}
\subsubsection{Statistik}
Für die Statistik kann kein Testprotokoll angewandt werden.
\subsubsection{Einstellungen}
\textbf{Einfache Ausführung}\ \\ \\\begin{tabular}[c]{|p{6cm}|p{6cm}|c|}
\hline
\textbf{Testfall} & \textbf{Erwartetes Ergebnis} & \textbf{Ergebnis eingetroffen?}\\
\hline
\end{tabular}
\textbf{Kombinierte Ausführung}\ \\ \\
\begin{tabular}[c]{|p{6cm}|p{6cm}|c|}
\hline
\textbf{Testfall} & \textbf{Erwartetes Ergebnis} & \textbf{Ergebnis eingetroffen?}\\
\hline
\end{tabular}
\textbf{Mehrfache Ausführung}\ \\ \\
\begin{tabular}[c]{|p{6cm}|p{6cm}|c|}
\hline
\textbf{Testfall} & \textbf{Erwartetes Ergebnis} & \textbf{Ergebnis eingetroffen?}\\
\hline
\end{tabular}
\subsubsection{Lautstärke}
\textbf{Einfache Ausführung}\ \\ \\
\begin{tabular}[c]{|p{6cm}|p{6cm}|c|}
\hline
\textbf{Testfall} & \textbf{Erwartetes Ergebnis} & \textbf{Ergebnis eingetroffen?}\\
\hline
\end{tabular}
\textbf{Kombinierte Ausführung}\ \\ \\
\begin{tabular}[c]{|p{6cm}|p{6cm}|c|}
\hline
\textbf{Testfall} & \textbf{Erwartetes Ergebnis} & \textbf{Ergebnis eingetroffen?}\\
\hline
\end{tabular}
\textbf{Mehrfache Ausführung}\ \\ \\
\begin{tabular}[c]{|p{6cm}|p{6cm}|c|}
\hline
\textbf{Testfall} & \textbf{Erwartetes Ergebnis} & \textbf{Ergebnis eingetroffen?}\\
\hline
\end{tabular}
\subsubsection{Suchen}
\textbf{Einfache Ausführung}\ \\ \\
\begin{tabular}[c]{|p{6cm}|p{6cm}|c|}
\hline
\textbf{Testfall} & \textbf{Erwartetes Ergebnis} & \textbf{Ergebnis eingetroffen?}\\
\hline
\end{tabular}
\textbf{Kombinierte Ausführung}\ \\ \\
\begin{tabular}[c]{|p{6cm}|p{6cm}|c|}
\hline
\textbf{Testfall} & \textbf{Erwartetes Ergebnis} & \textbf{Ergebnis eingetroffen?}\\
\hline
\end{tabular}
\textbf{Mehrfache Ausführung}\ \\ \\
\begin{tabular}[c]{|p{6cm}|p{6cm}|c|}
\hline
\textbf{Testfall} & \textbf{Erwartetes Ergebnis} & \textbf{Ergebnis eingetroffen?}\\
\hline
\end{tabular}
\subsubsection{Sonstiges}
\textbf{Einfache Ausführung}\ \\ \\
\begin{tabular}[c]{|p{6cm}|p{6cm}|c|}
\hline
\textbf{Testfall} & \textbf{Erwartetes Ergebnis} & \textbf{Ergebnis eingetroffen?}\\
\hline
\end{tabular}
\textbf{Kombinierte Ausführung}\ \\ \\
\begin{tabular}[c]{|p{6cm}|p{6cm}|c|}
\hline
\textbf{Testfall} & \textbf{Erwartetes Ergebnis} & \textbf{Ergebnis eingetroffen?}\\
\hline
\end{tabular}
\textbf{Mehrfache Ausführung}\ \\ \\
\begin{tabular}[c]{|p{6cm}|p{6cm}|c|}
\hline
\textbf{Testfall} & \textbf{Erwartetes Ergebnis} & \textbf{Ergebnis eingetroffen?}\\
\hline
\end{tabular}
\section{Entwicklungsumgebung}
\subsection{Software}
\begin{itemize}
	\item unix System
	\item MPD-Server	
	\item Avahi-Browser
\end{itemize}
\subsection{Hardware}
Keine Anforderungen spezifiziert
\subsection{Orgware}
\begin{itemize}
	\item git (Versionsverwaltungssoftware)
	\item cmake (Compiler)
	\item doxygen (Dokumentation)
	\item Editor nach Wahl
	\item Glade
\end{itemize}
\section{Glossar}
