\chapter{Pflichtenheft}
\section{Zielbestimmungen}
\subsection{Muss-Kriterien}
\renewcommand{\labelitemi}{•}
\begin{itemize}
	\item Server-Verbindung
	\begin{itemize}
		\item Platzhalter
		\item Platzhalter
		\item Platzhalter
	\end{itemize}
	\item Client-Einstellungen
	\begin{itemize}
		\item Platzhalter
		\item Platzhalter
		\item Platzhalter
	\end{itemize}
	\item Musik-Steuerung
	\begin{itemize}
		\item Platzhalter
		\item Platzhalter
		\item Platzhalter
	\end{itemize}
	\item Sonstiges
	\begin{itemize}
		\item Platzhalter
		\item Platzhalter
		\item Platzhalter
	\end{itemize}
\end{itemize}
\subsection{Wunsch-Kriterien}
\begin{itemize}
		\item Platzhalter
		\item Platzhalter
		\item Platzhalter
\end{itemize}
\subsection{Abgrenzungskriterien}
\begin{itemize}
		\item Platzhalter
		\item Platzhalter
		\item Platzhalter
\end{itemize}
\section{Produkteinsatz}
Welche Anwendungsbereiche (Zweck), Zielgruppen (Wer mit welchen Qualifikationen), Betriebsbedingungen (Betriebszeit,
Aufsicht)?\ \\ \\
Der MPD-Client ist nicht auf bestimmte Gewerbe beschränkt, ein jeder soll diesen Client
verwenden können. Grundlage für die Verwendung der Software ist die General Public License (GPL)
Version 3 vom 29 Juni 2007.\ \\ \\
Definition der GPL:
\begin{center}
http://www.gnu.org/licenses/gpl.html
\end{center}
\subsection{Anwendungsbereiche}
Einzelpersonen verwenden dieses System, um überall da wo mit 
einem Unix-artigen Betriebssystem Musik abgespielt werden soll.
\subsection{Zielgruppen}
Personengruppen die komfortabel von überall aus auf ihre Musik und Playlist zugreifen
wollen ohne diese jedes mal aufwändig synchronisieren zu müssen (z.B. durch Abgleich von Datenträgern).\ \\ \\
Es werden Basiskenntnisse zum Aufbau einer Netzwerkverbindung und zur Nutzung des Internets vorausgesetzt.
Aufgrund der für das System vorgesehenen Betriebsumgebung sind ebenso Kenntnisse im Umgang mit Unix nötig.\ \\ \\
Der Benutzer muss die Systemsprache Englisch beherrschen.
\subsection{Betriebsbedingungen}
Das System soll sich bezüglich der Betriebsbedingungen nicht sonderlich von vergleichbaren Systemen bzw.
Anwendungen unterscheiden und dementsprechend folgend Punkte erfüllen:
\begin{itemize}
	\item Betriebsdauer: Täglich, 24 Stunden
	\item Keinerlei Wartung soll nötig sein
	\item Sicherungen der Konfiguration müssen vom Benutzer vorgenommen werden
\end{itemize}
\section{Produktumgebung}
\subsection{Software}
\begin{itemize}
	\item Avahi Daemon
	\item MPD-Client
\end{itemize}
Ein MPD-Server ist nicht unbedingt von nöten.
\subsection{Hardware}
Minimale Hardwareanforderungen:
Empfohlene Hardwareanforderungen:
\subsection{Orgware}
\begin{itemize}
	\item git (Versionsverwaltungssoftware)
	\item cmake (Compiler)
	\item doxygen (Dokumentation)
	\item Editor nach Wahl
	\item Glade
\end{itemize}
\section{Produktfunktionen}
Funktionen des MPD-Clients.\ \\ \\
Beim ersten Start des Systems soll eine Standard-Konfiguration geladen werden und die Verbindungseinstellungen
zu einem MPD-Server müssen vorgenommen werden. Bei jedem weiteren Start soll die Konfiguration geladen werden,
die vom Benutzer erstellt wurde, falls diese denn lokal gefunden werden kann. Der Benutzer soll sämtliche
Einstellungen selbstverständlich zu jeder Zeit ändern können.
\subsection{Allgemeine Funktionen}
\subsubsection{Starten und Beenden}
\begin{itemize}
	\item F\_0010 Der Benutzer kann das System zu jedem Zeitpunkt starten.
	\item F\_0020 Der Benutzer kann das System zu jedem Zeitpunkt beenden.
	\item F\_0030 Beim ersten Start wird ein Standart-System-Zustand geladen.
	\item F\_0040 Beim Beenden wird der aktuelle System-Zustand gespeichert.
	\item F\_0050 Bei jedem weiteren Start wird der letzte System-Zustand geladen.
\end{itemize}
\subsection{Benutzerfunktionen}
\subsubsection{Benutzer-Kennung}
Eine Benutzerkennung ist nicht erforderlich und wurde deshalb auch nicht implementiert.
\subsubsection{Persönliche Daten}
Verbindungseinstellungen müssen vorgenommen werden und können zu jedem Zeitpunkt geändert werden.
\begin{itemize}
	\item F\_0110 Der Benutzer kann Verbindungseinstellungen vornehmen und sie ändern
\end{itemize}
\subsubsection{Persönliche Konfiguration}
config.xml
log datei
\subsubsection{Persönliches Profil}
Da die Software auf Unix-artige Systeme beschränkt ist, wurde keine Profil-Verwaltung implementiert. Die
verschiedenen Profile werden durch die verschiedenen Profile des gesamten Betriebssystems definiert und differenziert.
\subsubsection{Persönliche Datenbank}
Eine persönliche Datenbank ist lokal nicht vorhanden. Die Datenbank des Benutzers befindet sich auf dem MPD-Server.
Einzig und alleine modulare Erweiterungen des MPD-Clients können lokale Datenbank-Implementierungen erfordern.
\subsubsection{Kommunikation (Chat)}
Kommunikation von MPD-Client zu MPD-Client kann theoretisch implementiert werden, eine solche Schnittstelle ist vorhanden.
Allerdings wurde hierauf verzichtet, da im Vordergrund das Abspielen und Verwalten von Musik steht und es deutlich
einfachere und bessere Systeme gibt, mit Hilfe derer man kommunizieren kann.
\subsubsection{Suchen}
Eine einfache Textsuche zum finden von Titeln, Alben oder Interpreten innerhalb der Abspiellisten wurde implementiert.
Dabei springt die Markierung des Textes beim eingeben von Zeichen in die Suche zu der ersten übereinstimmenden
Stelle in der Playlist des Clients. Erst beim bestätigen der Eingabe im Suchfeld wird die Auswahl gefiltert.
\begin{itemize}
        \item F\_0210 Der Benutzer kann seine Queue durchsuchen.
\end{itemize}
\subsection{Abspielfunktionen}
\begin{itemize}
	\item Play
	\item Skip
	\item Stop
	\item Pause
	\item Shuffle
	\item Loop
\end{itemize}
\subsubsection{Initialisierung}
Platzhalter
\subsubsection{Verlauf}
\subsection{Administrator-Funktionen}
Durch das Unix-artige System wird auch der Administrator-Zugriff geregelt. Sobald sich der Benutzer im Unix System
als Administrator befindet, kann er auch den MPD-Client administrieren. Ein zusätzlicher Administrator-Modus wurde also
nicht implementiert.
\section{Produktdaten}
\section{Produktleistungen}
\section{Benutzeroberfläche}
\subsection{Bildschirmlayout}
\subsubsection{Startbildschirm}
\subsubsection{Einstellungsfenster}
\subsubsection{Verbindungsfenster}
\subsubsection{Benutzermenü}
\section{Produktmodellierung}
\section{Qualitätsanforderungen}
\section{Globale Testszenarien und Testfälle}
\section{Entwicklungsumgebung}
\subsection{Software}
\subsection{Hardware}
\subsection{Orgware}
\section{Ergänzungen}
\section{Glossar}
