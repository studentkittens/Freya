\chapter{Richtlinien}
\section{Programmierrichtlinien}
\renewcommand{\labelitemi}{•}
\begin{itemize}
\item Allman-Stil.
\item Tabstop = 4 Leerzeichen.
\item Keine globalen Variablen.
\item Sinnvolle Variablenbenennung, "'lowercase"'.
\item Klassenmethoden nur in Ausnahmefällen bzw. nur mit guten Gründen.
\item Valgrind darf keine Laufzeitfehler bringen.
\item "'camelcase"' bei Objektnamen, C-Style bei Funktionsnamen - Präzise Namen.
\item Modulare Gestaltung.
\item Code-Sauberkeit ist wichtiger als Code Performance.
\item "'make"' sollte keine Warnungen ausgeben, die man leicht umgehen könnte.
\item "'make test"' soll vollständig durchlaufen.
\end{itemize}
\subsection{Begründung}
Diese Programmierrichtlinen sorgen für ein einfaches, übersichtliches und einheitliches Arbeiten.
Jeder kann sich ohne größere Umstände in den Code eines anderen einlesen. Dies gewährleistet eine
hohe Wartbarkeit der Programm-Codes und beugt außerdem Fehlern vor. Das Programm ist leicht
erweiterbar ohne große Anpassungen vornehmen zu müssen.
\section{Toolauswahl}
\begin{itemize}
\item Avahi-Daemon
\item git
\item Glade
\item doxygen
\end{itemize}
\subsection{Begründung}
Avahi-Daemon ist ein Dienst, durch den man schnell und einfach MPD-Server im Netz finden und eine
Verbindung zu den Servern aufbauen kann. Git dient zur Versionsverwaltung. Glade bietet eine perfekte
Trennung von der grafischen Oberfläche zum Kontrollkern des Programms außerdem kann mit Glade sehr
einfach eine grafische Oberfläche erstellt werden. Doxygen !!!!!!!Literate-Programming (KNUT)!!!!
\section{Bibliotheken}
\begin{itemize}
\item C++
\item gtkmm3
\item libmpd
\item libxml2
\item libnotify
\item Avahi-glib
\end{itemize}
\subsection{Begründung}
C++ wurde gewählt um die Fähigkeiten der Authoren zu erweitern. Außerdem gibt es für Java nur wenige
oder sehr schlechte Bibliotheken für dieses Projekt. Gtkmm3 bietet ein dynamisches Layout und ist
leichtgewichtiger als qt. Swing ist unter C++ nicht nutzbar. Libmpd liefert libmpdclient mit.
Libxml2 liefert eine standardisierte config nach Xml-Standards, ist sehr leichtgewichtig und überall
installiert. Libnotify liefert Benachrichtigungen über interne Events und ist auf den meisten Linux
Distributionen verbreitet. Avahi-glib dient als Server-Browser.\ \\ \\
Als primäre Entwicklerpalttform wurde Linux gewählt.
