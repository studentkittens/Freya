\chapter{Richtlinien}
\section{Programmierrichtlinien}
\renewcommand{\labelitemi}{•}
\begin{itemize}
\item Allman-Stil.
\item Tabstop = 4 Leerzeichen.
\item Keine globalen Variablen.
\item Sinnvolle Variablenbenennung, "'lowercase"'.
\item Klassenmethoden nur in Ausnahmefällen bzw. nur mit guten Gründen.
\item Valgrind darf keine Laufzeitfehler bringen, die nicht von Gtk oder anderen Bibliotheken stammen.
\item "'camelcase"' bei Objektnamen, C-Style bei Funktionsnamen - Präzise Namen.
\item Modulare Gestaltung.
\item Model-View-Controller Pattern
\item Code-Sauberkeit ist wichtiger als Code Performance.
\item "'make"' sollte keine Warnungen ausgeben, die man leicht umgehen könnte.
\item "'make test"' soll vollständig durchlaufen.
\end{itemize}
\subsection{Begründung}
Diese Programmierrichtlinen sorgen für ein einfaches, übersichtliches und einheitliches Arbeiten.
Jeder kann sich ohne größere Umstände in den Code eines anderen einlesen. Dies gewährleistet eine
hohe Wartbarkeit der Programm-Codes und beugt außerdem Fehlern vor. Das Programm ist leicht
erweiterbar ohne große Anpassungen vornehmen zu müssen.
\section{Toolauswahl}
\begin{itemize}
\item git
\item Glade
\item doxygen
\end{itemize}
\subsection{Begründung}
Git dient zur Versionsverwaltung. Glade bietet eine perfekte Trennung von der grafischen 
Oberfläche zum Kontrollkern des Programms außerdem kann mit Glade sehr
einfach eine grafische Oberfläche erstellt werden. Doxygen (Literate-Programming (KNUT))
\section{Bibliotheken}
\begin{itemize}
\item C++
\item gtkmm3 \footnote{http://www.gtkmm.org/de/index.html}
\item libmpdclient \footnote{http://www.musicpd.org/doc/libmpdclient/files.html}
\item libxml2 \footnote{http://xmlsoft.org/index.html}
\item libnotify \footnote{http://developer.gnome.org/libnotify/}
\item Avahi-glib \footnote{http://avahi.org/wiki/WikiStart\#WhatisAvahi}
\end{itemize}
\subsection{Begründung}
C++ wurde gewählt um die Fähigkeiten der Authoren zu erweitern. Außerdem gibt es für Java nur wenige
oder sehr alte Bibliotheken für dieses Projekt. Gtkmm3 bietet ein dynamisches Layout und ist
leichtgewichtiger als Qt, außerdem ist es einfacher in der Handhabung und lässt sich auf den meisten 
Desktopumgebungen besser integrieren. Swing ist aus Sicht der Authoren nicht geeignet.
libmpdclient liefert eine eigene lowlevel C-Bibliothek.
Libxml2 liefert eine standardisierte config nach Xml-Standards, ist sehr leichtgewichtig und überall
installiert. Libnotify liefert Benachrichtigungen über interne Events und ist auf den meisten Linux
Distributionen verbreitet. Avahi-glib ist ein Interface für den Avahidaemon, der optional ist. Avahi 
dient als Server-Borwser, kann allerdings nur MPD Server finden, die mit einer zeroconf veröffentlicht 
wurden.\ \\ \\
Primäre Entwicklerplattform ist ein Unix-System nach Wahl.
