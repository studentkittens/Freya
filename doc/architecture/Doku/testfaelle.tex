\section{Globale Testszenarien und Testfälle}
\subsection{Cxxtest}
Als Testframework wurde CxxTest ausgewählt. Die Gründe für diese Entscheidung werden gut von der offiziellen zusammengefasst: 
\begin{quote}\footnote{http://cxxtest.tigris.org/}
    CxxTest is a JUnit/CppUnit/xUnit-like framework for C/C++.\ \\
    It is focussed on being a lightweight framework that is well suited for 
    integration into embedded systems development projects.\ \\
    CxxTest's advantages over existing alternatives are that it:
    \begin{itemize}
        \item Doesn't require RTTI
        \item Doesn't require member template functions
        \item Doesn't require exception handling
        \item Doesn't require any external libraries (including memory management, file/console I/O, graphics libraries)
        \item Is distributed entirely as a set of header files (and a python script).
        \item Doesn't require the user to manually register tests and test suites 
    \end{itemize}
    This makes it extremely portable and usable.\ \\ \\
\end{quote} 
\subsection{Testfälle}
Für Teile des Programmes die nicht vom Testprotokoll erfasst werden, und automatisch gestestet werden sollen Testfälle mit Cxxtest
geschrieben werden.   
\subsection{Testprotokoll}
Um Fehler aufzuspüren, die die grafische Oberfläche betreffen, wurde ein Testprotokoll erstellt in dem zunächst
alle möglichen Funktionen der grafischen Oberfläche aufgelistet werden. Außerdem müssen diese Funktionen mit 
anderen Funktionen kombiniert und mehrfach ausgeführt werden. Zu jedem dieser Fälle ist ein zu erwartendes Ergebnis
festzulegen und anschließend zu überprüfen ob das erwartete Ergebnis eingetroffen ist. Das eingetroffene Ergebnis
ist ebenfalls zu protokollieren. Es wurden jeweils die Buttons, sowie die Shortcuts geprüft.
\subsubsection{Abspielfunktionen}
\textbf{Einfache Ausführung:}\ \\ \\
\begin{tabularx}{\textwidth}{|X|X|l|}
    \hline
    \textbf{Testfall} & \textbf{Erwartetes Ergebnis} & \textbf{Ergebnis eingetroffen?}\\
    \hline
    Play & Musik spielt ab.\newline Play wird zu Pause. & Ja\\
    \hline
    Pause & Musik pausiert.\newline Pause wird zu Play. & Ja\\
    \hline
    Next & Nächstes Lied abspielen & Ja\\
    \hline
    Previous & Vorheriges Lied abspielen & Ja\\
    \hline
    Stop & Beende abspielen\newline Pause wird zu Play. & Ja\\
    \hline
    Skipping & An Liedposition springen & Ja\\
    \hline
    Random & Musik der Queue zufällig abspielen & Ja\\
    \hline 
    Repeat & Ein Lied wiederholen & Ja\\
    \hline
    Repeat all & Queue wiederholen & Ja\\
    \hline
    Consume Mode & Ein abgespieltes Lied entfernen & Ja\\
    \hline
    Single Mode & Ein Lied abspielen, dann Stoppen & Ja\\
    \hline
\end{tabularx}
\newpage
\textbf{Kombinierte Ausführung:}\ \\ \\
\begin{tabularx}{\textwidth}{|X|X|l|}
    \hline
    \textbf{Testfall} & \textbf{Erwartetes Ergebnis} & \textbf{Ergebnis eingetroffen?}\\
    \hline
    Play, Stop, Play(1) & Musik spielt ab.\newline Musik stoppt.\newline Musik spielt ab. & Ja\\
    \hline
    Play, Pause, Play(2) & Musik spiel ab.\newline Musik pausiert.\newline Musik spiel ab. & Ja\\
    \hline
    Next, Next(3) & Skip weiter.\newline Skip weiter. & Ja\\
    \hline
    Previous, Previous(4) & Skip zurück.\newline Skip zurück & Ja\\
    \hline
    Next, Previous(5) & Skip weiter.\newline Skip zurück & Ja\\
    \hline
    Previous, Next(6) & Skip zurück.\newline Skip weiter & Ja\\
    \hline
    Random, Repeat all & Musik der Queue zufällig abspielen\newline Queue wiederholen & Ja\\
    \hline
    Random, Consume Mode & Musik der queue zufällig abspielen\newline Ein abgespieltes Lied entfernen & Ja\\
    \hline
    Random, Single Mode & Musik der Queue zufällig abspielen\newline Ein Lied abspielen, dann stoppen & Ja\\
    \hline
    Consume Mode, Single Mode & Ein abgespieltes Lied entfernen\newline Ein Lied abspielen, dann Stoppen & Ja\\
    \hline
    Consume Mode, Repeat all & Kann nur einmal durchlaufen & Ja\\
    \hline
    Random, 1 & Musik der Queue zufällig abspielen\newline 1 & Ja\\
    \hline
    Random, 2 & Musik der Queue zufällig abspielen\newline 2 & Ja\\
    \hline
    Random, 3 & Musik der Queue zufällig abspielen\newline 3 & Ja\\
    \hline
    Random, 4 & Musik der Queue zufällig abspielen\newline 4 & Ja\\
    \hline
    Random, 5 & Musik der Queue zufällig abspielen\newline 5 & Ja\\
    \hline
    Random, 6 & Musik der Queue zufällig abspielen\newline 6 & Ja\\
    \hline
    Repeat all, 1 & Queue wiederholen\newline 1 & Ja\\
    \hline
    Repeat all, 2 & Queue wiederholen\newline 2 & Ja\\
    \hline
    Repeat all, 3 & Queue wiederholen\newline 3 & Ja\\
    \hline
    Repeat all, 4 & Queue wiederholen\newline 4 & Ja\\
    \hline
\end{tabularx}
\begin{tabularx}{\textwidth}{|X|X|l|}
    %\begin{tabularx}[c]{|p{6cm}|p{6cm}|c|}
    \hline
    \textbf{Testfall} & \textbf{Erwartetes Ergebnis} & \textbf{Ergebnis eingetroffen?}\\
    \hline
    Repeat all, 5 & Queue wiederholen\newline 5 & Ja\\
    \hline
    Repeat all, 6 & Queue wiederholen\newline 6 & Ja\\
    \hline
    Consume Mode, 1 & Ein abgespieltes Lied entfernen\newline 1 & Ja\\
    \hline
    Consume Mode, 2 & Ein abgespieltes Lied entfernen\newline 2 & Ja\\
    \hline
    Consume Mode, 3 & Ein abgespieltes Lied entfernen\newline 3 & Ja\\
    \hline
    Consume Mode, 4 & Ein abgespieltes Lied entfernen\newline 4 & Ja\\
    \hline
    Consume Mode, 5 & Ein abgespieltes Lied entfernen\newline 5 & Ja\\
    \hline
    Consume Mode, 6 & Ein abgespieltes Lied entfernen\newline 6 & Ja\\
    \hline
    Single Mode, 1 & Ein Lied abspielen, dann Stoppen\newline 1 & Ja\\
    \hline
    Single Mode, 2 & Ein Lied abspielen, dann Stoppen\newline 2 & Ja\\
    \hline
    Single Mode, 3 & Ein Lied abspielen, dann Stoppen\newline 3 & Ja\\
    \hline
    Single Mode, 4 & Ein Lied abspielen, dann Stoppen\newline 4 & Ja\\
    \hline
    Single Mode, 5 & Ein Lied abspielen, dann Stoppen\newline 5 & Ja\\
    \hline
    Single Mode, 6 & Ein Lied abspielen, dann Stoppen\newline 6 & Ja\\
    \hline
\end{tabularx}
\newpage
Im folgenden wird auf das Protokoll der kombinierten Ausführung referenziert.\ \\ \\
\textbf{Mehrfache Ausführung:}\ \\ \\
\begin{tabularx}{\textwidth}{|X|X|l|}
    \hline
    \textbf{Testfall} & \textbf{Erwartetes Ergebnis} & \textbf{Ergebnis eingetroffen?}\\
    \hline
    Fall 1 x 10 & Fall 1 x 10 & Ja\\
    \hline
    Fall 2 x 10 & Fall 2 x 10 & Ja\\
    \hline
    Fall 3 x 10 & Fall 3 x 10 & Ja\\
    \hline
    Fall 4 x 10 & Fall 4 x 10 & Ja\\
    \hline
    Fall 5 x 10 & Fall 5 x 10 & Ja\\
    \hline
    Fall 6 x 10 & Fall 6 x 10 & Ja\\
    \hline
    Fall 12 x 10 & Fall 12 x 10 & Ja\\
    \hline
    Fall 13 x 10 & Fall 13 x 10 & Ja\\
    \hline
    Fall 14 x 10 & Fall 14 x 10 & Ja\\
    \hline
    Fall 15 x 10 & Fall 15 x 10 & Ja\\
    \hline
    Fall 16 x 10 & Fall 16 x 10 & Ja\\
    \hline
    Fall 17 x 10 & Fall 17 x 10 & Ja\\
    \hline
    Fall 18 x 10 & Fall 18 x 10 & Ja\\
    \hline
    Fall 19 x 10 & Fall 19 x 10 & Ja\\
    \hline
    Fall 20 x 10 & Fall 20 x 10 & Ja\\
    \hline
    Fall 21 x 10 & Fall 21 x 10 & Ja\\
    \hline
    Fall 22 x 10 & Fall 22 x 10 & Ja\\
    \hline
    Fall 23 x 10 & Fall 23 x 10 & Ja\\
    \hline
    Fall 24 x 10 & Fall 24 x 10 & Ja\\
    \hline
    Fall 25 x 10 & Fall 25 x 10 & Ja\\
    \hline
    Fall 26 x 10 & Fall 26 x 10 & Ja\\
    \hline
    Fall 27 x 10 & Fall 27 x 10 & Ja\\
    \hline
    Fall 28 x 10 & Fall 28 x 10 & Ja\\
    \hline 
    Fall 29 x 10 & Fall 29 x 10 & Ja\\
    \hline
    Fall 30 x 10 & Fall 30 x 10 & Ja\\
    \hline
    Fall 31 x 10 & Fall 31 x 10 & Ja\\
    \hline
    Fall 32 x 10 & Fall 32 x 10 & Ja\\
    \hline
    Fall 33 x 10 & Fall 33 x 10 & Ja\\
    \hline
    Fall 34 x 10 & Fall 34 x 10 & Ja\\
    \hline
    Fall 35 x 10 & Fall 35 x 10 & Ja\\
    \hline
\end{tabularx}
\subsubsection{Queue-Funktionen}
\textbf{Einfache Ausführung}\ \\ \\
\begin{tabularx}{\textwidth}{|X|X|l|}
    \hline
    \textbf{Testfall} & \textbf{Erwartetes Ergebnis} & \textbf{Ergebnis eingetroffen?}\\
    \hline
    Remove & Ein Lied aus Queue entfernen & Ja\\
    \hline
    Clear & Alle Lieder aus Queue entfernen & Ja\\
    \hline
    Save as Playlist & Queue als Playlist speichern & Ja\\
    \hline
    Suchen & Nach eingegebenem Wort suchen & Ja\\
    \hline
\end{tabularx}
\newpage
\textbf{Kombinierte Ausführung}\ \\ \\
Kombinierte Ausführung der Funktionen der Queue machen nicht wirklich viel Sinn da z.B.
die Funktion Clear die Queue löscht. Auch Save as Playlist wird wohl kaum öfter als einmal
pro Queue angewandt. Die einzige Kombination die Sinn macht getestet zu werden ist die 
folgende:\ \\ \\
\begin{tabularx}{\textwidth}{|X|X|l|}
    \hline
    \textbf{Testfall} & \textbf{Erwartetes Ergebnis} & \textbf{Ergebnis eingetroffen?}\\
    \hline
    Suchen, Remove & Nach eingegebenem Wort suchen\newline Ein Lied aus der Queue entfernen & Ja\\
    \hline
\end{tabularx}
\ \\ \\
\textbf{Mehrfache Ausführung}\ \\ \\
Die mehrfache Ausführung ist ähnlich unsinnig wie die der kombinierten Ausführung.
Mehrmals hintereinander die Queue löschen ist nicht möglich, genauso wie man wohl kaum 
mehrmals die gleiche Playlist erstellt. So bleibt wieder nur ein Testfall zu prüfen:\ \\ \\
\begin{tabularx}{\textwidth}{|X|X|l|}
    \hline
    \textbf{Testfall} & \textbf{Erwartetes Ergebnis} & \textbf{Ergebnis eingetroffen?}\\
    \hline
    Suchen, Remove x 10 & Nach eingegebenem Wort suchen\newline Ein Lied aus der Queue entfernen x 10& Ja\\
    \hline
\end{tabularx}
\subsubsection{Playlist-Funktionen}
\textbf{Einfache Ausführung}\ \\ \\
\begin{tabularx}{\textwidth}{|X|X|l|}
    \hline
    \textbf{Testfall} & \textbf{Erwartetes Ergebnis} & \textbf{Ergebnis eingetroffen?}\\
    \hline
    Hinzufügen & Playlist hinzufügen & Ja\\
    \hline
    Ersetzen & Playlist ersetzen & Ja\\
    \hline
    Playlist entfernen & Playlist löschen & Ja\\
    \hline
\end{tabularx}
\ \\ \\
\textbf{Kombinierte Ausführung}\ \\ \\
\begin{tabularx}{\textwidth}{|X|X|l|}
    \hline
    \textbf{Testfall} & \textbf{Erwartetes Ergebnis} & \textbf{Ergebnis eingetroffen?}\\
    \hline
    Hinzufügen, Hinzufügen & Playlist hinzufügen\newline Playlist hinzufügen & Ja\\
    \hline
    Ersetzen, Ersetzen & Playlist ersetzen\newline Playlist ersetzen & Ja\\
    \hline
    Entfernen, Entfernen & Playlist entfernen\newline Playlist entfernen & Ja\\
    \hline
    Hinzufügen, Entfernen & Playlist hinzufügen\newline Playlist löschen & Ja\\
    \hline
    Ersetzen, Entfernen & Playlist ersetzen\newline Playlist entfernen & Ja\\
    \hline
\end{tabularx}
\newpage
Im folgenden wird auf das Protokoll der kombinierten Ausführung referenziert.\ \\ \\
\textbf{Mehrfache Ausführung:}\ \\ \\
\begin{tabularx}{\textwidth}{|X|X|l|}
    \hline
    \textbf{Testfall} & \textbf{Erwartetes Ergebnis} & \textbf{Ergebnis eingetroffen?}\\
    \hline
    Fall 1 x 10 & Fall 1 x 10 & Ja\\
    \hline
    Fall 2 x 10 & Fall 2 x 10 & Ja\\
    \hline
    Fall 3 x 10 & Fall 3 x 10 & Ja\\
    \hline
    Fall 4 x 10 & Fall 4 x 10 & Ja\\
    \hline
    Fall 5 x 10 & Fall 5 x 10 & Ja\\
    \hline
\end{tabularx}
\subsubsection{Dateibrowser-Funktionen}
\textbf{Einfache Ausführung}\ \\ \\
\begin{tabularx}{\textwidth}{|X|X|l|}
    \hline
    \textbf{Testfall} & \textbf{Erwartetes Ergebnis} & \textbf{Ergebnis eingetroffen?}\\
    \hline
    Hinzufügen & Zur Queue hinzufügen & Ja\\
    \hline
    Alle Hinzufügen & Alle zur Queue hinzufügen & Ja\\
    \hline
    Ersetzen & Queue durch Auswahl ersetzen & Ja\\
    \hline
    Aktualisieren & Dateibrowser aktualisieren & Ja\\
    \hline
    Neu einlesen & Dateibrowser neu einlesen & Ja\\
    \hline
    Suchen & Nach eingegebenem Wort suchen & Ja\\
    \hline
\end{tabularx}
\ \\ \\

\newpage
\textbf{Kombinierte Ausführung}\ \\ \\
\begin{tabularx}{\textwidth}{|X|X|l|}
    \hline
    \textbf{Testfall} & \textbf{Erwartetes Ergebnis} & \textbf{Ergebnis eingetroffen?}\\
    \hline
    Hinzufügen\newline Hinzufügen & Zur Queue hinzufügen\newline Zur Queue hinzufügen & Ja\\
    \hline
    Alle Hinzufügen\newline Alle hinzufügen &  Alle zur Queue hinzufügen\newline Alle zur Queue hinzufügen & Ja\\
    \hline
    Ersetzen\newline Ersetzen & Queue durch Auswahl ersetzen\newline Queue durch Auswahl ersetzen & Ja\\
    \hline
    Aktualisieren\newline Aktualisieren & Dateibrowser aktualisieren\newline Dateibrowser aktualisieren & Ja\\
    \hline
    Neu einlesen\newline Neu einlesen & Dateibrowser neu einlesen\newline Dateibrowser neu einlesen & Ja\\
    \hline
    Suchen\newline Suchen & Nach eingegebenem Wort suchen\newline Nach eingegebenem Wort suchen & Ja\\
    \hline
    Hinzufügen\newline Alle Hinzufügen & Zur Queue hinzufügen\newline Alle zur Queue hinzufügen & Ja\\
    \hline
    Hinzufügen\newline Ersetzen & Zur Queue hinzufügen\newline Queue durch Auswahl ersetzen & Ja\\
    \hline
    Hinzufügen\newline Aktualisieren & Zur Queue hinzufügen\newline Dateibrowser aktualisieren & Ja\\
    \hline
    Hinzufügen\newline Neu einlesen & Zur Queue hinzufügen\newline Dateibrowser neu einlesen & Ja\\
    \hline
\end{tabularx}
\begin{tabularx}{\textwidth}{|X|X|l|}
    \hline
    \textbf{Testfall} & \textbf{Erwartetes Ergebnis} & \textbf{Ergebnis eingetroffen?}\\
    \hline
    Hinzufügen\newline Suchen & Zur Queue hinzufügen\newline Nach eingegebenem Wort suchen & Ja\\
    \hline
    Alle Hinzufügen\newline Ersetzen & Alle zur Queue hinzufügen\newline Queue durch Auswahl ersetzen & Ja\\
    \hline
    Alle Hinzufügen\newline Aktualisieren & Alle zur Queue hinzufügen\newline Dateibrowser aktualisieren & Ja\\
    \hline
    Alle Hinzufügen\newline Neu einlesen & Alle zur Queue hinzufügen\newline Dateibrowser neu einlesen & Ja\\
    \hline
    Alle Hinzufügen\newline Suchen & Alle zur Queue hinzufügen\newline Nach eingegebenem Wort suchen & Ja\\
    \hline
    Ersetzen\newline Aktualisieren & Queue durch Auswahl ersetzen\newline Dateibrowser aktualisieren & Ja\\
    \hline
    Ersetzen\newline Neu einlesen & Queue durach Auswahl ersetzen\newline Dateibrowser neu einlesen & Ja\\
    \hline
    Ersetzen\newline Suchen & Queue durch Auswahl ersetzen\newline Nach eingegebenem Wort suchen & Ja\\
    \hline
    Aktualisieren\newline Neu einlesen & Dateibrowser aktualisieren\newline Dateibrowser neu einlesen & Ja\\
    \hline
    Aktualisieren\newline Suchen & Dateibrowser aktualisieren\newline Nach eingegebenem Wort suchen & Ja\\
    \hline
    Neu einlesen\newline Suchen & Dateibrowser neu einlesen\newline Nach eingegebenem Wort suchen & Ja\\
    \hline
\end{tabularx}
\ \\ \\
Im folgenden wird auf das Protokoll der kombinierten Ausführung referenziert.\ \\ \\
\textbf{Mehrfache Ausführung}\ \\ \\
\begin{tabularx}{\textwidth}{|X|X|l|}
    \hline
    \textbf{Testfall} & \textbf{Erwartetes Ergebnis} & \textbf{Ergebnis eingetroffen?}\\
    \hline
    Fall 1 x 10 & Fall 1 x 10 & Ja\\
    \hline
    Fall 2 x 10 & Fall 2 x 10 & Ja\\
    \hline
    Fall 3 x 10 & Fall 3 x 10 & Ja\\
    \hline
    Fall 4 x 10 & Fall 4 x 10 & Ja\\
    \hline
    Fall 5 x 10 & Fall 5 x 10 & Ja\\
    \hline
    Fall 6 x 10 & Fall 6 x 10 & Ja\\
    \hline
    Fall 7 x 10 & Fall 7 x 10 & Ja\\
    \hline
    Fall 8 x 10 & Fall 8 x 10 & Ja\\
    \hline
    Fall 9 x 10 & Fall 9 x 10 & Ja\\
    \hline
    Fall 10 x 10 & Fall 10 x 10 & Ja\\
    \hline
\end{tabularx}
\begin{tabularx}{\textwidth}{|X|X|l|}
    \hline
    \textbf{Testfall} & \textbf{Erwartetes Ergebnis} & \textbf{Ergebnis eingetroffen?}\\
    \hline
    Fall 11 x 10 & Fall 11 x 10 & Ja\\
    \hline
    Fall 12 x 10 & Fall 12 x 10 & Ja\\
    \hline
    Fall 13 x 10 & Fall 13 x 10 & Ja\\
    \hline
    Fall 14 x 10 & Fall 14 x 10 & Ja\\
    \hline
    Fall 15 x 10 & Fall 15 x 10 & Ja\\
    \hline
    Fall 16 x 10 & Fall 16 x 10 & Ja\\
    \hline
    Fall 17 x 10 & Fall 17 x 10 & Ja\\
    \hline
    Fall 18 x 10 & Fall 18 x 10 & Ja\\
    \hline
    Fall 19 x 10 & Fall 19 x 10 & Ja\\
    \hline
    Fall 20 x 10 & Fall 20 x 10 & Ja\\
    \hline
    Fall 21 x 10 & Fall 21 x 10 & Ja\\
    \hline
\end{tabularx}
\subsubsection{Statistik}
Für die Statistik kann kein Testprotokoll angewandt werden.
\subsubsection{Einstellungen}
\textbf{Einfache Ausführung}\ \\ \\
\begin{tabularx}{\textwidth}{|X|X|l|}
    \hline
    \textbf{Testfall} & \textbf{Erwartetes Ergebnis} & \textbf{Ergebnis eingetroffen?}\\
    \hline
    Zeige Liste & Zeige Avahi Liste & Ja\\
    \hline
\end{tabularx}
\ \\ \\
\textbf{Kombinierte Ausführung}\ \\ \\

Es existieren keine Buttons oder Shortcuts die kombiniert werden könnten.\ \\ \\

\begin{tabularx}{\textwidth}{|X|X|l|}
    \hline
    \textbf{Testfall} & \textbf{Erwartetes Ergebnis} & \textbf{Ergebnis eingetroffen?}\\
    \hline
    Fall 1 x 10 & Fall 1 x 10 & Ja\\
    \hline
\end{tabularx}
\subsubsection{Lautstärke}
\textbf{Einfache Ausführung}\ \\ \\
\begin{tabularx}{\textwidth}{|X|X|l|}
    \hline
    Lautstärke erhöhen & Lautstärke erhöhen & Ja\\
    \hline
    Lautstärke verringern & Lautstärke verringern & Ja\\
    \hline
    \textbf{Testfall} & \textbf{Erwartetes Ergebnis} & \textbf{Ergebnis eingetroffen?}\\
    \hline
\end{tabularx}
\ \\ \\
\textbf{Kombinierte Ausführung}\ \\ \\
\begin{tabularx}{\textwidth}{|X|X|l|}
    \hline
    \textbf{Testfall} & \textbf{Erwartetes Ergebnis} & \textbf{Ergebnis eingetroffen?}\\
    \hline
    Lautstärke erhöhen\newline Lautstärke verringern & Lautstärke erhöhen\newline Lautstärke verringern & Ja\\
    \hline
\end{tabularx}
\ \\ \\
Im folgenden wird auf das Protokoll der kombinierten Ausführung referenziert.\ \\ \\
\textbf{Mehrfache Ausführung}\ \\ \\
\begin{tabularx}{\textwidth}{|X|X|l|}
    \hline
    \textbf{Testfall} & \textbf{Erwartetes Ergebnis} & \textbf{Ergebnis eingetroffen?}\\
    \hline
    Fall 1 x 10 & Fall 1 x 10 & Ja\\
    \hline
\end{tabularx}
\subsubsection{Sonstiges}
\textbf{Einfache Ausführung}\ \\ \\
\begin{tabularx}{\textwidth}{|X|X|l|}
    \hline
    \textbf{Testfall} & \textbf{Erwartetes Ergebnis} & \textbf{Ergebnis eingetroffen?}\\
    \hline
    Verbinden & Verbindung zum MPD-Server & Ja\\
    \hline
    Trennen & Verbindung zum Server trennen & Ja\\
    \hline
    Beenden & MPD-Client beenden & Ja\\
    \hline
\end{tabularx}
\ \\ \\
\textbf{Kombinierte Ausführung}\ \\ \\
\begin{tabularx}{\textwidth}{|X|X|l|}
    \hline
    \textbf{Testfall} & \textbf{Erwartetes Ergebnis} & \textbf{Ergebnis eingetroffen?}\\
    \hline
    Verbinden\newline Verbinden & Verbindung zum MPD-Server\newline Verbindung zum MPD-Server & Ja\\
    \hline
    Verbinden\newline Trennen & Verbindung zum MPD-Server\newline Verbindzung zum Server trennen & Ja\\
    \hline
    Verbinden\newline Beenden & Verbindung zum MPD-Server\newline MPD-Client beenden & Ja\\
    \hline
    Trennen\newline Beenden & Verbindung zum Server trennen\newline MPD-Client beenden & Ja\\
    \hline
\end{tabularx}
\ \\ \\
Im folgenden wird auf das Protokoll der kombinierten Ausführung referenziert.\ \\ \\
\textbf{Mehrfache Ausführung}\ \\ \\
\begin{tabularx}{\textwidth}{|X|X|l|}
    \hline
    \textbf{Testfall} & \textbf{Erwartetes Ergebnis} & \textbf{Ergebnis eingetroffen?}\\
    \hline
    Fall 1 x 10 & Fall 1 x 10 & Ja\\
    \hline
    Fall 2 x 10 & Fall 2 x 10 & Ja\\
    \hline
    Fall 3 x 10 & Nur 1 x ausführbar & Ja\\
    \hline
    Fall 4 x 10 & Nur 1 x ausführbar & Ja\\
    \hline
\end{tabularx}
