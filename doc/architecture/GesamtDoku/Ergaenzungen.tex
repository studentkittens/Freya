\chapter{Ergänzungen}
\section{Realisierung}
Das System muss mit den Programmiersprachen C und/oder C++ realisiert werden. Dabei ist auf
Objektorientierung zu achten, um Modularität und Wartbarkeit gewährleisten zu können.
Es können beliebige Entwicklungsumgebungen verwendet werden. Um einfaches und sicheres arbeiten
ermöglichen zu können, soll die Versionsverwaltungssoftware \"git\" benutzt werden, um die
Entwicklungsdateien zu speichern und zu bearbeiten. Zu dem Projekt soll eine ausführliche
Dokumentation erstellt werden, um dauerhafte Wartbarkeit und Anpassung des MPD-Client  gewährleisten
zu können, dazu gehören auch entsprechende Software-Diagramme (wie z.B. UML).
\section{Die nächste Version}
Aufgrund des modularen Aufbaus kann das System beliebig oft und in verschiedene Richtungen weiterentwickelt werden.
