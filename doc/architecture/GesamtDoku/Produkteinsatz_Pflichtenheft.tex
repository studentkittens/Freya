\chapter{Produkteinsatz}
Welche Anwendungsbereiche (Zweck), Zielgruppen (Wer mit welchen Qualifikationen), Betriebsbedingungen (Betriebszeit,
Aufsicht)?\ \\ \\
Der MPD-Client ist nicht auf bestimmte Gewerbe beschränkt, ein jeder soll diesen Client
verwenden können. Grundlage für die Verwendung der Software ist die General public license (GNU)
Version 3 vom 29 Juni 2007.\ \\ \\
Definition der GNU:
\begin{center}
http://www.gnu.org/licenses/gpl.html
\end{center}
\section{Anwendungsbereiche}
Einzelpersonen verwenden dieses System, um überall da wo mit einem Rechner und dem
einem Unix-artigen Betriebssystem Musik abgespielt werden soll, dies auf einfache und
komfortable Weise tun zu können.
\section{Zielgruppen}
Personengruppen die komfortabel von überall aus auf ihre Musik und Abspiellisten zugreifen
wollen ohne diese jedes mal aufwändig synchronisieren zu müssen (z.B. durch Abgleich von Datenträgern).\ \\ \\
Es werden Basiskenntnisse zum Aufbau einer Netzwerkverbindung und zur Nutzung des Internets vorausgesetzt.
Aufgrund der für das System vorgesehenen Betriebsumgebung sind ebenso Kenntnisse im Umgang mit Unix nötig.\ \\ \\
Solange keine weiteren Sprachpakete installiert worden sind, muss der Benutzer die Systemsprache Englisch verstehen.
\section{Betriebsbedingungen}
Das System soll sich bezüglich der Betriebsbedingungen nicht sonderlich von vergleichbaren Systemen bzw.
Anwendungen unterscheiden und dementsprechend folgend Punkte erfüllen:
\begin{itemize}
		\item Betriebsdauer: Täglich, 24 Stunden
		\item Keinerlei Wartung soll nötig sein
		\item Sicherungen der Konfiguration müssen vom Benutzer vorgenommen werden
\end{itemize}
