\chapter{Produktfunktionen}
Funktionen des MPD-Clients.\ \\ \\
Beim ersten Start des Systems soll eine Standard-Konfiguration geladen werden und die Verbindungseinstellungen
zu einem MPD-Server müssen vorgenommen werden. Bei jedem weiteren Start soll die Konfiguration geladen werden,
die vom Benutzer erstellt wurde, falls diese denn lokal gefunden werden kann. Der Benutzer soll sämtliche
Einstellungen selbstverständlich zu jeder Zeit ändern können.
\section{Allgemeine Funktionen}
\subsection{Starten und Beenden}
\renewcommand{\labelitemi}{•}
\begin{itemize}
	\item F\_0010 Der Benutzer kann das System zu jedem Zeitpunkt starten.
	\item F\_0020 Der Benutzer kann das System zu jedem Zeitpunkt beenden.
	\item F\_0030 Beim ersten Start wird ein Standart-System-Zustand geladen.
	\item F\_0040 Beim Beenden wird der aktuelle System-Zustand gespeichert.
	\item F\_0050 Bei jedem weiteren Start wird der letzte System-Zustand geladen.
\end{itemize}
\section{Benutzerfunktionen}
\subsection{Benutzer-Kennung}
Der Benutzer verfügt über ein persönliches Passwort, dass er zu jeder Zeit ändern kann. Vorrausgesetzt
er hat das aktuelle Passwort eingegeben.
\begin{itemize}
	\item F\_0110 Der Benutzer kann ein Passwort erzeugen und es ändern
\end{itemize}
\subsection{Persönliche Daten}
Neben den Passwort, gibt es auch Verbindungseinstellungen, die vorgenommen und ebenfalls zu jedem Zeitpunkt
geändert werden können.
\begin{itemize}
	\item F\_0210 Der Benutzer kann Verbindungseinstellungen vornehmen und sie ändern
\end{itemize}
\subsection{Persönliche Konfiguration}
Platzhalter
\subsection{Persönliches Profil}
Da die Software auf Unix-artige Systeme beschränkt ist, wurde keine Profil-Verwaltung implementiert. Die
verschiedenen Profile werden durch die verschiedenen Profile des gesamten Betriebssystems definiert und differenziert.
\subsection{Persönliche Datenbank}
Eine persönliche Datenbank ist lokal nicht vorhanden. Die Datenbank des Benutzers befindet sich auf dem MPD-Server.
Einzig und alleine modulare Erweiterungen des MPD-Clients können lokale Datenbank-Implementierungen erfordern.
\subsection{Kommunikation (Chat)}
Kommunikation von MPD-Client zu MPD-Client kann theoretisch implementiert werden, eine solche Schnittstelle ist vorhanden.
Allerdings wurde hierauf verzichtet, da im Vordergrund das Abspielen und Verwalten von Musik steht und es deutlich
einfachere und bessere Systeme gibt, mit Hilfe derer man kommunizieren kann.
\subsection{Suchen}
Eine einfache Textsuche zum finden von Titeln, Alben oder Interpreten innerhalb der Abspiellisten wurde implementiert
werden. Dabei springt die Markierung des Textes beim eingeben von Zeichen in die Suche zu der ersten übereinstimmenden
Stelle in der Playlist des Clients. Erst beim bestätigen der Eingabe im Suchfeld wird die Auswahl gefiltert.
\begin{itemize}
        \item F\_0310 Der Benutzer kann seine Playliste durchsuchen.
\end{itemize}
\section{Abspielfunktionen}
Platzhalter
\subsection{Initialisierung}
Platzhalter
\subsection{Verlauf}
\section{Administrator-Funktionen}
Durch das Unix-artige System wird auch der Administrator-Zugriff geregelt. Sobald sich der Benutzer im Unix System
als Administrator befindet, kann er auch den MPD-Client administrieren. Ein zusätzlicher Administrator-Modus wurde also
nicht implementiert.
