\documentclass[11pt]{scrreprt}%oldschool: report

\usepackage[utf8]{inputenc}
\usepackage{ngerman}
\usepackage{fullpage} % kleinere Ränder

\usepackage{comment} % für größere comments: \begin{comment} ... \end{comment}

% *** für eingefügte (pdf-)Grafiken
\usepackage[pdftex]{graphicx} 
\pdfminorversion=6
% ***

\usepackage{enumerate} %für geschachtelte Aufzählungen

% *** java listings
\usepackage{listings} 
\usepackage{courier} % courier schrift
\lstset{numbers=left, numberstyle=\tiny, basicstyle=\ttfamily  ,numbersep=5pt, tabsize=2}
\lstset{language=Java}
% ***

\usepackage{amsmath} %Matheformeln usw.
\usepackage{amssymb} %mathfrak

\usepackage[bookmarks=true]{hyperref} % hyperrefs aktivieren
\setcounter{secnumdepth}{3} %Numerierung bis Tiefe 3, also ab \paragraph ohne

%*** title usw.
\title{Lastenheft}
\subtitle{Teil der Software Engineering II Studienarbeit WS 2011/2012, Inf 3}
\author{
Christopher Pahl,\\
Christoph Piechula,\\
Eduard Schneider,\\
und Marc Tigges}
\date{\today}
%***

%newcommands
%\newcommand{\neuesKommando}{Was zu tun ist}

\begin{document}
\maketitle
\tableofcontents
%\part{welcher Teil}
\chapter{Einführung}
Ausgangspunkt für die Überlegungen dieses Lastenhefts, ist eine Studienarbeit im Studiengang Informatik Semester drei, Software Engineering II.
Im Rahmen dieser Studienarbeit, soll nun ein solcher Music-Player-Daemon-Client (MPD-Client) erstellt werden, der zusätzlich zu den üblichen Funktionen noch einige erweiterte Möglichkeiten bieten soll. Hierfür muss zu aller erst geklärt werden, was ein Music-Player-Daemon ist:
\ \\ \\
Der MPD ist eine Client/Server-Architektur, in der die Clients und Server (MPD ist der Server) über ein Netzwerk interagieren. MPD ist also nur die Hälfte der Gleichung. Zur Nutzung von MPD, muss ein MPD-Client (auch bekannt als MPD-Schnittstelle) installiert werden.

\section{Definition des MPD}
\begin{quote}
Der Music Player Daemon (kurz MPD) ist ein Unix-Systemdienst, der das Abspielen
von Musik auf einem Computer ermöglicht. Er unterscheidet sich von gewöhnlichen
Musik-Abspielprogrammen dadurch, dass eine strikte Trennung von Benutzeroberfläche und Programmkern vorliegt. Dadurch ist die grafische Benutzeroberfläche auswechselbar und auch eine Fernsteuerung des Programms über das Netzwerk möglich.
Die Schnittstelle zwischen Client und Server ist dabei offen dokumentiert und der
MPD selbst freie und quelloffene Software.\ \\ \\
Der MPD kann wegen seines geringen Ressourcenverbrauchs nicht nur auf Standardrechnern sondern auch auf einem abgespeckten Netzwerkgerät mit Audioausgang
betrieben werden und von allen Computern oder auch Mobiltelefonen / PDAs im Netzwerk ferngesteuert werden. Es ist auch möglich den Daemon und den Client zur Fernsteuerung lokal auf dem gleichen Rechner zu betreiben, er fungiert dann als normaler Medienspieler, der jedoch von einer Vielzahl unterschiedlicher Clients angesteuert werden kann, die sich in Oberflächengestaltung und Zusatzfunktionen unterscheiden. Mittlerweile existieren auch zahlreiche Clients, die eine Webschnittstelle bereitstellen.\ \\ \\
Der MPD spielt die Audioformate Ogg Vorbis, FLAC, OggFLAC, MP2, MP3, MP4/AAC, MOD, Musepack und wave ab. Zudem können FLAC-, OggFLAC-
, MP3- und OggVorbis-HTTP-Streams abgespielt werden. Die Schnittstelle kann
auch ohne manuelle Konfiguration mit der Zeroconf-Technik angesteuert werden.
Des Weiteren wird Replay Gain, Gapless Playback, Crossfading und das Einlesen
von Metadaten aus ID3-Tags, Vorbis comments oder der MP4-Metadatenstruktur
unterstützt.
\footnote{Zitat aus: http://de.wikipedia.org/wiki/Music\_Player\_Daemon}
\end{quote}

\newpage
\section{Definition des MPD-Client}
Der Music Player Daemon Client ist nun die Schnittstelle zum MPD. Über diesen Client kann der MPD gesteuert werden. Es gibt viele verschiedene Clients mit unterschiedlichsten Funktionen, da der Client nicht auf den Funktionsumfang des MPD begrenzt ist. Das heißt im Klartext, dass der Client zwar nur die Funktionen über das Netzwerk steuern kann, die vom MPD implementiert sind, aber nicht, dass er deshalb auch keine lokalen Dienste bzw. Funktionen anwenden kann. So kann ein Client beispielsweise auch Album-Cover speichern, Funktionen eines Equalizers bereitstellen, Musik Taggen, und vieles mehr.

\section{Ziel und Zielgruppe sowie grobe Planung}
Ziel ist es also, einen solchen Client durch Anwendung Software Engineering spezifischer Methoden für den MPD zu erstellen, der zusätzlich Nicht-MPD-bezogene Funktionen bietet. 
\ \\ \\
Die Zielgruppe umfasst dabei jede Person die einen Rechner benutzt um darauf Musik abzuspielen und das Betriebssystem Linux verwendet.
\ \\ \\
Für dieses Projekt haben wir nun ein Team aus vier Personen zusammengestellt und werden über einen Zeitraum von drei bis vier Monaten besagten Client erstellen. Dabei halten wir uns an die Vorgehensweise, die vom Wasserfallmodell mit Rücksprung beschrieben wird. Die wesentlichen Schritte sind dabei die Initialisierung des Projekts, die Analyse, der Entwurf, die Realisierung, die Einführung und schließlich die Anwendung. Abgabe Termin der Studienarbeit, ist der 14. Januar 2012.

\input{IST}
\input{SOLL}
\input{Schnittstellen}
\input{FunktionaleAnforderungen}
\input{NichtFunktionaleAnforderungen}
\input{Risikoakzeptanz}
\input{Skizze}
\input{Lieferumfang}
\chapter{Abnahmekriterien}

\section{Platzhalter}

\begin{quote}
Platzhalter
\footnote{Zitat aus: http://de.wikipedia.org/wiki/Music\_Player\_Daemon}
\end{quote}

\newpage
\subsection{Platzhalter}

\renewcommand{\labelitemi}{•}
\begin{itemize}
	\item Platzhalter
	
	\renewcommand{\labelitemi}{--}
	\begin{itemize}
		\item Platzhalter
	\end{itemize}
\end{itemize}


\end{document}
