\documentclass[11pt]{scrreprt}%oldschool: report

\usepackage[utf8]{inputenc}
\usepackage{ngerman}
\usepackage{fullpage} % kleinere Ränder

\usepackage{comment} % für größere comments: \begin{comment} ... \end{comment}

% *** für eingefügte (pdf-)Grafiken
\usepackage[pdftex]{graphicx} 
\pdfminorversion=6
% ***

\usepackage{enumerate} %für geschachtelte Aufzählungen

% *** java listings
\usepackage{listings} 
\usepackage{courier} % courier schrift
\lstset{numbers=left, numberstyle=\tiny, basicstyle=\ttfamily  ,numbersep=5pt, tabsize=2}
\lstset{language=Java}
% ***

\usepackage{amsmath} %Matheformeln usw.
\usepackage{amssymb} %mathfrak

\usepackage[bookmarks=true]{hyperref} % hyperrefs aktivieren
\setcounter{secnumdepth}{3} %Numerierung bis Tiefe 3, also ab \paragraph ohne

%*** title usw.
\title{Lastenheft}
\subtitle{Teil der Software Engineering II Studienarbeit WS 2011/2012, Inf 3}
\author{
Christopher Pahl,\\
Christoph Piechula,\\
Eduard Schneider,\\
und Marc Tigges}
\date{\today}
%***

%newcommands
%\newcommand{\neuesKommando}{Was zu tun ist}

\begin{document}
\maketitle
\tableofcontents
%\part{welcher Teil}
\chapter{Einleitung}
Ziel dieser Studienarbeit ist die vollständige Bearbeitung einer vorgegebenen Aufgabenstellung
nach einem selbst gewählten Vorgehensmodell. Die Aufgabenstellung schreibt vor, sich in einer
Gruppe zusammen zu finden und gemeinsam ein Software-Projekt zu bearbeiten und dabei strukturiert
 und professionell vorzugehen.
\begin{quote}
    \section{Rahmenbedingungen}
    \renewcommand{\labelitemi}{•}
    \begin{itemize}
        \item Persistente Datenspeicherung
        \begin{itemize}
	    \item Datei oder Datenbank (wenn schon bekannt)
        \end{itemize}
        \item Netzwerk-Programmierung
        \begin{itemize}
	    \item Eine verteilte Architektur (z.B.: Client/Server)
        \end{itemize}
        \item GUI
        \begin{itemize}
	    \item Swing
	    \item Web-basiert
        \end{itemize}
    \end{itemize}
    \section{Prozess-Anforderungen}
    \begin{itemize}
        \item Dokumentation aller Phasen(Analyse bis Testen)
        \item Auswahl eines konkreten Prozessmodells
        \begin{itemize}
	    \item Z.B. sd\&m, M3, RUP, Agile Methoden ...
	    \item Begründung (warum dieser Prozess passt zu Ihrem System)
        \end{itemize}
        \item Erstellung der Dokumente und UML-Diagramme
        \begin{itemize}
	    \item Visio
	    \item UML Werkzeuge (freie Wahl)
        \end{itemize}
        \item Fertige Implementierung 
        \begin{itemize}
	    \item Es kann mehr spezifiziert sein als implementiert
        \end{itemize}
        \item Spezifikation von Testszenarien
        \begin{itemize}
	    \item und der Beleg der erfolgreichen Ausführung
        \end{itemize}
        \item Lauffähiges System
    \end{itemize}
    \section{Mögliche Themen}
    \begin{itemize}
        \item CRM Systeme
        \begin{itemize}
	    \item Bibliothek
	    \item Musikshop
	    \item ...
        \end{itemize}
        \item Kommunikationssysteme
	\item Chat-Variationen (Skype, etc.)
	\item File-Verwaltungs-Systeme (eigener Cloud-Dienst)
	\item ...
    \end{itemize}
    \item Portale
    \begin{itemize}
	\item Mitfahrgelegenheit
	\item Dating-Agentur ;)
	\item ...
    \end{itemize}
\footnote{Folie Anforderungen, Autor Prof. Dr. Philipp Schaible, WS 2011/2012, Inf 3}
\end{quote}
Diese Arbeit ist wichtig, um den Studenten zu zeigen, wie man in einem Team zusammenarbeitet und nach
Software-Engineering-Methoden qualitativ hochwertige Software erstellt. Es geht im Folgenden um einen
Music-Player-Daemon-Client (Näheres bitte der Definition entnehmen). Dieses Thema wird behandelt, da es
alle Rahmenbedingungen abdeckt und im Interesse der Autoren liegt. Die Besonderheit liegt darin, dass
sich diese Software nach Fertigstellung auch wirklich anwenden lässt. Ziel ist die Erweiterung der
Fähigkeitn im Bereich der Software Engineering sowie das Erlernen von Methoden für wissenschaftliches Arbeiten.




\chapter{Produkteinsatz}
Für welche Anwendungsbereiche und Zielgruppe ist die Software vorgesehen?\ \\ \\
Die Software soll überall da eingesetzt werden, wo Musik abgespielt werden soll. Dabei ist man nicht auf einen Rechner beschränkt, auch Fernseher und Musik-Spieler mit Internetzugang und entsprechender Softwareunterstützung können theoretisch eine solches Programm verwenden.\ \\ \\
Hauptsächlich soll sich diese Software allerdings an Nutzer eines Rechners mit einem Unix-artigen System richten. Des weiteren soll die Zielgruppe vorerst auf auf Benutzer, die der englischen Sprache mächtig sind beschränkt werden.\ \\ \\
\chapter{Produktfunktionen}
Welche sind die Hauptfunktionen aus Sicht des Auftraggebers?
\section{Benutzerfunktionen}
Beim ersten Start des Systems soll eine Standard-Konfiguration geladen werden und die Verbindungseinstellungen
zu einem MPD-Server müssen vorgenommen werden. Bei jedem weiteren Start soll die Konfiguration geladen werden,
die vom Benutzer erstellt wurde, falls diese denn lokal gefunden werden kann. Der Benutzer soll sämtliche
Einstellungen selbstverständlich zu jeder Zeit ändern können.
\subsection{Starten und Beenden}
\renewcommand{\labelitemi}{•}
\begin{itemize}
	\item F\_0010 Der Benutzer kann das System zu jedem Zeitpunkt starten.
	\item F\_0020 Der Benutzer kann das System zu jedem Zeitpunkt beenden.
\end{itemize}
\subsection{Persönliche Daten}
Ein Benutzer verfügt über ein persönliches Passwort und einer persönlichen Verbindungseinstellung zum gewünschten
MPD-Server. Diese Daten können von dem Benutzer zu jeder Zeit angepasst werden, selbstverständlich nachdem das
Passwort eingegeben wurde.
\begin{itemize}
	\item F\_0110 Der Benutzer kann sich zu jeder Zeit seine Verbindungsdaten anzeigen lassen.
	\item F\_0120 Der Benutzer kann zu jeder Zeit seine persönlichen Daten anpassen.
\end{itemize}
\subsection{Persönliches Profil}
Da die Software auf Unix-artige Systeme beschränkt werden soll, geht ein angenehmer Vorteil mit einher, nämlich das
eine Profil-Verwaltung seitens des MPD-Clients nicht implementiert werden muss. Die verschiedenen Profile werden
durch die verschiedenen Profile des gesamten Betriebssystems definiert und differenziert.
\subsection{Persönliche Datenbank}
Eine persönliche Datenbank soll lokal nicht vorhanden sein. Die Datenbank des Benutzers befindet sich auf dem MPD-Server.
Einzig und alleine modulare Erweiterungen des MPD-Clients können lokale Datenbank-Implementierungen erfordern.
\subsection{Kommunikation (Chat)}
Kommunikation von MPD-Client zu MPD-Client kann theoretisch implementiert werden, eine solche Schnittstelle ist vorhanden.
Allerdings soll hierauf verzichtet werden, da im Vordergrund das Abspielen und Verwalten von Musik steht und es deutlich
einfachere und bessere Systeme gibt, mit Hilfe derer man kommunizieren kann.
\subsection{Suchen}
Eine einfache Textsuche zum finden von Titeln, Alben oder Interpreten innerhalb der Abspiellisten soll implementiert werden.
\begin{itemize}
	\item F\_0210 Der Benutzer kann seine Playliste durchsuchen.
\end{itemize}
\section{Administrator-Funktionen}
Durch das Unix-artige System soll auch der Administrator-Zugriff geregelt werden. Sobald sich der Benutzer im Unix System
als Administrator befindet, kann er auch den MPD-Client administrieren. Ein zusätzlicher Administrator-Modus muss also
nicht implementiert werden.

\chapter{Produktdaten}
Welche Daten sollen persistent gespeichert werden?\ \\ \\
Die vom Benutzer vorgenommenen Verbindungseinstellungen und Client spezifischen Einstellungen,
sollen auf dem Rechner lokal und persistent gespeichert werden. Nur so kann ermöglicht werden,
dass nach jedem Start des Systems diese Einstellungen geladen und übernommen werden können.\ \\
Außerdem soll eine Log-Datei auf den einzelnen Rechnern angelegt werden, die dieses System
verwenden. In dieser Log-Datei werden Nachrichten des Systems gespeichert, um eventuelle Fehler
leicht finden und beheben zu können. Es soll zusätzlich der Zustand des Systems abgespeichert werden,
wenn das System beendet wird um das System beim nächsten Start in diesen Zustand versetzen zu können.
\renewcommand{\labelitemi}{•}
\begin{itemize}
	\item D\_0010 Persönlichen Verbindungseinstellungen.
	\begin{itemize}
		\item Platzhalter
		\item Platzhalter
	\end{itemize}
	\item D\_0020 Persönliches Passwort.
	\item D\_0030 Client spezifische Einstellungen.
	\begin{itemize}
		\item Platzhalter
		\item Platzhalter
	\end{itemize}
	\item D\_0040 Eine Log-Datei.
	\begin{itemize}
		\item Platzhalter
		\item Platzhalter
	\end{itemize}
	\item D\_0050 Der Zustand.
	\begin{itemize}
		\item Platzhalter
		\item Platzhalter
	\end{itemize}
\end{itemize}

\chapter{Produktleistungen}


\chapter{Qualitätsanforderungen}


\chapter{Ergänzungen}

\end{document}
