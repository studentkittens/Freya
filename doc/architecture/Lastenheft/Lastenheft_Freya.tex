\documentclass[11pt]{scrreprt}%oldschool: report

\usepackage[utf8]{inputenc}
\usepackage{ngerman}
\usepackage{fullpage} % kleinere Ränder

\usepackage{comment} % für größere comments: \begin{comment} ... \end{comment}

% *** für eingefügte (pdf-)Grafiken
\usepackage[pdftex]{graphicx} 
\pdfminorversion=6
% ***

\usepackage{enumerate} %für geschachtelte Aufzählungen

% *** java listings
\usepackage{listings} 
\usepackage{courier} % courier schrift
\lstset{numbers=left, numberstyle=\tiny, basicstyle=\ttfamily  ,numbersep=5pt, tabsize=2}
\lstset{language=Java}
% ***

\usepackage{amsmath} %Matheformeln usw.
\usepackage{amssymb} %mathfrak

\usepackage[bookmarks=true]{hyperref} % hyperrefs aktivieren
\setcounter{secnumdepth}{3} %Numerierung bis Tiefe 3, also ab \paragraph ohne

%*** title usw.
\title{Lastenheft}
\subtitle{Teil der Software Engineering II Studienarbeit WS 2011/2012, Inf 3}
\author{
Christopher Pahl,\\
Christoph Piechula,\\
Eduard Schneider,\\
und Marc Tigges}
\date{\today}
%***

%newcommands
%\newcommand{\neuesKommando}{Was zu tun ist}

\begin{document}
\maketitle
\tableofcontents
%\part{welcher Teil}
<<<<<<< HEAD
\chapter{Zielbestimmungen}
\section{Muss-Kriterien}
\renewcommand{\labelitemi}{•}
\begin{itemize}
	\item Server-Verbindung
	\begin{itemize}
		\item Platzhalter
		\item Platzhalter
		\item Platzhalter
	\end{itemize}
	\item Client-Einstellungen
	\begin{itemize}
		\item Platzhalter
		\item Platzhalter
		\item Platzhalter
	\end{itemize}
	\item Musik-Steuerung
	\begin{itemize}
		\item Platzhalter
		\item Platzhalter
		\item Platzhalter
	\end{itemize}
	\item Sonstiges
	\begin{itemize}
		\item Platzhalter
		\item Platzhalter
		\item Platzhalter
	\end{itemize}
\end{itemize}
\section{Wunsch-Kriterien}
\begin{itemize}
		\item Platzhalter
		\item Platzhalter
		\item Platzhalter
\end{itemize}
\section{Abgrenzungskriterien}
\begin{itemize}
		\item Platzhalter
		\item Platzhalter
		\item Platzhalter
\end{itemize}
\chapter{Produkteinsatz}
Für welche Anwendungsbereiche und Zielgruppe ist die Software vorgesehen?\ \\ \\
Die Software soll überall da eingesetzt werden, wo Musik abgespielt werden soll. Dabei ist man nicht auf einen Rechner beschränkt, auch Fernseher und Musik-Spieler mit Internetzugang und entsprechender Softwareunterstützung können theoretisch eine solches Programm verwenden.\ \\ \\
Hauptsächlich soll sich diese Software allerdings an Nutzer eines Rechners mit einem Unix-artigen System richten. Des weiteren soll die Zielgruppe vorerst auf auf Benutzer, die der englischen Sprache mächtig sind beschränkt werden.\ \\ \\
\chapter{Produktfunktionen}
Welche sind die Hauptfunktionen aus Sicht des Auftraggebers?
\section{Benutzerfunktionen}
Beim ersten Start des Systems soll eine Standard-Konfiguration geladen werden und die Verbindungseinstellungen
zu einem MPD-Server müssen vorgenommen werden. Bei jedem weiteren Start soll die Konfiguration geladen werden,
die vom Benutzer erstellt wurde, falls diese denn lokal gefunden werden kann. Der Benutzer soll sämtliche
Einstellungen selbstverständlich zu jeder Zeit ändern können.
\subsection{Starten und Beenden}
\renewcommand{\labelitemi}{•}
\begin{itemize}
	\item F\_0010 Der Benutzer kann das System zu jedem Zeitpunkt starten.
	\item F\_0020 Der Benutzer kann das System zu jedem Zeitpunkt beenden.
\end{itemize}
\subsection{Persönliche Daten}
Ein Benutzer verfügt über ein persönliches Passwort und einer persönlichen Verbindungseinstellung zum gewünschten
MPD-Server. Diese Daten können von dem Benutzer zu jeder Zeit angepasst werden, selbstverständlich nachdem das
Passwort eingegeben wurde.
\begin{itemize}
	\item F\_0110 Der Benutzer kann sich zu jeder Zeit seine Verbindungsdaten anzeigen lassen.
	\item F\_0120 Der Benutzer kann zu jeder Zeit seine persönlichen Daten anpassen.
\end{itemize}
\subsection{Persönliches Profil}
Da die Software auf Unix-artige Systeme beschränkt werden soll, geht ein angenehmer Vorteil mit einher, nämlich das
eine Profil-Verwaltung seitens des MPD-Clients nicht implementiert werden muss. Die verschiedenen Profile werden
durch die verschiedenen Profile des gesamten Betriebssystems definiert und differenziert.
\subsection{Persönliche Datenbank}
Eine persönliche Datenbank soll lokal nicht vorhanden sein. Die Datenbank des Benutzers befindet sich auf dem MPD-Server.
Einzig und alleine modulare Erweiterungen des MPD-Clients können lokale Datenbank-Implementierungen erfordern.
\subsection{Kommunikation (Chat)}
Kommunikation von MPD-Client zu MPD-Client kann theoretisch implementiert werden, eine solche Schnittstelle ist vorhanden.
Allerdings soll hierauf verzichtet werden, da im Vordergrund das Abspielen und Verwalten von Musik steht und es deutlich
einfachere und bessere Systeme gibt, mit Hilfe derer man kommunizieren kann.
\subsection{Suchen}
Eine einfache Textsuche zum finden von Titeln, Alben oder Interpreten innerhalb der Abspiellisten soll implementiert werden.
\begin{itemize}
	\item F\_0210 Der Benutzer kann seine Playliste durchsuchen.
\end{itemize}
\section{Administrator-Funktionen}
Durch das Unix-artige System soll auch der Administrator-Zugriff geregelt werden. Sobald sich der Benutzer im Unix System
als Administrator befindet, kann er auch den MPD-Client administrieren. Ein zusätzlicher Administrator-Modus muss also
nicht implementiert werden.

\chapter{Produktdaten}
Welche Daten sollen persistent gespeichert werden?\ \\ \\
Die vom Benutzer vorgenommenen Verbindungseinstellungen und Client spezifischen Einstellungen,
sollen auf dem Rechner lokal und persistent gespeichert werden. Nur so kann ermöglicht werden,
dass nach jedem Start des Systems diese Einstellungen geladen und übernommen werden können.\ \\
Außerdem soll eine Log-Datei auf den einzelnen Rechnern angelegt werden, die dieses System
verwenden. In dieser Log-Datei werden Nachrichten des Systems gespeichert, um eventuelle Fehler
leicht finden und beheben zu können. Es soll zusätzlich der Zustand des Systems abgespeichert werden,
wenn das System beendet wird um das System beim nächsten Start in diesen Zustand versetzen zu können.
\renewcommand{\labelitemi}{•}
\begin{itemize}
	\item D\_0010 Persönlichen Verbindungseinstellungen.
	\begin{itemize}
		\item Platzhalter
		\item Platzhalter
	\end{itemize}
	\item D\_0020 Persönliches Passwort.
	\item D\_0030 Client spezifische Einstellungen.
	\begin{itemize}
		\item Platzhalter
		\item Platzhalter
	\end{itemize}
	\item D\_0040 Eine Log-Datei.
	\begin{itemize}
		\item Platzhalter
		\item Platzhalter
	\end{itemize}
	\item D\_0050 Der Zustand.
	\begin{itemize}
		\item Platzhalter
		\item Platzhalter
	\end{itemize}
\end{itemize}

\chapter{Produktleistungen}


\chapter{Qualitätsanforderungen}


\chapter{Ergänzungen}
=======
\chapter{Einführung}
Ausgangspunkt für die Überlegungen dieses Lastenhefts, ist eine Studienarbeit im Studiengang Informatik Semester drei, Software Engineering II.
Im Rahmen dieser Studienarbeit, soll nun ein solcher Music-Player-Daemon-Client (MPD-Client) erstellt werden, der zusätzlich zu den üblichen Funktionen noch einige erweiterte Möglichkeiten bieten soll. Hierfür muss zu aller erst geklärt werden, was ein Music-Player-Daemon ist:
\ \\ \\
Der MPD ist eine Client/Server-Architektur, in der die Clients und Server (MPD ist der Server) über ein Netzwerk interagieren. MPD ist also nur die Hälfte der Gleichung. Zur Nutzung von MPD, muss ein MPD-Client (auch bekannt als MPD-Schnittstelle) installiert werden.

\section{Definition des MPD}
\begin{quote}
Der Music Player Daemon (kurz MPD) ist ein Unix-Systemdienst, der das Abspielen
von Musik auf einem Computer ermöglicht. Er unterscheidet sich von gewöhnlichen
Musik-Abspielprogrammen dadurch, dass eine strikte Trennung von Benutzeroberfläche und Programmkern vorliegt. Dadurch ist die grafische Benutzeroberfläche auswechselbar und auch eine Fernsteuerung des Programms über das Netzwerk möglich.
Die Schnittstelle zwischen Client und Server ist dabei offen dokumentiert und der
MPD selbst freie und quelloffene Software.\ \\ \\
Der MPD kann wegen seines geringen Ressourcenverbrauchs nicht nur auf Standardrechnern sondern auch auf einem abgespeckten Netzwerkgerät mit Audioausgang
betrieben werden und von allen Computern oder auch Mobiltelefonen / PDAs im Netzwerk ferngesteuert werden. Es ist auch möglich den Daemon und den Client zur Fernsteuerung lokal auf dem gleichen Rechner zu betreiben, er fungiert dann als normaler Medienspieler, der jedoch von einer Vielzahl unterschiedlicher Clients angesteuert werden kann, die sich in Oberflächengestaltung und Zusatzfunktionen unterscheiden. Mittlerweile existieren auch zahlreiche Clients, die eine Webschnittstelle bereitstellen.\ \\ \\
Der MPD spielt die Audioformate Ogg Vorbis, FLAC, OggFLAC, MP2, MP3, MP4/AAC, MOD, Musepack und wave ab. Zudem können FLAC-, OggFLAC-
, MP3- und OggVorbis-HTTP-Streams abgespielt werden. Die Schnittstelle kann
auch ohne manuelle Konfiguration mit der Zeroconf-Technik angesteuert werden.
Des Weiteren wird Replay Gain, Gapless Playback, Crossfading und das Einlesen
von Metadaten aus ID3-Tags, Vorbis comments oder der MP4-Metadatenstruktur
unterstützt.
\footnote{Zitat aus: http://de.wikipedia.org/wiki/Music\_Player\_Daemon}
\end{quote}

\newpage
\section{Definition des MPD-Client}
Der Music Player Daemon Client ist nun die Schnittstelle zum MPD. Über diesen Client kann der MPD gesteuert werden. Es gibt viele verschiedene Clients mit unterschiedlichsten Funktionen, da der Client nicht auf den Funktionsumfang des MPD begrenzt ist. Das heißt im Klartext, dass der Client zwar nur die Funktionen über das Netzwerk steuern kann, die vom MPD implementiert sind, aber nicht, dass er deshalb auch keine lokalen Dienste bzw. Funktionen anwenden kann. So kann ein Client beispielsweise auch Album-Cover speichern, Funktionen eines Equalizers bereitstellen, Musik Taggen, und vieles mehr.

\section{Ziel und Zielgruppe sowie grobe Planung}
Ziel ist es also, einen solchen Client durch Anwendung Software Engineering spezifischer Methoden für den MPD zu erstellen, der zusätzlich Nicht-MPD-bezogene Funktionen bietet. 
\ \\ \\
Die Zielgruppe umfasst dabei jede Person die einen Rechner benutzt um darauf Musik abzuspielen und das Betriebssystem Linux verwendet.
\ \\ \\
Für dieses Projekt haben wir nun ein Team aus vier Personen zusammengestellt und werden über einen Zeitraum von drei bis vier Monaten besagten Client erstellen. Dabei halten wir uns an die Vorgehensweise, die vom Wasserfallmodell mit Rücksprung beschrieben wird. Die wesentlichen Schritte sind dabei die Initialisierung des Projekts, die Analyse, der Entwurf, die Realisierung, die Einführung und schließlich die Anwendung. Abgabe Termin der Studienarbeit, ist der 14. Januar 2012.

\chapter{Beschreibung der Ist-Analyse}

\section{Platzhalter}

\begin{quote}
Platzhalter
\footnote{Zitat aus: http://de.wikipedia.org/wiki/Music\_Player\_Daemon}
\end{quote}

\newpage
\subsection{Platzhalter}

\renewcommand{\labelitemi}{•}
\begin{itemize}
	\item Platzhalter
	
	\renewcommand{\labelitemi}{--}
	\begin{itemize}
		\item Platzhalter
	\end{itemize}
\end{itemize}

\chapter{Beschreibung des Soll-Konzepts}

\section{Platzhalter}

\begin{quote}
Platzhalter
\footnote{Zitat aus: http://de.wikipedia.org/wiki/Music\_Player\_Daemon}
\end{quote}

\newpage
\subsection{Platzhalter}

\renewcommand{\labelitemi}{•}
\begin{itemize}
	\item Platzhalter
	
	\renewcommand{\labelitemi}{--}
	\begin{itemize}
		\item Platzhalter
	\end{itemize}
\end{itemize}

\chapter{Beschreibung von Schnittstellen}

\section{Platzhalter}

\begin{quote}
Platzhalter
\footnote{Zitat aus: http://de.wikipedia.org/wiki/Music\_Player\_Daemon}
\end{quote}

\newpage
\subsection{Platzhalter}

\renewcommand{\labelitemi}{•}
\begin{itemize}
	\item Platzhalter
	
	\renewcommand{\labelitemi}{--}
	\begin{itemize}
		\item Platzhalter
	\end{itemize}
\end{itemize}

\chapter{Funktionale Anforderungen}

\section{Platzhalter}

\begin{quote}
Platzhalter
\footnote{Zitat aus: http://de.wikipedia.org/wiki/Music\_Player\_Daemon}
\end{quote}

\newpage
\subsection{Platzhalter}

\renewcommand{\labelitemi}{•}
\begin{itemize}
	\item Platzhalter
	
	\renewcommand{\labelitemi}{--}
	\begin{itemize}
		\item Platzhalter
	\end{itemize}
\end{itemize}

\chapter{Nichtfunktionale Anforderungen}

\section{Platzhalter}

\begin{quote}
Platzhalter
\footnote{Zitat aus: http://de.wikipedia.org/wiki/Music\_Player\_Daemon}
\end{quote}

\newpage
\subsection{Platzhalter}

\renewcommand{\labelitemi}{•}
\begin{itemize}
	\item Platzhalter
	
	\renewcommand{\labelitemi}{--}
	\begin{itemize}
		\item Platzhalter
	\end{itemize}
\end{itemize}

\chapter{Risikoakzeptanz}

\section{Platzhalter}

\begin{quote}
Platzhalter
\footnote{Zitat aus: http://de.wikipedia.org/wiki/Music\_Player\_Daemon}
\end{quote}

\newpage
\subsection{Platzhalter}

\renewcommand{\labelitemi}{•}
\begin{itemize}
	\item Platzhalter
	
	\renewcommand{\labelitemi}{--}
	\begin{itemize}
		\item Platzhalter
	\end{itemize}
\end{itemize}

\chapter{Skizze des Entwicklungszyklus}

\section{Platzhalter}

\begin{quote}
Platzhalter
\footnote{Zitat aus: http://de.wikipedia.org/wiki/Music\_Player\_Daemon}
\end{quote}

\newpage
\subsection{Platzhalter}

\renewcommand{\labelitemi}{•}
\begin{itemize}
	\item Platzhalter
	
	\renewcommand{\labelitemi}{--}
	\begin{itemize}
		\item Platzhalter
	\end{itemize}
\end{itemize}

\chapter{Lieferumfang}

\section{Platzhalter}

\begin{quote}
Platzhalter
\footnote{Zitat aus: http://de.wikipedia.org/wiki/Music\_Player\_Daemon}
\end{quote}

\newpage
\subsection{Platzhalter}

\renewcommand{\labelitemi}{•}
\begin{itemize}
	\item Platzhalter
	
	\renewcommand{\labelitemi}{--}
	\begin{itemize}
		\item Platzhalter
	\end{itemize}
\end{itemize}

\chapter{Abnahmekriterien}

\section{Platzhalter}

\begin{quote}
Platzhalter
\footnote{Zitat aus: http://de.wikipedia.org/wiki/Music\_Player\_Daemon}
\end{quote}

\newpage
\subsection{Platzhalter}

\renewcommand{\labelitemi}{•}
\begin{itemize}
	\item Platzhalter
	
	\renewcommand{\labelitemi}{--}
	\begin{itemize}
		\item Platzhalter
	\end{itemize}
\end{itemize}

>>>>>>> 55dcd75023956d00bd62d25d2216cce0d324150f

\end{document}
