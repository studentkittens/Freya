\chapter{Produktfunktionen}
Welche sind die Hauptfunktionen aus Sicht des Auftraggebers?

\section{Benutzerfunktionen}
Beim ersten Start des Systems soll eine Standard-Konfiguration geladen werden und die Verbindungseinstellungen zu einem MPD-Server müssen vorgenommen werden. Bei jedem weiteren Start soll die Konfiguration geladen werden, die vom Benutzer erstellt wurde, falls diese denn lokal gefunden werden kann. Der Benutzer soll sämtliche Einstellungen selbstverständlich zu jeder Zeit ändern können.
\subsection{Starten und Beenden}
\renewcommand{\labelitemi}{•}
\begin{itemize}
	\item F\_0010 Der Benutzer kann das System zu jedem Zeitpunkt Starten.
	\item F\_0020 Der Benutzer kann das System zu jedem Zeitpunkt beenden.
\end{itemize}
\subsection{Persönliche Daten}
Ein Benutzer verfügt über ein persönliches Passwort und sowie einer persönlichen Verbindungseinstellung zum gewünschten MPD-Server. Diese Daten können von dem Benutzer zu jeder Zeit angepasst werden, selbstverständlich nachdem das Passwort eingegeben wurde.
\begin{itemize}
	\item F\_0110 Der Benutzer kann sich zu jeder Zeit seine Verbindungsdaten anzeigen lassen.
	\item F\_0120 Der Benutzer kann zu jeder Zeit seine persönlichen Daten anpassen.
\end{itemize}
\subsection{Persönliches Profil}
Da die Software auf Unix-artige Systeme beschränkt werden soll, geht ein angenehmer Vorteil mit einher, nämlich dass eine Profil-Verwaltung seitens des MPD-Clients nicht implementiert werden muss. Die verschiedenen Profile werden durch die verschiedenen Profile des gesamten Betriebssystems definiert und differenziert.
\subsection{Persönliche Datenbank}
Eine persönliche Datenbank soll lokal nicht vorhanden sein. Die Datenbank des Benutzers befindet sich auf dem MPD-Server. Einzig und alleine modulare Erweiterungen des MPD-Clients können lokale Datenbank-Implementierungen erfordern.
\subsection{Kommunikation (Chat)}
Kommunikation von MPD-Client zu MPD-Client kann theoretisch implementiert werden, eine solche Schnittstelle ist vorhanden. Allerdings soll hierauf verzichtet werden, da im Vordergrund das Abspielen und Verwalten von Musik steht und es deutlich einfachere und bessere Systeme gibt, mit Hilfe derer man kommunizieren kann.
\subsection{Suchen}
Eine einfache Textsuche zum finden von Titeln, Alben oder Interpreten innerhalb der Abspiellisten soll implementiert werden.
\begin{itemize}
	\item F\_0210 Der Benutzer kann nach Titeln suchen.
	\item F\_0220 Der Benutzer kann nach Alben suchen.
	\item F\_0230 Der Benutzer kann nach Interpreten suchen.
\end{itemize}
\section{Administrator-Funktionen}
Durch das Unix-artige System soll auch der Administrator-Zugriff geregelt werden. Sobald sich der Benutzer im Unix System als Administrator befindet, kann er auch den MPD-Client administrieren. Ein zusätzlicher Administrator-Modus muss also nicht implementiert werden.