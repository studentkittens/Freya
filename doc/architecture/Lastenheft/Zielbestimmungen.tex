\chapter{Zielbestimmungen}
Welche Ziele sollen durch den Einsatz der Software erreicht werden?\ \\

Dem einzelnen Benutzer soll das abspielen von Musik über eine Netzwerkverbindung ermöglicht werden, dabei soll die Steuerung von einem lokalen Client übernommen werden. Die Musik soll in einer zentralen Datenbank angelegt und über die Soundkarte eines Servers abgespielt werden. Die Client-Rechner sollen die Ausgabe steuern und Abspiellisten auf dem Server verwalten können.\ \\

Die Bedienung soll für alle Benutzer sehr einfach und komfortabel über einen lokalen Client realisiert werden. Bei jedem Start des Clients, soll die letzte Sitzung wiederhergestellt werden, falls keine Daten einer beendeten Sitzung gefunden werden, sollen Standardeinstellungen verwendet werden.\ \\

Standardmäßig sollen den Benutzern folgende Funktionen zur Verfügung stehen:

\renewcommand{\labelitemi}{•}
\begin{itemize}
	\item Abspielen von Musik
	\item Steuerung von Musik (Play, Stop, Skip, ...)
	\item Decodieren von Musik
	\item Input-Stream via HTTP
\end{itemize}

Weitere Funktionen müssen modular integrierbar sein, allerdings müssen sie noch nicht implementiert werden. Einige Beispiele für weitere Funktionen wären:

\begin{itemize}
	\item Finden von Album-Informationen
	\item Profil-Steuerung
	\item Visualisierung 
	\item etc....
\end{itemize}

Vorerst soll die Sprache der Software auf Deutsch beschränkt sein. Auch hier ist es möglich die Sprachen zu einem späteren Zeitpunkt modular zu erweitern.\ \\ \\

*******PLATZ FUER MEHR!!!**********