\chapter{Abhängigkeiten}

\section{Allgemeine Abhängigkeiten}

\renewcommand{\labelitemi}{•}
\begin{itemize}
	\item gtkmm-3.0
	\item libmpdclient
	\item libglyr
\end{itemize}

\section{Entwicklungsabhängigkeiten}
\begin{itemize}
	\item git (Versionsverwaltung)
	\renewcommand{\labelitemi}{--}
	\begin{itemize}
		\item git commit -a -m ("'Commit-Message"' Übertragen)
		\item git push (Änderungen auf den Github-Server laden)
		\item git pull (Änderungen von dem Github-Server laden)
	\end{itemize}
	\renewcommand{\labelitemi}{•}
	\item cmake (Buildsystem)
	\renewcommand{\labelitemi}{--}
	\begin{itemize}
		\item cmake . (Buildfiles erstellen)
		\item make (Kompilieren)
		\item sudo make install (Installieren)
	\end{itemize}
	\renewcommand{\labelitemi}{•}
	\item Editor nach Wahl (gVim, nano, Codeblocks, etc)
	\item Glade (Oberflächendesigner)
	\renewcommand{\labelitemi}{--}
	\begin{itemize}
		\item Erstellt ein XML File, kann von gtk geladen werden.
		\item Callbacks und Signale müssen im Code gehandelt werden.
	\end{itemize}
	\renewcommand{\labelitemi}{•}
	\item Primäre Entwicklerplattform: Linux
\end{itemize}

BEGRÜNDUNG!!!