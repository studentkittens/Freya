\newpage
\section{Testfälle}

\subsection{Testen der GUI}
Zum Testen der GUI des Freya Clients wurde ein Protokoll erstellt, welches den
jeweiligen Testfall, das erwartete Ereignis sowie das Resultat auflistet. Hierzu wird
eine Liste aller möglichen GUI Testfälle erstellt. Diese soll während der Implementierung 
und nach ,,Fertigstellung'' der Anwendung durchgangen werden.
\\
\\
Die Testfälle sollen desweiteren in einer ,,einfachen-'', ,,kombinierten-'', einer ,,mehrfach-''
Ausführung durchlaufen werden.
Durch dieses Verfahren können Programm-und Anwenderfehler erkannt und beseitigt werden.
Das Testen der GUI über ein Testframework wäre prinzipiell auch möglich, ist jedoch aus zeitlichen
Gründen, die man für die Einarbeitung und Konfiguration für so ein Framework benötigt, nicht realisierbar.
\\
\\
\textbf{Beispielauszug Testprotokoll ,,Einfache Ausführung'':}
\\
\begin{tabularx}{\textwidth}{|X|X|l|}
    \hline
    \textbf{Testfall} & \textbf{Erwartetes Ergebnis} & \textbf{Ergebnis eingetroffen?}\\
    \hline
    Remove & Ein Lied aus Queue entfernen & Ja/Nein?\\
    \hline
    Clear & Alle Lieder aus Queue entfernen & Ja/Nein?\\
    \hline
    Save as Playlist & Queue als Playlist speichern & Ja/Nein?\\
    \hline
    Suchen & Nach eingegebenem Wort suchen & Ja/Nein?\\
    \hline
\end{tabularx}


\subsection{Testen des ,,Backends''}

Um die eigentliche Anwendung zu Testen, soll das ,,Cxxtest Testframework''
verwendet werden. Durch den Einsatz eines Frameworks soll eine möglichst effiziente Methode des Testens bereitgestellt
und das Fehlerrisiko beim Testen auf ein Minimum reduziert werden. Jedes Modul soll eine eigene ,,Testsuite'' bekommen und die
Ausführung soll über CMake automatisiert werden. Abarbeitung durch ein ,,test'' Target, sodass einfach 'make test' gestartet werden kann.
Die TestSuite Dateien bestehen aus einem Header der sich nach folgenden Konventionen richten soll: 

\begin{itemize}
	\item File- bzw Headername der jeweiligen TestSuite soll nach dem Muster \verb+check_<modulname>.hh+ aufgebaut werden
	\item Klassenname der jeweiligen Testsuite \verb+ <Klassenname>TestSuite+
	\item Die TestSuite muss die TestSuite Cxxtest sowie das zu testende Modul einbinden
	\item Die TestSuite muss von CxxTest::TestSuite abgeleitet werden 
	\item Der Name der zu testenden Funktion soll nach dem Muster \verb+void test<func_name(void)+ aufgebaut werden
\end{itemize}  


Eine TestSuite soll wie folgt aussehen:


\begin{lstlisting}[caption={TestSuite Template check\_notify.hh Beispielcodefragment für die Klasse ,,Notify''},label={lst:cxxtemp}]
 #include <cxxtest/TestSuite.h>
 #include "Notify.hh"
 
 class NotifyTestSuite : public CxxTest::TestSuite
 {
  
 public:
  
      void testsend_big( void )
      {
          /* insert testcode here */
 	  
 	  ...
 	   
          /* check if result is valid */    
          TS_ASSERT( ... );
      }
   
 private:
      /* data */
 };
\end{lstlisting}


\section{Doxygen}
Als interne Entwicklerdokumentation soll das Tool Doxygen\footnote{http://www.stack.nl/~dimitri/doxygen/} verwendet werden.
Durch den Einsatz eines Dokumentationstools wird für die Entwicklern innerhalb des Teams eine zentrale interne Dokumentation geschaffen,
auf die jederzeit zugegriffen werden kann. 
\emph{Literate programming}\footnote{http://en.wikipedia.org/wiki/Literate\_programming} dient laut Prof. Donald E. Knuth\footnote{http://de.wikipedia.org/wiki/Donald\_Ervin\_Knuth} 
es erstklassige Dokumentation weil sie den Entwickler dazu zwingt genau über sein Software Design nachzudenken, wodurch
Fehlentscheidungen automatisch minimiert und die Code Qualität verbessert werden.
