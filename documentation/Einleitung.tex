\chapter{Einleitung}
Ziel dieser Studienarbeit ist die vollständige Bearbeitung einer vorgegebenen Aufgabenstellung nach einem selbst gewählten Vorgehensmodell. Die Aufgabenstellung schreibt vor, sich in einer Gruppe zusammen zu finden und gemeinsam ein Software-Projekt zu bearbeiten und dabei strukturiert und professionell vorzugehen.\\

\begin{quote}
\section{Rahmenbedingungen}

\renewcommand{\labelitemi}{•}
\begin{itemize}
	\item Persistente Datenspeicherung (Datei oder Datenbank)
	\item Netzwerk Programmierung (Verteilte Architektur z.B. Client-Server)
	\item Grafisches-User-Interface (Swing, Web-basiert,...)
\end{itemize}

\section{Prozess-Anforderungen}

\begin{itemize}
	\item Dokumentation aller Phasen(Analyse bis Testen)
	\item Auswahl eines konkreten Prozessmodells (Begründung der Wahl)
	\item Erstellung der Dokumente und UML-Diagramme (Freie Wahl der Werkzeuge)
	\item Fertige Implementierung (Es kann mehr spezifiziert sein als implementiert)
	\item Spezifikation von Testszenarien (Und Beleg der erfolgreichen Ausführung)
	\item Lauffähiges System 
\end{itemize}

\footnote{Folie Anforderungen, Autor Philipp Schaible, WS 2011/2012, Inf 3}
\end{quote}

Diese Arbeit ist wichtig, um den Studenten zu zeigen, wie man in einem Team zusammenarbeitet und nach Software-Engineering-Methoden qualitativ hochwertige Software erstellt.
Es geht im Folgenden um einen Music-Player-Daemon-Client (Näheres bitte der Definition entnehmen). Dieses Thema wird behandelt, da es alle Rahmenbedingungen abdeckt und im Interesse der Autoren liegt. Die Besonderheit liegt darin, dass sich diese Software nach Fertigstellung auch wirklich anwenden lässt. Ziel ist die Erweiterung der Fähigkeiten im Bereich des Software Engineering sowie das Erlernen von Methoden für wissenschaftliches Arbeiten.
