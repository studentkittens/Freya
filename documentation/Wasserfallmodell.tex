\chapter{Wasserfallmodell mit Rücksprung}

\section{Definition}

\begin{quote}
Das Wasserfallmodell ist ein lineares (nicht iteratives) Vorgehensmodell in der Softwareentwicklung, bei dem der Softwareentwicklungsprozess in Phasen organisiert wird. Dabei gehen die Phasenergebnisse wie bei einem Wasserfall immer als bindende Vorgaben für die nächsttiefere Phase ein.\\

Im Wasserfallmodell hat jede Phase vordefinierte Start- und Endpunkte mit eindeutig definierten Ergebnissen. In Meilensteinsitzungen am jeweiligen Phasenende werden die Ergebnisdokumente verabschiedet. Zu den wichtigsten Dokumenten zählen dabei das Lastenheft sowie das Pflichtenheft. In der betrieblichen Praxis gibt es viele Varianten des reinen Modells. Es ist aber das traditionell am weitesten verbreitete Vorgehensmodell.\\

Der Name „Wasserfall“ kommt von der häufig gewählten grafischen Darstellung der fünf bis sechs als Kaskade angeordneten Phasen.
Ein erweitertes Wasserfallmodell mit Rücksprungmöglichkeiten (gestrichelt).\\

Erweiterungen des einfachen Modells (Wasserfallmodell mit Rücksprung) führen iterative Aspekte ein und erlauben ein schrittweises „Aufwärtslaufen“ der Kaskade, sofern in der aktuellen Phase etwas schieflaufen sollte, um den Fehler auf der nächsthöheren Stufe beheben zu können.\\

\footnote{Zitat aus:  http://de.wikipedia.org/wiki/Wasserfallmodell}
\end{quote}

\begin{figure}[h]
\centering
\includegraphics[scale=0.35]{567px-Wasserfallmodell.png}
\end{figure}
\footnote{Wasserfallmodell mit Rücksprung, Bild-Quelle: http://upload.wikimedia.org/wikipedia/commons/thumb/e/e5/Wasserfallmodell.svg/567px-Wasserfallmodell.svg.png}

\section{Warum dieses Modell?}
Wir haben uns für das Wasserfallmodell mit Rücksprung entschieden, weil dieses Modell alle Phasen der Entwicklung klar abgrenzt und sich optimal auf einen professionellen Softwareentwicklungsvorgang abbilden lässt.
Dieses Modell ermöglicht eine klare Planung und Kontrolle unseres Softwareprojekts, da die Anforderungen stets die gleichen bleiben und der Umfang einigermaßen gut abschätzbar ist.\\

Für die erweiterte Version dieses Modells, nämlich mit Rücksprung, haben wir uns entschieden, um ein paar Nachteile dieses Modells auszuhebeln. Beispielsweise sind die klar voneinander abgegrenzten Phasen in der Realität oft nicht umsetzbar. Des weiteren sind wir somit flexibler gegenüber Änderungen.
